\secrel{Генераторные функции и yield}\label{yield}

Ключевое слово \file{yield}\ в \py\ превращает любую функцию, в которой оно
используется, в функцию-\term{генератор}. Вызов генератора вместо
выполнения функции возвращает объект-\term{итератор}. Если его использовать в
качестве параметра цикла \file{for}, или явно вызывать встроенй метод
\verb|__next__()|, то вы сможете использовать \term{ленивые вычисления}\ в
обычной императивной программе на \py.

\begin{quotation}\noindent
\term{Ленивые вычисления} (англ. lazy evaluation, также отложенные вычисления)\
--- применяемая в некоторых (функциональных) языках программирования стратегия
вычисления, согласно которой вычисления следует откладывать до тех пор, пока не
понадобится их результат.
\end{quotation}

В рамках \py\ полная реализация ленивый вычислений недоступна \ref{lazy}, тем не
менее использование генераторов позволяет вычислять функции в бесконечном цикле,
возвращая промежуточные результаты. Также на генераторных функциях построен
механизм \term{логического вывода в возвратами}, используемый в языке \prolog,
который мы рассмотрим далее.

