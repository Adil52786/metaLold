\secrel{PLY: библиотека синтаксического разбора для \py}\label{ply}\secdown

\clearpage
Библиотека PLY позволяет писать \term{парсеры} для достаточно сложных языков.
Если вам для какой-то задачи потребуется применение \term{инфиксного
синтаксиса}, типа разбора арифметических выражений, вы сможете без особых усилий
добавить для них \term{синтаксический анализатор}.

\begin{description}
\item[Лексер] обрабатывает \term{входной поток} единичных символов из файла или
строки, группируя их в \term{токены}. Каждый токен имеет тип \verb|.type|,
значение \verb|.value|, и дополнительные поля типа имени файла исходного кода,
или позиции токена (строка, столбец).
\item[Парсер синтаксиса]\ читает \term{поток токенов}, или работает напрямую с
символами и текстовыми строками, в зависимости от того, какие алгоритмы
разоработа используются.
\end{description}

\clearpage
\lst{syntax/ply/lex.py}{language=Python}
\lst{syntax/ply/yacc.py}{language=Python}

\term{Парсер} может состоять из обоих компонентов, или только из одного, в
зависимости от сложности синтаксиса входного языка, и того, нужно ли нам
\term{распознавать} рекурсивно вложенные синтаксические конструкции, или
достаточно только определить тип \term{лексем} (для подстветки синтаксиса).

\clearpage
Каждое правило лексера задается в виде функции, её docstring задает регулярное
выражение, которому должна удовлетворять группа символов, чтобы быть
распознанной. Функция получает на вход параметр \verb|t| содержащий
\textit{состояние лексера}; \verb|t.value| содержит распознанную группу символов
в виде строки, которую мы возвращаем из функции через вызов конструктора фрейма
соответствующего типа.

\bigskip
\noindent
\verb|ply.lex.lex()| проходит по исходному коду текущего модуля \py, находит
функции соответствующие шаблону правил лексера, и синтезирует функцию-лексер.



\secrel{\F\ лексер}\label{plyforth}

Несмотря на то что диалекты \F/\pyf\ требуют только реализацию \term{лексера},
есть смысл немного сэкономить усилия, и не заморачиваться с написанием
традиционного посимвольного разбора ``до пробела''. Еще одно достоинство
использования PLY: с ее помощью мы можем автоматически определять тип для каждой
лексемы, в частности разпознавать примитивные типы \ref{prim}\ и вызывать
соответствующие конструкторы.

\begin{description}%[nosep]
\item[tokens] список токенов, которые может распознавать парсер\\
так как мы специально обеспечили возможность использования фрей\-мов-примитивов
в качестве \term{литералов}, в этом списке должны быть перечислены тэги (c
маленькой буквы)
\item[t\_ignore] символы которые не будут участвовать в разборе (пробелы) 
\item[t\_number()] правило лексера распознающее числа
\item[t\_symbol()] правило распознающее символы как имена форт-слов
\item[t\_error()] обработка синтаксических ошибок: нераспознанные символы
\end{description}

\secup
