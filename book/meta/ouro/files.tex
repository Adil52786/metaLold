% \clearpage
\secrel{Файлы проекта}\label{circfiles}

% \lst{meta/files.ini}{}
% \lst{meta/eclfiles.ini}{}
% \lst{meta/ebldfiles.ini}{}
% \lst{meta/vimfiles.ini}{}

Начиная новый программный проект, мы каждый раз снова и снове делаем одни и те
же действия, даже используя визард в вашей любимой IDE: создать структуру
каталогов, прописать маски производных файлов для git, задать
мультиплатформенную кодировку для файлов и т.д. При этом от проекта к проекту
некоторые файлы немного меняются, особенно это относится к Makefile и другим
файлам, хранящим список компилируемых модулей или опции компиляции.
Для встраиваемых систем самым геморным будет постоянно отслеживать список файлов
и своевременно обновлять Makefile. И в итоге, для
автогенерируемого кода нам нужно автоматизировать и создание Makefile отслеживая
зависимости между множеством файлов.

