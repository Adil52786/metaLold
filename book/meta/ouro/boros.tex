\secrel{Ouroboros: минималистичный форк \metal}\label{ouro}\secdown

\noindent
\begin{tabular}{l p{7.7cm}}
\multirow{1}{*}{\fig{meta/ouro/boros.png}{height=.5\textheight}} &

\url{https://github.com/ponyatov/Ouroboros}

\medskip
\noindent
В отдельный проект была выделена специальная минималистичная версия \metal:
никакой документации, всего несколько файлов, и функциональность необходимая
только для трансляции метамодели на \metal\ в интерпретатор на \py.

{\scriptsize (*) это Python который ест сам себя. это хвост}\\
\end{tabular}
\medskip

\noindent
Весь ввод/вывод ограничен только выводом строк на консоль, для экспериментов
доступна текстовая REPL-консоль, доступ к которой можно получить в окне работы
интерпретатора \py\ в \eclipse. Для работы под \win\ сделана легкая адаптация
для запуска сессии в \vim, что добавило несколько файлов, загромождающих каталог
проекта.

\clearpage
\emph{Запись в файлы не поддерживается}: вы должны вручную скопировать
куски исходного кода из консоли, вручную проверяя их корректность при замене в
коде интерпретатора.

Смысл отдельного проекта в том, чтобы выделить минимальное самодостаточное
\term{ядро \metal}, способное оттранслировать только самого себя, не таская всю
мощь большой системы, и \emph{зафиксировать} это ядро. Маленький объем кода
упрощает изучение, интеграцию в другие системы, и портирование интерпретатора на
другие языки. Одноременно метамодель, написанная на \metal, становится
\term{формальной спецификацией языка}, так как она не пререгружена фичами,
библиотеками и другой информацией.

\clearpage
\secrel{Метаинформация о программном модуле}\label{ouroinfo}

\lst{meta/ouro/info.ml}{}
\clearpage
\secrel{Файлы проекта}\label{circfiles}

% \lst{meta/files.ini}{}
% \lst{meta/eclfiles.ini}{}
% \lst{meta/ebldfiles.ini}{}
% \lst{meta/vimfiles.ini}{}

Начиная новый программный проект, мы каждый раз снова и снове делаем одни и те
же действия, даже используя визард в вашей любимой IDE: создать структуру
каталогов, прописать маски производных файлов для git, задать
мультиплатформенную кодировку для файлов и т.д. При этом от проекта к проекту
некоторые файлы немного меняются, особенно это относится к Makefile и другим
файлам, хранящим список компилируемых модулей или опции компиляции.
Для встраиваемых систем самым геморным будет постоянно отслеживать список файлов
и своевременно обновлять Makefile. И в итоге, для
автогенерируемого кода нам нужно автоматизировать и создание Makefile отслеживая
зависимости между множеством файлов.



\secup
