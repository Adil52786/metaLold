\secrel{Раскрутка языка (bootstrap)}\label{circ}\secdown

В этой книге нам нужно показать всю мощь языка специально заточенного под
метапрограммирования. Лучшим способом для этого является его \term{bootstrap},
или \term{раскрутка}: написать \term{метациркулярную} реализацию языка
программирования\ --- \emph{на нем самом}.

\secrel{Метациркулярный интерпретатор}

\begin{quotation}
Метациркулярный интерпретатор является интерпретатором, написанным в (возможно,
более базовой) реализации того же языка. Обычно это делается для того, чтобы
экспериментировать с добавлением новых функций на язык или созданием другого
диалекта.
\end{quotation}

В целях демонстрации того, как работает язык программирования, в литературе
часто применяют этот метод: некоторые части интерпретатора описываются на том же
языке программирования. Это позволяет не только показать внутреннее устройство,
но и служит реальным примером применения.

Если в комплект поставки включить полную метациркулярную реализацию языка,
пользователь также может адаптировать язык под свои нужды, или написать свой
клон, но для этого должно выполняться одно очень важное, критическое условие\
--- \emph{документация должна поставляться} не как руководство
пользователя, а \emph{как учебник по написанию собственной версии языка}.

\bigskip
Понять метациркулярность компилятора очень просто: у нас есть исходный код
компилятора для некоторого языка программирования, и исполняемый файл этого
компилятора, оба версии N. Исходный код модифицируется, подается на вход
\file{компилятора-N}, в результате получем исполняемый код
\file{компилятора-N+1}. Для тестирования новой версии мы еще раз подаем
исходный код N+1 на вход \file{компилятора-N+1}, и он собирает сам себя. Такой
способ в частности применяется при сборке GCC из GNU Compilers Collection.

\bigskip
Для интерпретаторов динамических языков используется другой способ \cite{plai},
похожий на то как мы написали всю внутреннюю механику \hico\ на \py: 

\secup