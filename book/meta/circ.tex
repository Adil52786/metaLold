\secrel{Раскрутка языка (bootstrap)}\label{circ}\secdown

В этой книге нам нужно показать всю мощь языка специально заточенного под
метапрограммирования. Лучшим способом для этого является его \term{bootstrap},
или \term{раскрутка}: написать \term{метациркулярную} реализацию языка
программирования\ --- \emph{на нем самом}.

\secrel{Метациркулярный интерпретатор}

\begin{quotation}
Метациркулярный интерпретатор является интерпретатором, написанным в (возможно,
более базовой) реализации того же языка. Обычно это делается для того, чтобы
экспериментировать с добавлением новых функций на язык или созданием другого
диалекта.
\end{quotation}

В целях демонстрации того, как работает язык программирования, в литературе
часто применяют этот метод: некоторые части интерпретатора описываются на том же
языке программирования. Это позволяет не только показать внутреннее устройство,
но и служит реальным примером применения.

Если в комплект поставки включить полную метациркулярную реализацию языка,
пользователь также может адаптировать язык под свои нужды, или написать свой
клон, но для этого должно выполняться одно очень важное, критическое условие\
--- \emph{документация должна поставляться} не как руководство
пользователя, а \emph{как учебник по написанию собственной версии языка}.

\bigskip
Понять метациркулярность \textit{компилятора} очень просто: у нас есть исходный
код компилятора для некоторого языка программирования, и исполняемый файл этого
компилятора, оба версии N. Исходный код модифицируется, подается на вход
\file{компилятора-N}, в результате получем исполняемый код
\file{компилятора-N+1}. Для тестирования новой версии мы еще раз подаем исходный
код N+1 на вход \file{компилятора-N+1}, и он собирает сам себя. Такой способ в
частности применяется при сборке GCC из GNU Compilers Collection.
Собственно говоря, это единственный способ написать самодостаточный компилятор
такого системного языка как \emc: стартовая версия компилируется другим
(коммерческим) компилятором (раньше писали на ассемблере), а затем
происходит \term{раскрутка компилятора}.

\clearpage
Для интерпретаторов динамических языков используется другой способ \cite{plai},
похожий на то как мы написали всю внутреннюю механику \metal\ на \py:
реализация нового языка-N+1 представляется в виде множества явно выделенных
структур данных, которые интерпретируются исполняющим кодом на языке-N.

\bigskip
Метапрограммирование, а точнее \term{кодогенерация}, предлагает еще одну
альтернативу: ядро интерпретатора, написанное на языке \py, выполняет
метапрограмму на языке \metal, который \emph{генерирует исходный код} реализации
интерпрератора на языке Java\note{и прочее барахло в файлах проекта для
Android}, которое в итоге будет работать как интерпретатор языка \F\ на
мобильном телефоне. Это самый обкуренный пример, специально переусложненный
для демонстрации, на самом деле бы пока ограничимся только цепочкой
\py$\rightarrow$\metal$\rightarrow$\py$\rightarrow$\metal$_{N+1}$.

\secrel{Метамодель языка \metal\ с генерацией кода}

\metal\ запускается с загрузкой инициализационного файла \file{metaL.ini}
\ref{ini}, который по умолчанию содержит полную модель системы и множество
других определений (для встроенного программирования, и для самораскрутки
системы). Запустив систему, вы получаете возможность делать \term{разработку
через клонирование}: при контрактной разработке передается система \metal,
дополненная функционалом, заказанным клиентом.

Такой подход особенно хорош если у вас множество заказчиков, которым вы
поставляете примерно одну и ту же систему, но с различными модификациями. Вы
можете наследовать значительные блоки исходного кода и дизайна\note{структуры
данных, реализация алгоритмов, компоненты, документацию \ref{doc},..},
прописывая для каждого клиента только те блоки, которые вы ему поставляете, а
кодогенерация сама будет отслеживать обновления и зависимости.

\clearpage
\lst{meta/header.ini}{language=Python}
Здесь вы сразу видите два варианта \term{строчных комментариев}: в стилях \py\ и
\F. Одновременно \file{metaL.ini} также является последним этапом
интеграционного тестирования, проверяющим работу всех фич языка.

\secrel{Ouroboros: минималистичный форк \metal}\label{ouro}\secdown

\noindent
\begin{tabular}{l p{7.7cm}}
\multirow{1}{*}{\fig{meta/ouro/boros.png}{height=.5\textheight}} &

\url{https://github.com/ponyatov/Ouroboros}

\medskip
\noindent
В отдельный проект была выделена специальная минималистичная версия \metal:
никакой документации, всего несколько файлов, и функциональность необходимая
только для трансляции метамодели на \metal\ в интерпретатор на \py.

{\scriptsize (*) это Python который ест сам себя. это хвост}\\
\end{tabular}
\medskip

\noindent
Весь ввод/вывод ограничен только выводом строк на консоль, для экспериментов
доступна текстовая REPL-консоль, доступ к которой можно получить в окне работы
интерпретатора \py\ в \eclipse. Для работы под \win\ сделана легкая адаптация
для запуска сессии в \vim, что добавило несколько файлов, загромождающих каталог
проекта.

\clearpage
\emph{Запись в файлы не поддерживается}: вы должны вручную скопировать
куски исходного кода из консоли, вручную проверяя их корректность при замене в
коде интерпретатора.

Смысл отдельного проекта в том, чтобы выделить минимальное самодостаточное
\term{ядро \metal}, способное оттранслировать только самого себя, не таская всю
мощь большой системы, и \emph{зафиксировать} это ядро. Маленький объем кода
упрощает изучение, интеграцию в другие системы, и портирование интерпретатора на
другие языки. Одноременно метамодель, написанная на \metal, становится
\term{формальной спецификацией языка}, так как она не пререгружена фичами,
библиотеками и другой информацией.

\clearpage
\secrel{Файлы проекта}\label{circfiles}

% \lst{meta/files.ini}{}
% \lst{meta/eclfiles.ini}{}
% \lst{meta/ebldfiles.ini}{}
% \lst{meta/vimfiles.ini}{}

Начиная новый программный проект, мы каждый раз снова и снове делаем одни и те
же действия, даже используя визард в вашей любимой IDE: создать структуру
каталогов, прописать маски производных файлов для git, задать
мультиплатформенную кодировку для файлов и т.д. При этом от проекта к проекту
некоторые файлы немного меняются, особенно это относится к Makefile и другим
файлам, хранящим список компилируемых модулей или опции компиляции.
Для встраиваемых систем самым геморным будет постоянно отслеживать список файлов
и своевременно обновлять Makefile. И в итоге, для
автогенерируемого кода нам нужно автоматизировать и создание Makefile отслеживая
зависимости между множеством файлов.



\secup


\secup