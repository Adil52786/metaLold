\addcontentsline{toc}{chapter}{Литература}
\begin{thebibliography}{99}

\clearpage
\bibitem{py}\ \bibfig{bib/python.jpg}\\
\textbf{Язык программирования Python}\\
Россум, Г., Дрейк, Ф.Л.Дж., Откидач, Д.С.,\ldots\\
\url{http://rus-linux.net/MyLDP/BOOKS/python.pdf}

\clearpage
\bibitem{dragon2} \bibfig{bib/dragon2.png}\ \emph{Purple Dragon Book /2nd ed/}\\
\textbf{Компиляторы: принципы, технологии и инструментарий} 2 изд.\\
Альфред В. Ахо, Моника С. Лам, Рави Сети, Джеффри Д. Ульман.\\
М.: Вильямс, 2008.\\ ISBN 978-5-8459-1349-4.\\
\url{https://www.ozon.ru/context/detail/id/148568229/}

\clearpage
\bibitem{serlex} \bibfig{bib/serlex.jpg}\\
\textbf{Теория и реализация языков программирование}\\
В.А.Серебряков, М.П.Галочкин, Д.Р.Гончар, М.Г.Фуругян\\ 
\url{http://trpl7.ru/t-books/TRYAP_BOOK_Details.htm}\\
pdf: \url{http://trpl7.ru/t-books/_TRYAPBOOK_pdf.pdf}

\clearpage
\bibitem{sicp} \bibfig{bib/sicp.jpg}\ \textbf{\emph{SICP}\\
\href{https://drive.google.com/file/d/0B0u4WeMjO894X3lnWmhjUktKRk0/view?usp=sharing}{Структура
и интерпретация компьютерных программ}}\\
Харольд Абельсон, Джеральд Сассман\\
ISBN 5-98227-191-8\\
EN: \url{web.mit.edu/alexmv/6.037/sicp.pdf}\\
\url{https://www.ozon.ru/context/detail/id/148537956/}

\clearpage
\bibitem{bratko}\ \bibfig{bib/bratko.jpg}\\
\textbf{Программирование на языке Пролог\\для искусственного интеллекта}\\
Иван Братко\\
Мир, 1990\\ ISBN 5-03-001425-Х, 0-201-14224-4

\clearpage
\bibitem{minsky}\ \bibfig{bib/minsky.jpg}\\
\textbf{Фреймы для представления знаний}\\
Марвин Минский\\
 М.: Мир, 1979.\\
\url{https://royallib.com/book/minskiy_marvin/freymi_dlya_predstavleniya_znaniy.html}\\
\url{https://www.ozon.ru/context/detail/id/31747338/}

\clearpage
\bibitem{starting}\ \bibfig{bib/starting.jpg}\\
\textbf{Начальный курс программирования на языке ФОРТ}\\
Лео Броуди \\
М.: Финансы и статистика, 1990. - 352 с.  ISBN 5-279-00262-6.\\
Пер. с англ. В.А.Кондратенко. Под ред. Б.А.Кацева, В.А.Кириллина. Предисловие
И.В.Романовского.\\
\url{http://www.nncron.ru/download/sf.pdf}

\clearpage
\bibitem{thinking}\ \bibfig{bib/thinking.jpg}\\
\textbf{Способ мышления\ --- ФОРТ}\\
Лео Броуди \\
\href{http://www.forth.org.ru/~cactus/files/brodie.rar}{OCR/перевод}\\
\url{http://thinking-forth.sourceforge.net/}

\clearpage
\bibitem{kelly}\ \bibfig{bib/kelly.jpg}\\
\textbf{Язык программирования ФОРТ}\\
М.Келли, Н.Спайс\\
Москва, ``Радио и связь'', 1993\\
\href{http://www.forth.org.ru/~cactus/files/kelly.rar}{OCR/перевод}

\clearpage
\bibitem{baranov}\ \bibfig{bib/baranov.png}\\
\textbf{Язык Форт и его реализации}\\
С.Н. Баранов, Н.Р. Ноздрунов\\
``Машиностроение" Ленинградское отделение, 1988\\
\href{http://www.forth.org.ru/~cactus/files/baranov2.rar}{OCR/перевод}

\clearpage
\bibitem{wikicc}\ \bibfig{bib/wikicc.png}\\
\textbf{Book: Compiler Construction}\\
(by) Wikipedians\\
\url{https://en.wikipedia.org/wiki/Book:Compiler_construction}

\clearpage
\bibitem{plai}\ \bibfig{bib/plai.jpg}\\
\textbf{Programming Languages: Application and Interpretation}\\
Shriram Krishnamurthi\\
\url{https://cs.brown.edu/~sk/Publications/Books/ProgLangs/2007-04-26/}

\clearpage
\bibitem{tyugu}\ \bibfig{bib/tyugu.jpg}\\
\textbf{Концептуальное программирование}\\
Энн Харальдович Тыугу\\
М.: Наука, 1984. 255 с\\
\ \\
ВНТК "СТАРТ"\ \url{http://start.iis.nsk.su}

\clearpage
\bibitem{moskvitin1}\ \bibfig{bib/moskvitin1.jpg}\\
\textbf{Решение задач на компьютерах: учебное пособие, ч.1\\
Постановка (спецификация) задач }\\
Москвитин А. А.\\

\clearpage
\bibitem{moskvitin2}\ \bibfig{bib/moskvitin2.jpg}\\
\textbf{Решение задач на компьютерах: учебное пособие, ч.2\\
Разработка программных средств}\\
Москвитин А. А.\\

\clearpage
\bibitem{psicc2}\ \bibfig{bib/psicc2.png}\\
\textbf{Practical UML Statecharts in C/C++, 2nd Edition:\\
Event-Driven Programming for Embedded Systems}\\
Quantum Leaps' Miro Samek\\
\url{http://www.state-machine.com/psicc2/}

\clearpage
\bibitem{andeitel}\ \bibfig{bib/andeitel.jpg}\\
\textbf{Android для разработчиков} 3е издание\\
Пол Дейтел, Харви Дейтел, Александер Уолд\\
\url{https://www.ozon.ru/context/detail/id/136331151/}

\clearpage
\bibitem{triska}\ \\
\textbf{The Power of Prolog}\\
Markus Triska\\
\url{https://www.metalevel.at/prolog}

\clearpage
\bibitem{lamot}\ \bibfig{bib/lamot.jpg}\\
\textbf{Секреты программирования игр}\\
Андре Ла Мот, Ратклифф Д., Семинаторе М., Тайлер Д.\\
СПб: Питер, 1995. — 718 с. — 5-88782-037-3.\\
\url{https://3dgame-creator.ru/sekrety-programmirovaniya-igr/}

\end{thebibliography}
