\addcontentsline{toc}{section}{Литература}
\begin{thebibliography}{99}

\clearpage
\bibitem{py}\ \bibfig{bib/python.jpg}\\
\textbf{Язык программирования Python}\\
Россум, Г., Дрейк, Ф.Л.Дж., Откидач, Д.С.,\ldots\\
\url{http://rus-linux.net/MyLDP/BOOKS/python.pdf}

\clearpage
\bibitem{dragon2} \bibfig{bib/dragon2.png}\ \emph{Purple Dragon Book /2nd ed/}\\
\textbf{Компиляторы: принципы, технологии и инструментарий} 2 изд.\\
Альфред В. Ахо, Моника С. Лам, Рави Сети, Джеффри Д. Ульман.\\
М.: Вильямс, 2008.\\ ISBN 978-5-8459-1349-4.\\
\url{https://www.ozon.ru/context/detail/id/148568229/}

\clearpage
\bibitem{sicp} \bibfig{bib/sicp.jpg}\ \textbf{\emph{SICP}\\
\href{https://drive.google.com/file/d/0B0u4WeMjO894X3lnWmhjUktKRk0/view?usp=sharing}{Структура
и интерпретация компьютерных программ}}\\
Харольд Абельсон, Джеральд Сассман\\
ISBN 5-98227-191-8\\
EN: \url{web.mit.edu/alexmv/6.037/sicp.pdf}\\
\url{https://www.ozon.ru/context/detail/id/5322055/}

\clearpage
\bibitem{bratko}\ \bibfig{bib/bratko.jpg}\\
\textbf{Программирование на языке Пролог\\для искусственного интеллекта}\\
Иван Братко\\
Мир, 1990\\ ISBN 5-03-001425-Х, 0-201-14224-4

\clearpage
\bibitem{minsky}\ \bibfig{bib/minsky.jpg}\\
\textbf{Фреймы для представления знаний}\\
Марвин Минский\\
 М.: Мир, 1979.\\
\url{https://royallib.com/book/minskiy_marvin/freymi_dlya_predstavleniya_znaniy.html}\\
\url{https://www.ozon.ru/context/detail/id/31747338/}

\clearpage
\bibitem{starting}\ \bibfig{bib/starting.jpg}\\
\textbf{Начальный курс программирования на языке ФОРТ}\\
Лео Броуди \\
М.: Финансы и статистика, 1990. - 352 с.  ISBN 5-279-00262-6.\\
Пер. с англ. В.А.Кондратенко. Под ред. Б.А.Кацева, В.А.Кириллина. Предисловие
И.В.Романовского.\\
\url{http://www.nncron.ru/download/sf.pdf}

\end{thebibliography}
