\secrel{\A\ органайзер}\label{android}\secdown

\noindent
\begin{tabular}{l|p{8.3cm}}
\tfig{android/android_plan.png}{height=.45\textheight} &
Мобильниый телефон\ --- компьютер, который всегда с собой. Но \emph{удобных
средств программирования}, работающих на мобильном \textit{телефоне},
практически нет.
Разработка под \A\ и реализация on-device системы программирования\ --- тема
отдельной большой книги. Для начала можно попробовать написать \emph{органайзер,
программируемый пользоваталем}
\\ \end{tabular}

\medskip\noindent
На планшете есть несколько сред разработки для \emc, \py\ и \js, но на
телефоне они совершенно неюзабельны: сказывается ограничение размеров экрана, и
для управления зачастую доступен всего один палец\note{попробуйте
поредактивровать код на \py\ вися на одной руке в транспорте, двухчасовой
пробке, или на велосипеде в гамаке}. Полностью графический интерфейс тоже часто
оказывается неудобным\ --- лично меня задолбали ёбдизайнеры, из-за которых
приходиться зумать.

\clearpage
\begin{description}
\item[PIM] Personal Information Manager

Функции, выполняемые органайзером\note{персональным информационным
менеджером}:
\begin{itemize}[nosep]
  \item 
\emph{планирование задач} для контроля за их самостоятельным и сторонним
выполнением (ToDo list, task-трекер, мобильный CRM);
  \item 
планирование событий, привязанных к датам и времени (праздники или встречи);
  \item 
\emph{напоминальники и зудильники} об определённых пользователем событиях;
  \item 
управление контактами (адресно-телефонная книга);
  \item 
записная книжка и листки-липучки;
  \item 
личные записи (дневник);
  \item 
интеграция с электронной почтой и мессенджерами
  \item 
\emph{персональная база знаний}.
\end{itemize}

\item[PPS] Personal Planning System, система персонального планирования\\
специально заточенная на трекинг задач, с точки зрения конкретного
человека.
\end{description}

\noindent
Несмотря на десятки лет усилий, даже такие гиганты как Google и Microsoft не
смогли решить проблему создания полноценного органайзера, в который by design
должен был превратиться смартфон:
\begin{itemize}
  \item доступны только примитивные типы задач, при этом на практике нужно
  множество вариантов, от простого будильника, встречи, действий привязанных по
  месту, периодические задачи с разным масштабом\note{от 30 минут для отдыха
  глаз, до года для дней рожденья}, до задач характерных для систем
  groupware
  \item полностью отсутствует функционал трекинга проектов: групповые
  задачи, зависимости задач по времени, исполнителям и ресурсам, делегирование и
  контроль,\ldots
  \item отсутствие средств индивидуальной адаптации, включая средства
  программирования пользователем, и доступ к внешним приложениям, библиотекам и
  сенсорам
\end{itemize}

\noindent
Для таких внутренне сложных приложений как органайзер, планировщик или трекер
задач, подход традиционных приложений с пользовательским GUI не подходит. Чем
больше автор приложения усложняет его, добавляя все новые и новые функции, тем
сложнее для пользователя становится его освоение. При этом даже для
очень переусложненного органайзера обязательно встретится случай использования,
под который не подходит ни один из предусмотренных в приложении вариантов.
Например потребуется задача, которая должна срабатывать одновременно по времени,
местоположению, и условию выполнения другой задачи, назначеная другому
пользователю в вашей рабочей группе.

Если наборот в дизайне органайзера пойти от архитектуры, построенной на
минимальном ядре, расширяемом пользователем через написание скриптов, мы снижаем
начальный порог вхождения (за счет уменьшения необходимых усилий для освоения),
одновременно предоставляя пользователю механизм персональной адаптации.

\medskip\noindent
\begin{tabular}{l p{8.7cm}}
\tfig{android/basact.png}{height=.6\textheight} &
Для создания проекта в Android Studio лучше всего подходит Basic Activity, так
как из коробки в графическом интерфейсе приложения есть элемент добавления новой
задачи.

\medskip
\begin{tabular}{l l}
Name & \metal \\
Package name & \file{io.github.ponyatov.metal} \\
Save location & \file{/home/ponyatov/metaL/Android} \\
Language & Java \\
Minimum API Level & API14\\
\end{tabular}
\\
\end{tabular}

\bigskip\noindent
\begin{tabular}{l p{9.5cm}}
&\\
\tfig{android/planning.png}{height=.2\textheight} &
Для начала стоит заменить иконку приложения, также создав комплект round icon
для совместимости с новыми версиями \A\ использующими ``пузырчатый'' интерфейс.
\\
\end{tabular}

\bigskip\noindent
Для подготовки иконок нужно иметь некоторые навыки для работы с графикой и
каким-то из графических редакторов, типа \href{https://www.gimp.org}{GIMP}. Если
у вас уже есть черновик иконки, можете воспользоваться визардом по адресу\\
\url{http://jgilfelt.github.io/AndroidAssetStudio/icons-launcher.html}\\
Загрузив в него черновик иконки, и включив эффект Shape:Bewel, вы сможете
скачать \file{ic\_launcher.zip} содержащий готовый набор иконок для разных
разрешений экрана. Копировать их в проект придется по одному файлу вручную, так
как в новых версиях Android Studio изменилась схема именования каталогов с
ресурсами: с \file{drawable\_}\ на \file{mipmap\_}.

\bigskip
\begin{lstlisting}
~/metaL$ find Android/ -type f \
	-regex .+ic_launcher.png$ -exec file {} +
	
Android/app/src/main/res/mipmap-xhdpi/ic_launcher.png:   
	PNG image data, 96 x 96, 8-bit/color RGBA, non-interl
Android/app/src/main/res/mipmap-xxhdpi/ic_launcher.png:  
	PNG image data, 144 x 144, 8-bit/color RGBA, non-inte
Android/app/src/main/res/mipmap-mdpi/ic_launcher.png:    
	PNG image data, 48 x 48, 8-bit/color RGBA, non-interl
Android/app/src/main/res/mipmap-xxxhdpi/ic_launcher.png: 
	PNG image data, 192 x 192, 8-bit/color RGBA, non-inte
Android/app/src/main/res/mipmap-hdpi/ic_launcher.png:    
	PNG image data, 72 x 72, 8-bit/color RGBA, non-interl
\end{lstlisting}

\clearpage

\fig{android/firstrun.png}{height=.85\textheight}
\fig{android/fixcolors.png}{height=.85\textheight}
\fig{android/addbutton.png}{height=.85\textheight}

\noindent
Цвета по умолчанию получаются слишком кислотные, стоит их приглушить поправив
файлы, содержащие цвета и стили для проекта:

\begin{lstlisting}[title=Android/app/src/main/res/values/colors.xml]
<?xml version="1.0" encoding="utf-8"?>
<resources>
    <color name="colorPrim">#008577</color>
    <color name="colorPrimDark">#00473B</color>
    <color name="colorAccent">#880022</color>
    <color name="colorBgr">#222211</color>
</resources>
\end{lstlisting}

\begin{lstlisting}[title=Android/app/src/main/res/values/styles.xml]
<resources>
<style name="AppTheme" parent="Theme.AppCompat">
<item name="colorPrimary">@color/colorPrimDark</item>
<item name="colorPrimaryDark">@color/colorPrimDark</item>
<item name="android:windowBackground">@color/colorBgr</item>
\end{lstlisting}

\secrel{Командная консоль}\label{anconsole}

Наше приложение будет построено в виде нескольких слайдеров\note{помним про
принцип ``интерфейса большого пальца''}, главным из которых является список
текущих задач \ref{antasks}, но стоит начать с реализации Activity командного
интерфейса, чтобы показать нечто работающее хотя бы в режиме постфиксного
калькулятора. Ввод команд неудобен для часто выполняемых действий, но для
нетиповых задач и универсальности великолепно подходят скрипты-однострочники на
языке, близком к \F\ \ref{forth}.

Так как пользователь может использовать команды в самых неблагоприятных
условиях, очень важно выбрать размер шрифтов и цветовую схему, наиболее
разборчивую на маленьких экранах дешевых телефонов:

\lst{android/consdimens.xml}{title=Android/res/values/dimens.xml}
\lst{android/constextview.xml}{title=Android/res/layout/content\_main.xml}
\lst{android/conscolor.xml}{title=Android/res/values/colors.xml}

\secup
