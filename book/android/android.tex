\secrel{\A\ органайзер}\label{android}\secdown

\noindent
\begin{tabular}{l|p{8.3cm}}
\fig{android/android_plan.png}{height=.45\textheight} &
Мобильниый телефон\ --- компьютер, который всегда с собой. Но \emph{удобных
средств программирования}, работающих на мобильном \textit{телефоне}, а не
планшете, практически нет.

Программирование под \A\ и реализация on-device системы программирования\ ---
тема отдельной большой книги. Для начала можно попробовать написать
\emph{органайзер, программируемый пользоваталем}
\\ \end{tabular}

\clearpage
\begin{description}
\item[PIM] Personal Information Manager

Функции, выполняемые органайзером\note{персональным информационным
менеджером}:
\begin{itemize}[nosep]
  \item 
\emph{планирование задач} для контроля за их самостоятельным и сторонним
выполнением (ToDo list, task-трекер, мобильный CRM);
  \item 
планирование событий, привязанных к датам и времени (праздники или встречи);
  \item 
\emph{напоминальники и зудильники} об определённых пользователем событиях;
  \item 
управление контактами (адресно-телефонная книга);
  \item 
записная книжка и листки-липучки;
  \item 
личные записи (дневник);
  \item 
интеграция с электронной почтой и мессенджерами
  \item 
\emph{персональная база знаний}.
\end{itemize}

\item[PPS] Personal Planning System, система персонального планирования\\
специально заточенная на трекинг задач, с точки зрения конкретного
человека.
\end{description}

\noindent
Несмотря на десятки лет усилий, даже такие гиганты как Google и Microsoft не
смогли решить проблему создания полноценного органайзера, в который by design
должен был превратиться смартфон:
\begin{itemize}
  \item доступны только примитивные типы задач, при этом на практике нужно
  множество вариантов, от простого будильника, встречи, действий привязанных по
  месту, периодические задачи с разным масштабом\note{от 30 минут для отдыха
  глаз, до год для дней рожденья}, до задач характерных для систем трекинга
  проектов \note{групповые задачи, зависимости по времени и исполнителям,
  делегирование и контроль,\ldots}
  \item отсутствие средств индивидуальной адаптации, включая средства
  программирования пользователем, и доступ к внешним приложениям, библиотекам и
  сенсорам
\end{itemize}

\secup