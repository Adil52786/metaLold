\secrel{Событийная архитектура вместо ОСРВ}\label{event}\secdown

\clearpage
Разработчики встроенного программного обеспечения из разных отраслей независимо
друг от друга заново открывают шаблоны для создания параллельного программного
обеспечения, которое является более безопасным, более гибким и более простым для
понимания, чем обычные потоки и различные механизмы блокировки
ОСРВ\note{Операционная Система Реального Времени}.

Эти лучшие практики всегда предпочитают \emph{управляемые
событиями}\note{\term{event-driven} событийное программирование} асинхронные
неблокирующие инкапсулированные \term{активные объекты}, каждый из которых
управляется внутренним \term{конечным автоматом}, а не чистыми блокирующими
потоками RTOS.

В следующих разделах объясняются концепции, связанные с этим набирающим
популярность \term{реактивным} подходом, и, в частности, как он применяется к
встроенным системам реального времени, на которые ориентированы платформы и
инструменты Quantum Leaps \cite{psicc2}.

Практически все встроенные системы по своей природе \term{реактивны}, что
означает, что их основная задача\ --- реагировать на события, такие как нажатия
кнопок, касания экрана, тайм-ауты или прибытия некоторых пакетов данных.
Следовательно, большую часть времени встроенная система ожидает событий, и
только после распознавания события система реагирует, выполняя соответствующие
вычисления.

Две основные проблемы такого подхода:
\begin{itemize}[nosep]
\item выполнить правильные вычисления 
\item и выполнить их \emph{своевременно}.
\end{itemize}

\secrel{QP\texttrademark\ Real-Time Embedded Frameworks}\label{qp}

{\Huge \href{https://www.state-machine.com/}{$QuantumL^{e_a}Ps$}\ \cite{psicc2}}

\lst{event/game.c}{language=C}

\secup
