\secrel{Концептуальное программирование}\label{concept}\secdown

Кнут утверждает, что литературное программирование обеспечивает первоклассную
систему документирования, которая не является надстройкой над процессом
разработки, а наоборот естественным образом направляет процесс разработки через
изложение своих мыслей при создании программы. Кнут описывает разработку как
\emph{построение сети абстрактных концептов} о содержании и функционале
программы, что напрямую пересекается с концептуальным программированием
\cite{tyugu} как построение абстрактной модели предметной области.

В \cite{moskvitin1}\ использовано правильное слово по поводу решения задач,
которые не имеют заранее заданной технологии решения\ --- \term{томление}. В
современной разработке ПО такое ``томление'' занимает значитальную часть
времени, когда заказчик не знает чего он хочет, а исполнитель не уверен, или не
знает до конца весь набор приемов и ``\term{сниппетов}''\note{code snippet,
короткие типовые куски кода (шаблоны кода), которые тащатся из проекта в
проект методом копипасты, с мелкими изменениями или без}, которые он будет
использовать.

В некоторых областях даже нет сложившегося инструментария, и разработчик
вынужден макетировать программную систему, переделывать ее снова и снова,
пробовать варианты, и вообще \term{прототипировать}.

Очень часто в практике встречаются не полностью определенные задачи (полузадачи
\cite{moskvitin1}), когда нет точно определенного набора шагов, а можно указать
лишь \emph{ограничения на множество решений}.

\begin{framed}\noindent
\term{Концептуальное программирование}\ --- \emph{прототипирование} программных
систем \emph{через задание ограничений}, семантических, логических,\ldots
\emph{отношений между входными/выходными данными, и компонентами программы},
которую требуется синтезировать в качестве решения задачи.
\end{framed}

\secup
