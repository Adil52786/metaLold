\clearpage
\secrel{Сборка кросс-компилятора GNU Toolchain}\label{cross}

\begin{verbatim}
$ cd ~/metaL/mcu
$ make cross
\end{verbatim}

Для разработки принято использовать готовый инструментарий: обычно начиная
осваивать новый микроконтроллер, идут на официальный сайт, и устанавливают
рекомендованное ПО.

Здесь в качестве иллюстрации применена пошаговая сборка кросс-компилятора
\emph{GNU Toolchain} из исходного кода. Не стоит использовать этот метод для
сборки \linux-систем, но хотелось показать вариант ручной сборки на тот случай,
если вам понадобится какая-то очень специфичая конфигурация.

Если вы работаете под \linux, у вас большая часть этого ПО уже доступна в виде
пакетов дистрибутива. Чтобы их установить в систему, нужны привелегии
администратора, а это не всегда доступно. 

\clearpage\noindent
Текущий каталог (полный путь)
\begin{lstlisting}[language=make]
CWD    = $(CURDIR)
\end{lstlisting}
Иногда нужно название модуля, взятое как называние текущего каталога
\begin{lstlisting}[language=make]
MODULE = $(notdir $(CURDIR))
\end{lstlisting}
\file{.PHONY} задает список целей, не являющихся файлами: эти цели используются
как имена при вызове различных частей \file{Makefile} заданных вручную
\begin{lstlisting}[language=make]
.PHONY: cross all clean dirs gz cclibs binutils gcc
\end{lstlisting}
\emph{Сборка кросс-компилятора}
\begin{lstlisting}[language=make]
cross: dirs gz cclibs binutils gcc0 newlib gcc
\end{lstlisting}
Создание временных каталогов
\begin{lstlisting}[language=make]
TMP		?= $(CWD)/tmp
SRC		?= $(TMP)/src
GZ		?= $(HOME)/gz
CROSS	?= $(CWD)/$(TARGET)
SYSROOT	?= $(CROSS)/sysroot

dirs:
	mkdir -p $(TMP) $(SRC) $(GZ) $(CROSS) $(SYSROOT)
\end{lstlisting}
\begin{description}[nosep]
\item[\file{TMP}] каталог в котором будут собираться пакеты\\(out of
tree build\ --- отдельно от исходного кода)
\item[\file{SRC}] каталог для распаковки исходного кода пакетов
\item[\file{GZ}] каталог для загрузки архивов пакетов\\
поскольку я часто собираю из исходников, свалка находится в домашнем каталоге 
\end{description}

При запуске \file{make}\ любую конфигурационную переменную можно переопределить
из командной строки, например;
\begin{lstlisting}
$ make TMP=/tmp cross
\end{lstlisting}

\begin{lstlisting}[language=make]
BINUTILS_VER	= 2.32
BINUTILS		= binutils-$(BINUTILS_VER)
BINUTILS_GZ		= $(BINUTILS).tar.xz

gz: $(GZ)/$(BINUTILS_GZ)

WGET = wget -P $(GZ) -c

$(GZ)/$(BINUTILS_GZ):
  $(WGET) http://ftp.gnu.org/gnu/binutils/$(BINUTILS_GZ)
\end{lstlisting}

В сборочном \file{Makefile}\ иногда меняются версии ПО, поэтому задание
переменных версий лучше перенести в самое начало файла.

Для сборки компилятора GNU GCC необходимо несколько библиотек\note{арифметика
произвольной точности, и полигональная оптимизация}.
\begin{lstlisting}[language=make]
GMP_VER			= 6.1.2
MPFR_VER		= 4.0.2
MPC_VER			= 1.1.0

GMP				= gmp-$(GMP_VER)
MPFR			= mpfr-$(MPFR_VER)
MPC				= mpc-$(MPC_VER)

GMP_GZ			= $(GMP).tar.xz
MPFR_GZ			= $(MPFR).tar.xz
MPC_GZ			= $(MPC).tar.gz

gz: $(GZ)/$(BINUTILS_GZ) \
	$(GZ)/$(GMP_GZ) $(GZ)/$(MPFR_GZ) $(GZ)/$(MPC_GZ)
	
$(GZ)/$(GMP_GZ):
	$(WGET) ftp://ftp.gmplib.org/pub/gmp/$(GMP_GZ)
$(GZ)/$(MPFR_GZ):
	$(WGET) https://www.mpfr.org/mpfr-current/$(MPFR_GZ)
$(GZ)/$(MPC_GZ):
	$(WGET) https://ftp.gnu.org/gnu/mpc/$(MPC_GZ)
\end{lstlisting}

\begin{lstlisting}[language=make]
ISL_VER			= 0.16.1
CLOOG_VER		= 0.18.1

ISL				= isl-$(ISL_VER)
CLOOG			= cloog-$(CLOOG_VER)

ISL_GZ			= $(ISL).tar.bz2
CLOOG_GZ		= $(CLOOG).tar.gz

gz: $(GZ)/$(BINUTILS_GZ) \
	$(GZ)/$(GMP_GZ) $(GZ)/$(MPFR_GZ) $(GZ)/$(MPC_GZ) \
	$(GZ)/$(ISL_GZ) $(GZ)/$(CLOOG_GZ)

$(GZ)/$(ISL_GZ):
  $(WGET) \
    ftp://gcc.gnu.org/pub/gcc/infrastructure/$(ISL_GZ)
    
$(GZ)/$(CLOOG_GZ):
  $(WGET) \
    ftp://gcc.gnu.org/pub/gcc/infrastructure/$(CLOOG_GZ)
\end{lstlisting}

\clearpage
\begin{lstlisting}[language=make]
BINUTILS_VER	= 2.32
GCC_VER			= 8.3.0

gz: $(GZ)/$(BINUTILS_GZ) $(GZ)/$(GCC_GZ) \

$(GZ)/$(BINUTILS_GZ):
  $(WGET) http://ftp.gnu.org/gnu/binutils/$(BINUTILS_GZ)
$(GZ)/$(GCC_GZ):
  $(WGET) https://ftp.gnu.org/gnu/gcc/$(GCC)/$(GCC_GZ)
\end{lstlisting}

\begin{lstlisting}[language=make]
cclibs: gmp mpfr mpc cloog isl

CFG = configure --prefix=$(CROSS)

CORENUM = $(shell grep processor /proc/cpuinfo|wc -l)
\end{lstlisting}

\fig{/tmp/gnugcc.pdf}{height=.6\textheight}

\begin{description}
\item[CFG] команда \file{configure} с параметрами для всех пакетов
\item[CORENUM] число процессорных ядер на рабочей станции\\ для параллельной
сборки
\item[--prefix] задает путь установки
\end{description}

\begin{lstlisting}[language=make]
CFG_CCLIBS = --disable-shared

CFG_GMP    = $(CFG_CCLIBS)

gmp: $(CROSS)/lib/libgmp.a
$(CROSS)/lib/libgmp.a: $(SRC)/$(GMP)/README
	rm -rf $(TMP)/$(GMP) ; \
	mkdir $(TMP)/$(GMP) ; cd $(TMP)/$(GMP) ; \
	$(SRC)/$(GMP)/$(CFG) $(CFG_GMP) && \
	make -j$(CORENUM) && make install
\end{lstlisting}

\begin{description}
\item[CFG\_CCLIBS] общие опции \file{configure} для всех библиотек
\item[CFG\_GMP] \verb|--disable-shared|\\ собирать только статические библиотеки
\end{description}

\noindent
Для распаковки исходного кода из архивов применяется набор шаблонов:
\begin{lstlisting}[language=make]
$(SRC)/%/README: $(GZ)/%.tar.bz2
	cd $(SRC) ; bzcat $< | tar x && touch $@
$(SRC)/%/README: $(GZ)/%.tar.xz
	cd $(SRC) ; xzcat $< | tar x && touch $@
$(SRC)/%/README: $(GZ)/%.tar.gz
	cd $(SRC) ;  zcat $< | tar x && touch $@
\end{lstlisting}

\begin{lstlisting}[language=make]
CFG_MPFR   = $(CFG_CCLIBS) --with-gmp=$(CROSS)

mpfr: $(CROSS)/lib/libmpfr.a
$(CROSS)/lib/libmpfr.a: $(SRC)/$(MPFR)/README
	$(CFG) $(CFG_MPFR) && make && make install
\end{lstlisting}

\begin{lstlisting}[language=make]
CFG_MPC    = $(CFG_CCLIBS) --with-gmp=$(CROSS)
\end{lstlisting}

Для сборки компилятора GCC с набором оптимизаций циклов Graphite:

\begin{lstlisting}[language=make]
CFG_CLOOG	= $(CFG_CCLIBS) --with-gmp-prefix=$(CROSS)

cloog: $(CROSS)/lib/libcloog-isl.a
$(CROSS)/lib/libcloog-isl.a: $(SRC)/$(CLOOG)/README
	mkdir $(TMP)/$(CLOOG) ; cd $(TMP)/$(CLOOG) ; \ 
	$(CFG) $(CFG_CLOOG) && make && make install

CFG_ISL		= $(CFG_CCLIBS) --with-gmp-prefix=$(CROSS)

isl: $(CROSS)/lib/libisl.a
$(CROSS)/lib/libisl.a: $(SRC)/$(ISL)/README
	mkdir $(TMP)/$(ISL) ; cd $(TMP)/$(ISL) ; \ 
	$(CFG) $(CFG_ISL) && make && make install
\end{lstlisting}

\clearpage
\begin{lstlisting}[language=make]
TARGET ?= arm-none-eabi

CFG_WITHCCLIBS = --with-gmp=$(CROSS) \
				--with-mpfr=$(CROSS) --with-mpc=$(CROSS)

CFG_BINUTILS = --disable-nls --prefix=$(CROSS) \
			--target=$(TARGET) \
			--with-sysroot=$(SYSROOT) \
			--with-native-system-header-dir=/include \
			--enable-lto --disable-multilib \
			$(CFG_WITHCCLIBS)

binutils: $(CROSS)/bin/$(TARGET)-ld
$(CROSS)/bin/$(TARGET)-ld: $(SRC)/$(BINUTILS)/README
	$(CFG) $(CFG_BINUTILS) && make && make install
\end{lstlisting}

GCC собирается в два этапа: сначала \file{gcc0}, потом стандартная библиотека
\file{newlib} (libc), и в конце \file{gcc}:

\begin{lstlisting}[language=make]
CFG_GCC = $(CFG_BINUTILS) --with-newlib \
							--enable-languages="c"

gcc0: $(SRC)/$(GCC)/README
	cd $(TMP)/$(GCC) ; $(SRC)/$(GCC)/$(CFG) $(CFG_GCC)
	cd $(TMP)/$(GCC) ; \
		$(MAKE) -j2 all-host ; $(MAKE) install-host

gcc: $(CROSS)/$(TARGET)/lib/libc.a
	cd $(TMP)/$(GCC) ; \
		$(MAKE) -j2 all-target ; $(MAKE) install-target
\end{lstlisting}

\clearpage
\begin{lstlisting}[language=make]
NEWLIB_VER	= nano-2.1
NEWLIB		= newlib-$(NEWLIB_VER)

CFG_NEWLIB = $(CFG_BINUTILS) --prefix=$(SYSROOT) \
				--disable-newlib-supplied-syscalls \
				--infodir=$(CROSS)/share/info

newlib: $(SYSROOT)/lib/libc.a
$(SYSROOT)/lib/libc.a: $(SRC)/$(NEWLIB)/README
	rm -rf $(TMP)/$(NEWLIB) ; mkdir $(TMP)/$(NEWLIB) ; \
		cd $(TMP)/$(NEWLIB) ; \
	$(XPATH) $(SRC)/$(NEWLIB)/$(CFG) $(CFG_NEWLIB) && \
		$(MAKEJ) && $(MAKE) install
	mv $(SYSROOT)/$(TARGET)/* $(SYSROOT)/ ; \
		rmdir $(SYSROOT)/$(TARGET)
\end{lstlisting}
\clearpage
Исходный код \file{newlib} загружается с сервера контроля версий, в отличие от
остальных пакетов, использующих архивы:
\begin{lstlisting}[language=make]
$(SRC)/$(NEWLIB)/README:
	cd $(SRC) ; git clone --depth=1 \
		https://github.com/iperry/$(NEWLIB).git
#	cd $(SRC) ; git clone \
#		https://keithp.com/cgit/newlib.git
\end{lstlisting}
