\secrel{STM32F030F4P6 blue pill}\label{f030}\secdown

\begin{itemize}
  \item File / New Project
  \item MCU Selector / STM32F030F4Px TSSOP20 Flash:16K RAM:4K
  \item Start Project
\end{itemize}

File / Save Project as / \verb|/home/dpon/metaL/mcu/test/cubeF030|

\fig{mcu/arm/pinout.png}{height=\textheight}

\fig{mcu/arm/f030pins.jpg}{height=\textheight}

\fig{mcu/arm/f030sch.png}{width=\textwidth}

\fig{mcu/arm/f030top.png}{height=\textheight}

На плате установлен внешний кварц на 8 МГЦ, будем использовать его в качестве
источника \emph{внешней тактовой частоты} \term{HSE}:

\medskip
\fig{mcu/arm/HSE.png}{width=\textwidth}

\clearpage
8 МГц для тактирования, и /8=1 МГц для периферии вполне достаточно, чтобы
сравнить производительность/потребление с одуриной

\bigskip
\fig{mcu/arm/HPclock.png}{width=\textwidth}

\noindent
и при этом иметь рабочий интерфейс по последовательному порту UART 9600 8N1

\bigskip
\fig{mcu/arm/uartclk.png}{height=.5\textheight}

\bigskip
\cm{0}\ имеет две шины, и три тактовых частоты:
\begin{description}
\item[SYSCLK] системная тактовая частота, тактируется \emph{ядро процессора}
\item[AHB] \term{Advanced Host Bus}: тактирование \emph{Flash/RAM} и \emph{DMA}
\item[APB] \term{Advanced Peripherial Bus}: тактирование \emph{периферии}
(ввода/вывода)
\end{description} 


\begin{itemize}
  \item 
Project Manager
\item Project
\item Toolchain Folder Location \verb|/home/dpon/metaL/mcu/test/cubeF030/|
\item Toolchain/IDE \verb|Makefile|
\item Code Generator
\item Add nessesary library files as reference
\item Generate peripherial init as spearate .c/.h files
\item Keep User Code
\item Advanced Settings
\item Set all LL (replace from HAL)
\end{itemize}

\secup
