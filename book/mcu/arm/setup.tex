\secrel{Настройка рабочего стенда (железо и ПО)}\secdown

\url{https://www.youtube.com/watch?v=DCjhJCk_bIE}

\href{https://www.aliexpress.com/item/FREE-SHIPPING-ST-Link-V2-stlink-mini-STM8STM32-STLINK-simulator-download-programming-With-Cover/1814606455.html}{\fig{mcu/arm/cnstlink.jpg}{height=.65\textheight}}
\href{https://www.aliexpress.com/item/48-MHz-STM32F030F4P6-Small-Systems-Development-Board-CORTEX-M0-Core-32bit-Mini-System-Development-Panels/32831635311.html}{\fig{mcu/arm/stm32f040.png}{height=.65\textheight}}

Минимальный комплект\ --- китайский клон программатора ST-Link v2 и отладочная
плата на базе самого мелкого и дешевого микроконтроллера STM32F030F4P6 (55р в
розницу в самом жлобском магазине).

\bigskip
Работу с линейкой STM32 рекомендую начинать с использования фирменного
комфигуратора\note{это не опечатка, поймете когда взгляните на сгенерированный
код\ --- первый блин комом. Идея применения \metal\ появилась из
соображений следования идеологии ``тяп-ляп и в продакшын'', но заменить ляпалки
на более контролируемый инструмент}\ CubeMX.
Его использование обеспечит вам быстрый старт, он возьмет на себя часть
сложности по начальной инициализации микроконтроллера. При этом никто не
запрещает вам спутстится на более низкий уровень, уйти на программирование на
чистом CMSIS, и избавиться от vendor lock\ --- с этого момента вы не будете
зависеть от закидонов ST Microelectronics, известной своей политикой левой пятки
с библиотеками StdPeriph, HAL и напахал.

\secrel{Расширение \metal/mcu}

\noindent
Если вы этого еще не сделале, склонируйте корневой проект \metal:

\begin{verbatim}
$ cd ~
$ git clone -o gh https://github.com/ponyatov/metaL.git
$ cd ~/metaL
\end{verbatim}
Если проект существует, обновитесь до последней master ветки:
\begin{verbatim}
$ git pull
\end{verbatim}
Загрузите модуль расширения:
\begin{verbatim}
$ git submodule update --init --remote --recursive -- mcu
\end{verbatim}

\secrel{Сборка кросс-компилятора GNU Toolchain}

\begin{verbatim}
$ cd ~/metaL/mcu
$ make cross
\end{verbatim}

Для разработки принято использовать готовый инструментарий: обычно начиная
осваивать новый микроконтроллер, идут на официальный сайт, и устанавливают
рекомендованное ПО.

Здесь в качестве иллюстрации применена пошаговая сборка кросс-компилятора
\emph{GNU Toolchain} из исходного кода. Не стоит использовать этот метод для
сборки \linux-систем, но хотелось показать вариант ручной сборки на тот случай,
если вам понадобится какая-то очень специфичая конфигурация.

Если вы работаете под \linux, у вас большая часть этого ПО уже доступна в виде
пакетов дистрибутива. Чтобы их установить в систему, нужны привелегии
администратора, а это не всегда доступно. 

\clearpage\noindent
Текущий каталог (полный путь)
\begin{lstlisting}[language=make]
CWD = $(CURDIR)
\end{lstlisting}
Иногда нужно название модуля, взятое как называние текущего каталога
\begin{lstlisting}[language=make]
MODULE = $(notdir $(CURDIR))
\end{lstlisting}
\file{.PHONY} задает список целей, не являющихся файлами: эти цели используются
как имена при вызове различных частей \file{Makefile} заданных вручную
\begin{lstlisting}[language=make]
.PHONY: cross all clean
\end{lstlisting}
\emph{Сборка кросс-компилятора}
\begin{lstlisting}[language=make]
cross: dirs gz cclibs binutils gcc
\end{lstlisting}


\secrel{CubeMX}

\url{https://www.st.com/b/en/development-tools/stm32cubemx.html}

\secup