\secrel{Настройка рабочего стенда (железо и ПО)}\secdown

\url{https://www.youtube.com/watch?v=DCjhJCk_bIE}

\href{https://www.aliexpress.com/item/FREE-SHIPPING-ST-Link-V2-stlink-mini-STM8STM32-STLINK-simulator-download-programming-With-Cover/1814606455.html}{\fig{mcu/arm/cnstlink.jpg}{height=.65\textheight}}
\href{https://www.aliexpress.com/item/48-MHz-STM32F030F4P6-Small-Systems-Development-Board-CORTEX-M0-Core-32bit-Mini-System-Development-Panels/32831635311.html}{\fig{mcu/arm/stm32f040.png}{height=.65\textheight}}

Минимальный комплект\ --- китайский клон программатора ST-Link v2 и отладочная
плата на базе самого мелкого и дешевого микроконтроллера STM32F030F4P6 (55р в
розницу в самом жлобском магазине).

\bigskip
Работу с линейкой STM32 рекомендую начинать с использования фирменного
комфигуратора\note{это не опечатка, поймете когда взгляните на сгенерированный
код\ --- первый блин комом. Идея применения \metal\ появилась из
соображений следования идеологии ``тяп-ляп и в продакшын'', но заменить ляпалки
на более контролируемый инструмент}\ CubeMX.
Его использование обеспечит вам быстрый старт, он возьмет на себя часть
сложности по начальной инициализации микроконтроллера. При этом никто не
запрещает вам спутстится на более низкий уровень, уйти на программирование на
чистом CMSIS, и избавиться от vendor lock\ --- с этого момента вы не будете
зависеть от закидонов ST Microelectronics, известной своей политикой левой пятки
с библиотеками StdPeriph, HAL и напахал.

\secrel{Расширение \metal/mcu}

\noindent
Если вы этого еще не сделале, склонируйте корневой проект \metal:

\begin{verbatim}
$ cd ~
$ git clone -o gh https://github.com/ponyatov/metaL.git
$ cd ~/metaL
\end{verbatim}
Если проект существует, обновитесь до последней master ветки:
\begin{verbatim}
$ git pull
\end{verbatim}
Загрузите модуль расширения:
\begin{verbatim}
$ git submodule update --init --remote --recursive -- mcu
\end{verbatim}
установите кросс-компилятор GNU GCC из бинарного дистрибутива или пакета для
вашей ОС, или соберите его из исходных текстов
\ref{cross}

\secrel{CubeMX}

\url{https://www.st.com/b/en/development-tools/stm32cubemx.html}

\secup