\secrel{Настройка рабочего стенда (железо и ПО)}\secdown

\url{https://www.youtube.com/watch?v=DCjhJCk_bIE}

\href{https://www.aliexpress.com/item/FREE-SHIPPING-ST-Link-V2-stlink-mini-STM8STM32-STLINK-simulator-download-programming-With-Cover/1814606455.html}{\fig{mcu/arm/cnstlink.jpg}{height=.65\textheight}}
\href{https://www.aliexpress.com/item/48-MHz-STM32F030F4P6-Small-Systems-Development-Board-CORTEX-M0-Core-32bit-Mini-System-Development-Panels/32831635311.html}{\fig{mcu/arm/stm32f040.png}{height=.65\textheight}}

\noindent
Минимальный комплект\ --- китайский клон программатора ST-Link v2-1 и отладочная
плата на базе самого мелкого и дешевого микроконтроллера STM32F030F4P6 (70р в
розницу в самом жлобском магазине).

\bigskip
Работу с линейкой STM32 рекомендую начинать с использования фирменного
комфигуратора\note{это не опечатка, поймете когда взгляните на сгенерированный
код\ --- первый блин комом. Идея применения \metal\ появилась из
соображений следования идеологии ``тяп-ляп и в продакшын'', но заменить ляпалки
на более контролируемый инструмент}\ CubeMX.
Его использование обеспечит вам быстрый старт, он возьмет на себя часть
сложности по начальной инициализации микроконтроллера. При этом никто не
запрещает вам спуститься на более низкий уровень, уйти на программирование на
чистом CMSIS, и избавиться от vendor lock\ --- с этого момента вы не будете
зависеть от закидонов ST Microelectronics, известной своей политикой левой пятки
с библиотеками StdPeriph, HAL и напахал.

\secrel{Расширение \metal/mcu}

\noindent
Если вы этого еще не сделали, склонируйте корневой проект \metal:

\begin{verbatim}
$ cd ~
$ git clone -o gh https://github.com/ponyatov/metaL.git
$ cd ~/metaL
\end{verbatim}
Если проект существует, обновитесь до последней master ветки:
\begin{verbatim}
$ git pull
\end{verbatim}
Загрузите модуль расширения:
\begin{verbatim}
$ git submodule update --init --remote --recursive -- mcu
\end{verbatim}
установите кросс-компилятор GNU GCC из бинарного дистрибутива или пакета для
вашей ОС, или соберите его из исходных текстов
\ref{cross}

\secrel{CubeMX}

\begin{itemize}
  \item 
\url{https://www.st.com/b/en/development-tools/stm32cubemx.html}
  \item
Get Software
\item
unzip \verb|en.SetupSTM32CubeMX-5.1.0-RC6.zip -> tmp|
\item
run installer \verb|./SetupSTM32CubeMX-5.1.0.linux|
\item 
Java required !
\item
\verb|next > accept > next > I have read Policy > next|
\item 
\verb|/home/ponyatov/metaL/mcu/STM32CubeMX|
\item 
\verb|next > Done|
\item 
\verb|~/metaL/mcu$ ./STM32CubeMX/STM32CubeMX &|
\end{itemize}

\fig{mcu/arm/cubeinst.png}{height=.2\textheight}

\fig{mcu/arm/stm32l0.png}{width=\textwidth}

\fig{mcu/arm/f0install.png}{height=.35\textheight} Install Now

\secrel{Atollic TrueSTUDIO}

\url{https://atollic.com/truestudio/}

\href{http://download.atollic.com/TrueSTUDIO/installers/Atollic_TrueSTUDIO_for_STM32_linux_x86_64_v9.3.0_20190212-0734.tar.gz}{Download v9.3.0}

\href{http://easyelectronics.ru/cozdanie-minimalnogo-proekta-pod-stm32-v-atollic-true-studio.html}{пошаговое руководство по созданию пустого проекта без Makefile}

\begin{itemize}
\item
\href{https://istarik.ru/blog/arduino/102.html}{Описание платы F103 BluePill и запуск Arduino STM32}
\item
\href{https://istarik.ru/blog/arduino/104.html}{прерывания Arduino STM32}
\item
\href{https://istarik.ru/blog/arduino/105.html}{таймеры и ШИМ Arduino STM32}
\item
\href{https://istarik.ru/blog/stm32/107.html}{Перешивка BluePill в ST-Link}
\item
\href{https://istarik.ru/blog/stm32/106.html}{создание проекта в CubeMX и программирование через USB Serial}
\end{itemize}

\begin{lstlisting}
$ cd /tmp/ ; tar zx < ~/Загрузки/Atollic_TrueSTUDIO_for_STM32_linux_x86_64_v9.3.0_20190212-0734.tar.gz
$ cd Atollic_TrueSTUDIO_for_STM32_9.3.0_installer/
$ sudo ./install.sh

Do you want to install to '/opt/Atollic_TrueSTUDIO_for_STM32_x86_64_9.3.0/'?
3) Change

/home/ponyatov/metaL/mcu/Atollic

~/metaL/mcu$ ./Atollic/ide/TrueSTUDIO

Workspace: /home/ponyatov/metaL/mcu/test
[v] use as default

File > New > General Project > cubeF030

File > New > Convert to C/C++ Project

C Project > Executable/Atollic ARM Tools

project > \rms > Properties > 
	Resourse > UTF-8/Unix > Apply
	C/C++ Build
		> Settings > [x] GNU Elf parser
		> Toolchain Editor > Current builder > GNU Make builder

\end{lstlisting}

\secup
