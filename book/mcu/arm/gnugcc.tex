\secrel{Сборка кросс-компилятора GNU Toolchain}

\begin{verbatim}
$ cd ~/metaL/mcu
$ make cross
\end{verbatim}

Для разработки принято использовать готовый инструментарий: обычно начиная
осваивать новый микроконтроллер, идут на официальный сайт, и устанавливают
рекомендованное ПО.

Здесь в качестве иллюстрации применена пошаговая сборка кросс-компилятора
\emph{GNU Toolchain} из исходного кода. Не стоит использовать этот метод для
сборки \linux-систем, но хотелось показать вариант ручной сборки на тот случай,
если вам понадобится какая-то очень специфичая конфигурация.

Если вы работаете под \linux, у вас большая часть этого ПО уже доступна в виде
пакетов дистрибутива. Чтобы их установить в систему, нужны привелегии
администратора, а это не всегда доступно. 

\clearpage\noindent
Текущий каталог (полный путь)
\begin{lstlisting}[language=make]
CWD    = $(CURDIR)
\end{lstlisting}
Иногда нужно название модуля, взятое как называние текущего каталога
\begin{lstlisting}[language=make]
MODULE = $(notdir $(CURDIR))
\end{lstlisting}
\file{.PHONY} задает список целей, не являющихся файлами: эти цели используются
как имена при вызове различных частей \file{Makefile} заданных вручную
\begin{lstlisting}[language=make]
.PHONY: cross all clean dirs gz cclibs gcc
\end{lstlisting}
\emph{Сборка кросс-компилятора}
\begin{lstlisting}[language=make]
cross: dirs gz cclibs binutils gcc
\end{lstlisting}
Создание временных каталогов
\begin{lstlisting}[language=make]
TMP ?= $(CWD)/tmp
SRC ?= $(CWD)/src
GZ  ?= $(HOME)/gz

.PHONY: dirs
dirs:
	mkdir -p $(TMP) $(SRC) $(GZ)
\end{lstlisting}
\begin{description}
\item[\file{TMP}] каталог в котором будут собираться пакеты\\(out of
tree build\ --- отдельно от исходного кода)
\item[\file{SRC}] каталог для распаковки исходного кода пакетов
\item[\file{GZ}] каталог для загрузки архивов пакетов\\
поскольку я часто собирал из исходников, свалка находится в домашнем каталоге 
\end{description}

При запуске \file{make}\ любую конфигурационную переменную можно переопределить
из командной строки, например;
\begin{lstlisting}
$ make TMP=/tmp cross
\end{lstlisting}

\begin{lstlisting}[language=make]
BINUTILS_VER	= 2.32
BINUTILS		= binutils-$(BINUTILS_VER)
BINUTILS_GZ		= $(BINUTILS).tar.xz

gz: $(GZ)/$(BINUTILS_GZ)

WGET = wget -P $(GZ) -c

$(GZ)/$(BINUTILS_GZ):
  $(WGET) http://ftp.gnu.org/gnu/binutils/$(BINUTILS_GZ)
\end{lstlisting}

В сборочном \file{Makefile}\ иногда меняются версии ПО, поэтому задание
переменных версий лучше перенести в самое начало файла.

Для сборки компилятора GNU GCC необходимо несколько библиотек\note{арифметика
произвольной точности, и полигональная оптимизация}.
\begin{lstlisting}[language=make]
GMP_VER			= 6.1.2
MPFR_VER		= 4.0.2
MPC_VER			= 1.1.0

GMP				= gmp-$(GMP_VER)
MPFR			= mpfr-$(MPFR_VER)
MPC				= mpc-$(MPC_VER)

GMP_GZ			= $(GMP).tar.xz
MPFR_GZ			= $(MPFR).tar.xz
MPC_GZ			= $(MPC).tar.gz

gz: $(GZ)/$(BINUTILS_GZ) \
	$(GZ)/$(GMP_GZ) $(GZ)/$(MPFR_GZ) $(GZ)/$(MPC_GZ)
	
$(GZ)/$(GMP_GZ):
	$(WGET) ftp://ftp.gmplib.org/pub/gmp/$(GMP_GZ)
$(GZ)/$(MPFR_GZ):
	$(WGET) https://www.mpfr.org/mpfr-current/$(MPFR_GZ)
$(GZ)/$(MPC_GZ):
	$(WGET) https://ftp.gnu.org/gnu/mpc/$(MPC_GZ)
\end{lstlisting}

\begin{lstlisting}[language=make]
ISL_VER			= 0.18
CLOOG_VER		= 0.18.1

ISL				= isl-$(ISL_VER)
CLOOG			= cloog-$(CLOOG_VER)

ISL_GZ			= $(ISL).tar.bz2
CLOOG_GZ		= $(CLOOG).tar.gz

gz: $(GZ)/$(BINUTILS_GZ) \
	$(GZ)/$(GMP_GZ) $(GZ)/$(MPFR_GZ) $(GZ)/$(MPC_GZ) \
	$(GZ)/$(ISL_GZ) $(GZ)/$(CLOOG_GZ)

$(GZ)/$(ISL_GZ):
  $(WGET) \
    ftp://gcc.gnu.org/pub/gcc/infrastructure/$(ISL_GZ)
$(GZ)/$(CLOOG_GZ):
  $(WGET) \
    ftp://gcc.gnu.org/pub/gcc/infrastructure/$(CLOOG_GZ)
\end{lstlisting}

\clearpage
\begin{lstlisting}[language=make]
BINUTILS_VER	= 2.32
GCC_VER			= 8.3.0

gz: $(GZ)/$(BINUTILS_GZ) $(GZ)/$(GCC_GZ) \

$(GZ)/$(BINUTILS_GZ):
  $(WGET) http://ftp.gnu.org/gnu/binutils/$(BINUTILS_GZ)
$(GZ)/$(GCC_GZ):
  $(WGET) https://ftp.gnu.org/gnu/gcc/$(GCC)/$(GCC_GZ)
\end{lstlisting}
