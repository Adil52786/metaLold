\secrel{Бинарный парсинг}\label{binparse}

Большой практической проблемой является анализ бинарных потоков данных. До сих
пор не создано общепринятого инструмента для декларативного описания
произвольного бинарного формата, и тем более средств синтеза кода для анализа
(бесконечных) потоков.

В качестве примера можно привести создание \term{диссекторов} для анализаторов
сетевых протоколов (Wireshark), и чтение любых файловых систем и бинарных
форматов файлов.
 
Бинарные файлы склонны иметь контекстно-чувствительную грамматику, и их разбор
требует средств, способных работать с первым типом по классификации Хомского
\cite{serlex}:

\medskip
\noindent
\href{http://ru.wikipedia.org/wiki/%D0%98%D0%B5%D1%80%D0%B0%D1%80%D1%85%D0%B8%D1%8F_%D0%A5%D0%BE%D0%BC%D1%81%D0%BA%D0%BE%D0%B3%D0%BE}{Иерархия
Хомского} 

\noindent
\href{https://ru.wikipedia.org/wiki/%D0%9A%D0%BE%D0%BD%D1%82%D0%B5%D0%BA%D1%81%D1%82%D0%BD%D0%BE-%D0%B7%D0%B0%D0%B2%D0%B8%D1%81%D0%B8%D0%BC%D0%B0%D1%8F_%D0%B3%D1%80%D0%B0%D0%BC%D0%BC%D0%B0%D1%82%D0%B8%D0%BA%D0%B0}{Контекстно-зависимая
грамматика}

\clearpage
Необходима полная поддержка структур \emph{с бинарными полями произвольной
длины}, (тегированных) объединений, и хранение бинарных данных в форматах
little/bigendian:

\noindent
\url{http://en.wikipedia.org/wiki/Struct_(C_programming_language)}

\noindent
\url{http://en.wikipedia.org/wiki/Tagged_union}

\bigskip
Разбор бинарных потоков требует обязательной поддержки восстановления из
ошибок при получении ``битых'' пакетов, также желательна возможность
их частичного разбора.

\bigskip
\href{https://qspace.library.queensu.ca/bitstream/handle/1974/22040/ElShakankiry_Ali_201708_MSC.pdf}{
Context Sensitive and Secure Parser Generation for
Deep Packet Inspection of Binary Protocols}
\copyright\ Ali ElShakankiry