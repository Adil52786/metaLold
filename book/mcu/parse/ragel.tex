\secrel{Синтаксический разбор команд на ragel}\label{ragel}

\url{https://github.com/calio/ragel-cheat-sheet}

\bigskip
Для обработки команд управления во встраиваемых системах необходим
\term{парсинг} текстового синтаксиса. Для обычных компьютеров существует
множество библиотек и утилит, легко решающих эту проблему: \file{flex/bison},
\file{ANTLR}, \file{PLY},\ldots Но эти средства не подходят для
микроконтроллеров с \emph{крайне ограниченным объемом ОЗУ\note{от
1Кило(!)байта}}, и очень урезанными библиотеками ввода/вывода и буферизации.
Также некоторой проблемой является \emph{применение таблиц разбора} в
сгененированном коде парсера, что сильно \emph{усложняет отладку}. Тем не менее
для микроконтроллеров есть пара утилит, генерирующих вполне вменяемый код
парсеров команд.

\clearpage
\url{http://www.colm.net/open-source/ragel/}

\bigskip
Утилита \file{ragel} умеет генерировать код на \emc\ явно реализующий конечный
автомат для разбора регулярных выражений. Большим достоинством является
\begin{itemize}[nosep]
\item
возможность задавать произвольный код на \emc\ не только \emph{на выходе} из
регулярного выражения,
\item
но и \emph{на входе} при переходе в состояние автомата, соответствующего началу
этого выражения.
\item 
для связи кода парсера с остальной прошивкой достаточно всего двух указателей:
на начало и конец буфера в памяти; никаких внешних библиотек типа \file{libc},
или оберточного кода не требуется
\item 
поддерживаются 8- и 16-битные символы 
\item
возможен разбор бинарных форматов, использующих байтовую организацию:
\term{бинарный парсинг} это отдельная большая тема
\ref{binparse}
\end{itemize}

