\secrel{Реализация ядра \metal\ на \pp}\label{pypy}\secdown

\noindent
\pp\ был задуман как реализация \py, написанная на \py, но благодаря наличию
фреймворка для реализации динамических языков программирования он является также
средством разработки новых языков программирования.
\pp\ состоит из стандартного интерпретатора и транслятора:
\begin{description}
\item[Интерпретатор] полностью реализует язык \py. Сам интерпретатор написан
на ограниченном подмножестве этого же языка, называемом \rpy\ (Restricted
Python). В отличие от стандартного \py, \rpy\ является статически
типизированным для более эффективной компиляции.
\item[Транслятор] является набором инструментов, который анализирует код \rpy\ и
переводит его в языки более низкого уровня, такие как \emc, байт-код \java\ или
CIL. Он также поддерживает подключаемые сборщики мусора и позволяет опционально
включать Stackless.
\item[JIT-компилятор] также \pp\ включает опциональный модуль встроенного
компилятора для трансляции кода в машинные инструкции во время исполнения
программы.
\end{description}

% \clearpage
\subsecly{Установка}

\begin{verbatim}
$ sudo apt update
$ sudo apt install pypy
$ sudo pip install -U rpython
\end{verbatim}

\pp\ может быть запущен как обычный интерпретатор:

\begin{verbatim}
$ pypy py/hello.py

hello
\end{verbatim}

\clearpage
Для получения быстрого исполняемого файла запускаем транслятор \rpy:
\begin{verbatim}
$ rpython hello.py
\end{verbatim}

% \bigskip
% \href{https://buildmedia.readthedocs.org/media/pdf/building-an-interpreter-with-rpython/latest/building-an-interpreter-with-rpython.pdf}{Tutorial
% on CyCy (интерпретатор Си)}

\begin{itemize}
  \item 
\href{https://morepypy.blogspot.com/2011/04/tutorial-writing-interpreter-with-pypy.html}{Andrew
Brown tutorial part 1}
\end{itemize}

\clearpage
Транслируемый модуль должен определять функцию \fn{target()} которая возвращает
точку входа. Транслятор работает с использованием метода \term{трассирующей
компиляции}, вызывая импортированный модуль через точку входа.

\lst{py/hello00.py}{language=Python}

\secup
