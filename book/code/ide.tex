\secrel{IDE: контуженные Борландом}\label{ide}\secdown

\fig{code/VS.png}{height=.31\textheight}
\fig{code/turboC.jpg}{height=.31\textheight}
\fig{code/IAR.png}{height=.31\textheight}

\clearpage
Если считать начиная с первых версий языка \st\ и рабочих станций компании
Xerox, история интегрированных сред разработки (IDE) насчитывает уже больше 40
(!) лет. И за это время прогресс IDE так и застрял на уровне мультиоконников
в стиле Borland.

Впрочем, одна вменяемая среда все же есть\ --- Emacs, но к
сожалению \term{средства расширения пользователем} используют довольно
специфичный \lisp.

Другая \emph{IDE-платформа} \eclipse\ по задумке выглядела перспективной, но
расширение с помощью \java\ убивает всю идею сложностью вхождения пользователю,
которому нужно только чуть-чуть поправить поведение типового редактора, или
добавить простую визуализацию \ref{ideviz}.

Хорошей базой для разработки средств разработки\note{метаIDE, или
\term{метасреда}} могут быть современные \st-системы типа
\href{https://pharo.org}{Pharo}, сочетающие легкий для освоения скриптовый язык,
встроенный интерактивный отладчик, и богатый GUI. К сожалению ценовая политика
поставщиков реализаций \st\ убила великолепный язык программирования, поэтому
его развитие почти остановилось из-за очень маленького пользовательского
сообщества.

\secrel{Проблемы традиционных IDE}\secdown

\secrel{Фиксация на файловом представлении}

Разработка фиксируется на редактировании файлов исходного кода, полностью
игнорируя критическую проблему\ --- \emph{необходимость передачи знаний между
разработчиками} \ref{litprog}.

Использование файлов хорошо совместимо с mainstream хранением публичных проектов
в репозиториях на GitHub, и необходимо для работы любых компиляторов, но по
факту \emph{необходимо хранение в форматах представления знаний}\ --- объектные
СУБД, семантические сети и т.п.

\secrel{Необходим фронтенд компилятора}

Для работы с исходным кодом необходима реализация большой части фронтенда
компилятора: синтаксический разбор для подсветки синтаксиса, построение таблиц
символов для ссылок и переходов по исходному тексту, препроцессор для скрытия
неиспользуемого кода, completion по программным обхектам, подстветка
синтаксических ошибок,\ldots

\secrel{MDI стиль интерфейса}

Большинство интерфейсов IDE построено по принципу множества окон в общем окне
среды. Каждая IDE пытается изобрести собственный встроенный \term{менедер окон},
вместо того чтобы позволить ОС выполнять свою прямую обязанность:
отображать рабочий стол унифицированно с любыми другими приложениями, и так же
унифицированно обеспечивать передачу объектов между приложениями.
Впрочем, эта проблема легко объясняется ублюдочностью Windows, у которой
менеджера окон никогда и не существовало, а передача объектов ограничивалась
только картинкой, текстом, и копипастой в соседнее окно \textit{того же
самого} приложения.

\secrel{Полностью отсутствуют средства визуализации}\label{ideviz}

Средства визуализации и редактирования структур данных специфичным для них
способом не рассматривается как необходимый функционал для разработки и отладки
программ. При использовании отладчика просмотр данных возможен максимум в виде
списка полей, с крайне ограниченным набором декодеров: hex, decimal и усе.

\secrel{Расширение пользователем усложнено}

IDE поддерживающие плагины или скрипты, требуют тяжелых в освоении языков
(\java\ или \lisp).

\secrel{Не поддерживается инкрементная компиляция}

Проблема специфична для микроконтроллеров: любая самая минимальная модификация
кода требует полной перекомпиляции, перезаливки всей прошивки и рестарта
процессора. Это ограничение компилятора/линкера и ПО программатора, которые не
умеют инкрементную компиляцию.

\secup

\clearpage
\secrel{Критически необходимо Literate Programming}\label{litprog}

\href{https://en.wikipedia.org/wiki/Literate_programming}{Литературное
программирование}\ (LP) предложено Доналдом Кнутом как решение проблемы
\emph{передачи знаний между разработчиками}. Традиционная разработка уже 60 лет
фиксируется на редактировании файлов исходного кода. В ранние года развития ЭВМ
такой подход вполне понятен\ --- вычислительных ресурсов едва хватало. Но этот
явный косяк в подходах не был увиден и решен даже в конце 90х, когда объемы
ОЗУ исчислялились мегабайтами, и жесткий диск достаточных объемов стоял в каждом
персональном компьютере.

Сейчас широко используются различные костыльные решения типа \file{Doxygen},
\file{javadoc} и т.п., но их функционал явно недостаточен для полноценного
документирования. Они вполне применимы для справочников по API библиотек, но
документирование требует применение гипертекста, диаграмм, а иногда и сложной
математической верстки.

Парадигма LP предполагает полное объяснение логики программы на естественном
языке, в формате пояснительной записки, в которую включаются фрагменты исходного
кода, передаваемого компилятору при сборке программы. \emph{При документировании
ПО необходим функционал сквозной увязки объектов в исходном коде, и элементов
документации}\ --- такая увязка при полноценном применении LP должна
обеспечиватся представлением проекта как \term{документной базы знаний}\note{с
поддержкой перекрестных ссылок между документацией и комментариями в исходном
коде, построением индексов объектов и терминов, нечетким поиском, визуализацией
структуры программ, и т.п.}.

К сожалению, классическое LP оказалось неприменимо из-за важной особенности\ ---
\emph{каждый фрагмент исходного кода (\term{блок программы}) должен быть описан
полностью}, так как на компиляцию он передается целиком. На практике такой
подход неприменим: мы должны полностью увязать текст документации с каждым
программным блоком не только по количеству использований (только один раз), но и
по порядку\ --- декларация функций и переменных должна идти до их первого
использования.

В то же время, \emph{документация на программный продукт\note{или
программно-аппаратный комплекс\ --- в этой книге это подразумевается, так как мы
говорим о встраиваемых системах}\ принципиально нелинейна}: вы это легко сможете
увидеть сами, если попытаетесь написать книгу по какой-нибудь достаточно сложной
программе. Книга предполагает линейное чтение, компоненты программы описываются
с разных точек зрения и в разных местах текста.
\begin{itemize}[nosep]
\item
В руководстве пользователя рассматривается интерфейс, и стыковка с внешними
системами.
\item
В руководстве программиста\note{для пользователя который планирует расширять
систему самостоятельно}\ --- API и внутреннее устройство, причем основная
реализация функция может быть описана в одном разделе, а код обеспечивающий
безопасность той же функции в другом.
\end{itemize}

Кнут предполагал уйти от написания программ с точки зрения компьютера, и
позволить программистам разрабатывать программы способами и очередностью кода,
определяемыми ходом мышления разработчика.

Пакет \file{WEB}, который использовал Кнут, работал в пакетном режиме, читал
``литературный'' код, и генерировал из него как файлы исходного кода для
компилятора, так и файлы на языке разметки документации\note{\TeX}.
В практическом смысле реализация LP должна быть сделана в IDE в виде
интерактивного просмотра документации. Изменение блоков кода должно отображаться
немедленно, или при обновлении страницы документации по команде.



\secup
