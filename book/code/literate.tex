\clearpage
\secrel{Критически необходимо Literate Programming}\label{litprog}

\href{https://en.wikipedia.org/wiki/Literate_programming}{Литературное
программирование}\ (LP) предложено Доналдом Кнутом как решение проблемы
\emph{передачи знаний между разработчиками}. Традиционная разработка уже 60 лет
фиксируется на редактировании файлов исходного кода. В ранние года развития ЭВМ
такой подход вполне понятен\ --- вычислительных ресурсов едва хватало. Но этот
явный косяк в подходах не был увиден и решен даже в конце 90х, когда объемы
ОЗУ исчислялились мегабайтами, и жесткий диск достаточных объемов стоял в каждом
персональном компьютере.

Сейчас широко используются различные костыльные решения типа \file{Doxygen},
\file{javadoc} и т.п., но их функционал явно недостаточен для полноценного
документирования. Они вполне применимы для справочников по API библиотек, но
документирование требует применение гипертекста, диаграмм, а иногда и сложной
математической верстки.

Парадигма LP предполагает полное объяснение логики программы на естественном
языке, в формате пояснительной записки, в которую включаются фрагменты исходного
кода, передаваемого компилятору при сборке программы. \emph{При документировании
ПО необходим функционал сквозной увязки объектов в исходном коде, и элементов
документации}\ --- такая увязка при полноценном применении LP должна
обеспечиватся представлением проекта как \term{документной базы знаний}\note{с
поддержкой перекрестных ссылок между документацией и комментариями в исходном
коде, построением индексов объектов и терминов, нечетким поиском, визуализацией
структуры программ, и т.п.}.

К сожалению, классическое LP оказалось неприменимо из-за важной особенности\ ---
\emph{каждый фрагмент исходного кода (\term{блок программы}) должен быть описан
полностью}, так как на компиляцию он передается целиком. На практике такой
подход неприменим: мы должны полностью увязать текст документации с каждым
программным блоком не только по количеству использований (только один раз), но и
по порядку\ --- декларация функций и переменных должна идти до их первого
использования.

В то же время, \emph{документация на программный продукт\note{или
программно-аппаратный комплекс\ --- в этой книге это подразумевается, так как мы
говорим о встраиваемых системах}\ принципиально нелинейна}: вы это легко сможете
увидеть сами, если попытаетесь написать книгу по какой-нибудь достаточно сложной
программе. Книга предполагает линейное чтение, компоненты программы описываются
с разных точек зрения и в разных местах текста.
\begin{itemize}[nosep]
\item
В руководстве пользователя рассматривается интерфейс, и стыковка с внешними
системами.
\item
В руководстве программиста\note{для пользователя который планирует расширять
систему самостоятельно}\ --- API и внутреннее устройство, причем основная
реализация функция может быть описана в одном разделе, а код обеспечивающий
безопасность той же функции в другом.
\end{itemize}

Кнут предполагал уйти от написания программ с точки зрения компьютера, и
позволить программистам разрабатывать программы способами и очередностью кода,
определяемыми ходом мышления разработчика.

Пакет \file{WEB}, который использовал Кнут, работал в пакетном режиме, читал
``литературный'' код, и генерировал из него как файлы исходного кода для
компилятора, так и файлы на языке разметки документации\note{\TeX}.
В практическом смысле реализация LP должна быть сделана в IDE в виде
интерактивного просмотра документации. Изменение блоков кода должно отображаться
немедленно, или при обновлении страницы документации по команде.

