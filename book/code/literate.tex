\clearpage
\secrel{Критически необходимо Literate Programming}\label{litprog}

\href{https://en.wikipedia.org/wiki/Literate_programming}{Литературное
программирование}\ (LP) предложено Доналдом Кнутом как решение проблемы
\emph{передачи знаний между разработчиками}. Традиционная разработка уже 60 лет
фиксируется на редактировании файлов исходного кода. В ранние года развития ЭВМ
такой подход вполне понятен\ --- вычислительных ресурсов едва хватало. Но этот
явный косяк в подходах не был увиден и решен даже в конце 90х, когда объемы
ОЗУ исчислялились мегабайтами, и жесткий диск достаточных объемов стоял в каждом
персональном компьютере.

Сейчас широко используются различные костыльные решения типа \file{Doxygen},
\file{javadoc} и т.п., но их функционал явно недостаточен для полноценного
документирования. Они вполне применимы для справочников по API библиотек, но
документирование требует применение гипертекста, диаграмм, а иногда и сложной
математической верстки.

Парадигма LP предполагает полное объяснение логики программы на естественном
языке, в формате пояснительной записки, в которую включаются фрагменты исходного
кода, передаваемого компилятору при сборке программы. \emph{При документировании
ПО необходим функционал сквозной увязки объектов в исходном коде, и элементов
документации}\ --- такая увязка при полноценном применении LP должна
обеспечиватся представлением проекта как \term{документной базы знаний}\note{с
поддержкой перекрестных ссылок между документацией и комментариями в исходном
коде, построением индексов объектов и терминов, нечетким поиском, визуализацией
структуры программ, и т.п.}.

К сожалению, классическое LP оказалось неприменимо из-за важной особенности\ ---
\emph{каждый фрагмент исходного кода (\term{блок программы}) должен быть описан
полностью}, так как на компиляцию он передается целиком. На практике такой
подход неприменим: мы должны полностью увязать текст документации с каждым
программным блоком не только по количеству использований (только один раз), но
\emph{и по порядку}\ --- декларация функций и переменных должна идти до их
первого использования.

В то же время, \emph{документация на программный продукт\note{или
программно-аппаратный комплекс\ --- в этой книге это подразумевается, так как мы
говорим о встраиваемых системах}\ принципиально нелинейна}: вы это легко сможете
увидеть сами, если попытаетесь написать книгу по какой-нибудь достаточно сложной
программе. Книга предполагает линейное чтение, компоненты программы описываются
с разных точек зрения и в разных местах текста.
\begin{itemize}[nosep]
\item
В руководстве пользователя рассматривается интерфейс, и стыковка с внешними
системами.
\item
В руководстве программиста\note{для пользователя который планирует расширять
систему самостоятельно}\ --- API и внутреннее устройство, причем основная
реализация функция может быть описана в одном разделе, а код обеспечивающий
безопасность той же функции в другом.
\end{itemize}

Кнут предполагал уйти от написания программ с точки зрения компьютера, и
позволить программистам разрабатывать программы способами и очередностью кода,
определяемыми ходом мышления разработчика.

Пакет \file{WEB}, который использовал Кнут, работал в пакетном режиме, читал
``литературный'' код, и генерировал из него как файлы исходного кода для
компилятора, так и файлы на языке разметки документации\note{\TeX}.
В практическом смысле реализация LP должна быть сделана в IDE в виде
интерактивного просмотра документации. Изменение блоков кода должно отображаться
немедленно, или при обновлении страницы документации по команде.

Первичность документации перед исходным кодом, и описание компонентов программы
в визуально-представимом виде стимулируют разработчика не только писать краткие
записи по вносимому коду, но и синхронизировать описание с модифицируемым кодом.
Традиционно для этого используются комментарии в коде, но система интерактивного
ввода позволяет набрасывать краткие записи\note{для таких sticky-записей вполне
подходят не только векторные диаграммы с активными элементами-ссылками на
программные элементы, но и просто ``кроки'' нарисованные мышью от руки}\ по ходу
работы, что в дальнейшем спасет legacy-разгребальщика от многодневных сессий в
отладчике и тонно-литров кофе в качестве антидепрессанта.

\clearpage
\secrel{Концепт vs модель}\label{concept}

\metal\ --- язык не только мета-, но и \term{концептуального программирования}.
\term{Концепт}\ это \emph{не полностью определенная} модель. Например, указав

\begin{lstlisting}
module: hello
\end{lstlisting}
\noindent
мы сужаем множество всех возможных программ до одного программного модуля.

\begin{lstlisting}
function: main << ( добавить в module:hello )
\end{lstlisting}
\noindent
накладывает ограничение на то что целевая программа будет на \emc/\cpp.

\medskip
В mainstream языках программирования мы досконально описываем каждый элемент
процесса, пошагово, точно прописываем все свойства объектов и структуру
программы. При программировании на \metal\ мы постепенно \emph{накладываем
ограничения}, сужая все множество возможных программ с помощью условий.
При этом чем больше и больше ограничений мы применям, тем ближе концепт
программы становится ее моделью:
\begin{description}
\item[модель] точное поэлементное описание системы с указанием всех
взаимосвязей между объектами.
\item[концепт] более широкое понятие так как он преднамеренно \emph{описывает
систему только частично}, указывая что у объекта \emph{может быть} такое-то
свойство, есть такой-то класс с некоторыми методами (но при этом не
декларируются \emph{все методы}).
\end{description}

\noindent
Для реализации \term{концептуального программирования} \cite{tyugu} необходима
\term{база знаний} в которой задаются
\begin{itemize}
  \item 
\emph{наборы ограничений}, которые мы называем \term{концептами},
\item 
\emph{куски кода}\ --- \term{сниппеты}, которые добавляются в результирующий
код при срабатывании правил и ограничений, причем эти куски \emph{заданы
параметрически} и частично изменяются при использовании
 \item
предыдущие проекты, заданные как наборы ограничений\ --- вы можете
\term{наследовать} их целиком или частями
\item 
\emph{правила логического вывода} которые осуществляют синтез кода по сети
ограничений
\item 
\term{скрипты}\ --- \term{императивные} процедуры, выполняемые внутри базы
знаний, для ее анализа или трансформации
\item
\term{демоны}\ --- скрипты, выполнемые в фоновом режиме: оптимизации, сборка
мусора, JIT- и фоновая компиляция скриптов и вычислительных функций в машинный
код
\end{itemize} 

