\secrel{emCin: загрузка кода на embedded \emc}\label{emcin}

Даже если каким-то чудом вам удастся найти триалку коммерческого интерактивного
анализатора кода или \href{https://www.sourcetrail.com/}{Sourcetrail}, внезапно
выяснится что:
\begin{itemize}[nosep]
  \item диалект \emc\ исходников вашей прошивки не поддерживается,
  \item при попытке загрузки проекта вываливается 100500 сообщений об ошибках
  синтаксиса
  \item анализатор умеет работать только с \cpp,
  \item ничего не знает про особенности кодирования под микроконтроллеры, и
  \item неспособен загрузить исходный код ядра Linux с учетом всей пары сотен
  настроек конфигурации, и уж тем более
  \item не имеет никакого понятия о Makefile, autohell, файлах проектов IAR и
  двух десятках сборочных .batников, наклёпанных кем-то из пяти ваших
  предшественников.
\end{itemize}

\clearpage
Короче, вы попали.
\bigskip

Можете не надеяться что, прочитав эту главу, вы сможете наклепать
супер-пупер-анализатор с рефакторингом и автогенератором тестов. Чтобы написать
инстумент, способный хотя бы обеспечить приличную навигацию по исходному коду,
нужна команда разработчиков, по квалификации стремящаяся к JetBrains.

Максимум что я могу вам предложить\ --- сделайте пару пробных шагов, вдруг вы
загонитесь, и запустите проект по разработке CodeShit Studio.
