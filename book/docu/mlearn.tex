\secrel{M-learning: мобильное обучение}

\url{https://en.wikipedia.org/wiki/M-learning}

\bigskip
M-learning или \term{мобильное обучение}\ --- это «обучение в разных контекстах,
через социальные и контентные взаимодействия, с использованием персональных
электронных устройств. \emph{Форма дистанционного обучения}, 
использующая электронные образовательные технологии на мобильных устройствах
\emph{в удобное для обучающихся время}.

M-Learning ориентируется на мобильность учащегося, взаимодействующего с
портативными технологиями. Использование мобильных инструментов для создания
учебных пособий и материалов становится важной частью неформального обучения.

M-learning удобен тем, что доступен практически из любого места. Совместное
использование практически мгновенно среди всех, кто использует один и тот же
контент, что приводит к получению мгновенных отзывов и подсказок. Этот
высокоактивный процесс, как оказалось, улучшает результаты экзаменов с
50\% до 70\%, и сокрашает показатель отсева в
технических областях на 22\%. M-Learning также обеспечивает высокую
мобильность, заменяя книги и заметки небольшими устройствами, заполненными
специальным обучающим контентом. M-Learning имеет дополнительное преимущество,
заключающееся в том, что он эффективен с точки зрения затрат, поскольку цена
цифрового контента на планшетах резко падает по сравнению с традиционными
носителями (книги, CD и DVD и т.д.). Например, один цифровой учебник стоит от
одной трети до половины стоимости комплекта бумажных учебников, одновременно
предоставляя возможности: поиск, пользовательские ссылки, копирование учебных
материалов с произвольным редактированием.
 

