\secrel{Метод Монте-Карло}\label{lrnkarlo}

\clearpage
Случайный показ произвольного объекта документации, включая узлы, которые на
него ссылаются, и узлы, на которые ведут исходящие ссылки. В результате
формируется \term{начальная точка просмотра} включающая взаимосвязи.
По каждому элементу пользователю предъявляется статистика покрытия, для
выявляения элементов документации, по которым пользователь прошел недостаточное
обучение.

\begin{itemize}
  \item 
Частота предъявления элементов выбирается в соответствии со статистикой
обучения, приоритет отдается областям \emph{смежным с наиболее посещаемыми} (они
интересуют пользователя так как статистика накручивается в процессе работы в
поисках ответов на проблемы).
  \item 
Одновременно в выборку \emph{добавляются области с
минимальным посещением, чтобы обеспечить равномерное покрытие} документации
просмотрами.
\end{itemize}
