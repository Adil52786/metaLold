% e-book
% Universal LaTeX headers for e-book publications
\documentclass[oneside,10pt]{book}
%% mobile phone optimized
\usepackage[paperwidth=118.8mm,paperheight=68.2mm,margin=2mm]{geometry}
%% font setup for screen reading
\renewcommand{\familydefault}{\sfdefault}\normalfont
%% hyperlinks pdf style
\usepackage[unicode,colorlinks=true]{hyperref}
%% fix heading styles for tiny paper
\usepackage{titlesec}
\titleformat{\chapter}{\Large\bfseries}{\thechapter.}{1em}{}
\titleformat{\section}{\large\bfseries}{\thesection.}{1em}{}
%% fix first blank page
\usepackage{atbegshi}% http://ctan.org/pkg/atbegshi
\AtBeginDocument{\AtBeginShipoutNext{\AtBeginShipoutDiscard}}
% graphics
\usepackage[pdftex]{graphicx}
\usepackage{calc}
\newcommand{\fig}[2]{\noindent\includegraphics[#2]{#1}}
\newcommand{\tfig}[2]{\raisebox{2ex-\height}{\noindent\includegraphics[#2]{#1}}}

%% bibliography
%\usepackage{titlesec}
\newcommand{\bibfig}[1]{\fig{#1}{height=.56\textheight}}

% xcolor fixes
\usepackage{xcolor}
\definecolor{red}{rgb}{0.7,0,0}
\definecolor{green}{rgb}{0,0.3,0}
\definecolor{blue}{rgb}{0,0,0.5}
\definecolor{magenta}{rgb}{0.7,0,0.7}

% Cyrillization
%% \usepackage[T1,T2A]{fontenc}
\usepackage[utf8]{inputenc}
%% \usepackage[cp1251]{inputenc}
\usepackage[english,russian]{babel}
\usepackage{indentfirst}

% relative sectioning
\usepackage{ifthen}
\newcounter{secdepth}\setcounter{secdepth}{0}
\newcommand{\secup}{\addtocounter{secdepth}{1}}
\newcommand{\secdown}{\addtocounter{secdepth}{-1}}
\newcommand{\secrel}[1]{
\ifthenelse{\equal{\value{secdepth}}{0}}{\part{#1}}{}
\ifthenelse{\equal{\value{secdepth}}{-1}}{\chapter{#1}}{}
\ifthenelse{\equal{\value{secdepth}}{-2}}{\section{#1}}{}
\ifthenelse{\equal{\value{secdepth}}{-3}}{\subsection{#1}}{}
\ifthenelse{\equal{\value{secdepth}}{-4}}{\subsubsection{#1}}{}
}
\newcommand{\secly}[1]{
\section*{#1}
\addcontentsline{toc}{section}{#1}
}
\newcommand{\subsecly}[1]{
\subsection*{#1}
\addcontentsline{toc}{subsection}{#1}
}

% misc
%% [nosep] option in lists/enums
\usepackage{enumitem}
%% complex tables
\usepackage{multirow}
%% frames
\usepackage{framed}

%% typical macros
\newcommand{\email}[1]{$<$\href{mailto:#1}{#1}$>$}
\newcommand{\note}[1]{\footnote{\ #1}}
\renewcommand{\emph}[1]{\textcolor{blue}{#1}}
\newcommand{\cp}[1]{\note{\copyright\ #1}}
\newcommand{\term}[1]{\textcolor{green}{\textit{#1}}}
\newcommand{\termdef}[2]{\textcolor{red}{\textbf{\textit{#1}}}\index{#2}}

\newcommand{\file}[1]{\texttt{#1}}
\newcommand{\class}[1]{\texttt{#1}}
\newcommand{\var}[1]{\texttt{#1}}
\newcommand{\fn}[1]{\texttt{#1}}

% comp
\newcommand{\emc}{$C$}
\newcommand{\cpp}{$C^{+_+}$}
\newcommand{\java}{$Java$}
\newcommand{\py}{\texttt{\textbf{Python}}}
\newcommand{\rpy}{\texttt{\textbf{RPython}}}
\newcommand{\pp}{\texttt{\textbf{PyPy}}}
\newcommand{\lisp}{$Lisp$}
\newcommand{\F}{\texttt{\textbf{Форт}}}
\newcommand{\js}{$JS$}
\newcommand{\metal}{\texttt{\textbf{metaL}}}
\newcommand{\prolog}{\texttt{\textbf{Prolog}}}
\newcommand{\pyf}{$pyF$}
\newcommand{\st}{$Smalltalk$}
\newcommand{\self}{$Self$}
\newcommand{\A}{Android}
\newcommand{\eclipse}{Eclipse}
\newcommand{\linux}{Linux}
\newcommand{\win}{Windows}
\newcommand{\wasm}{WASM}
\newcommand{\ems}{Emscripten}

\newcommand{\cm}[1]{Cortex-M#1}

%% listings
\usepackage{verbatim}
\usepackage{listings}
\lstset{
basicstyle=\small,
frame=single,
numbers=left,numberstyle=\small,numbersep=2mm,
xleftmargin=3mm,xrightmargin=2mm,
tabsize=4,
keywordstyle=\color{red},
commentstyle=\color{blue},
extendedchars=\true,
% fix russian comments: https://dxdy.ru/post509540.html
keepspaces = true,
%breaklines=false,breakatwhitespace=true
}
\newcounter{lstcounter}
\newcommand{\lst}[2]{\lstinputlisting[#2]{#1}}
% \noindent\refstepcounter{lstcounter}
% \noindent\begin{minipage}{\textwidth}
% \lstinputlisting[#2]{#1}
% \end{minipage}\\}


\author{Dmitry Ponyatov \email{dponyatov@gmail.com} CC BY-NC-ND}
\title{{\Huge \ \\\hico}\\пишем язык программирования\\на Python}
\date{draft: \today}

\begin{document}

\maketitle
\tableofcontents

\secly{Введение}\secdown

github: \url{https://github.com/ponyatov/hico}

\bigskip

Бесплатный черновик книги вы можете скачать на странице релизов:

\url{https://github.com/ponyatov/hico/releases/latest}

\bigskip\noindent
Замечания и комментарии присылайте на e-mail\\или открывайте issue на github.

\clearpage
\subsecly{Отказ от ответственности}

Эта книга посвящена очень \emph{примитивной} реализации того, что особо
грамотные товарищи никогда не захотят назвать языком программирования. Здесь вы
не найдете ленивых лямбд и жутких монад, живущих в лесу Хомского, и прочей
бурбулятристики про обощенный вывод типов. Тем не менее я постараюсь что-нибудь
добавить в отсутствующую нишу русскоязычной литературы по разработке языков
программирования \emph{для самых начинающих}.

Использованные методы и алгоритмы топорны, совершенно безграмотны и 
неэффективны. Язык реализации убог, императивен и багопротивен. Подача материала
больше похожа на поток сознания, рассматриваются только вещи, которые стыдно
упомянать даже студенту первого курса заочного отделения, все выводы
противоречат общепризнанным аксиомам, а достигнутые результаты смехотворны, и
напоминают рык комара.

\clearpage
\subsecly{Метапрограммирование}\label{meta}

\begin{quotation}\noindent
Метапрограммирование — вид программирования, связанный с созданием
\textit{программ, которые порождают другие программы} как результат своей работы
(в частности, на стадии компиляции их исходного кода), либо программ, которые
меняют себя во время выполнения (самомодифицирующийся код).
\end{quotation}

\begin{itemize}
  \item 
\url{https://www.youtube.com/watch?v=QKFrxEkVusg}
  \item 
\url{https://www.youtube.com/watch?v=bt6kU1kuHWA}
\end{itemize}

\noindent
Традиционно при написании программ стараются писать код максимально переносимым
между различными компиляторами. ОС и аппаратурой, для этого создают различные
фреймворки, HAL, стандартные библиотеки и т.п. В итоге вместо быстрых
эффективных программ получаются \textbf{Jаба}троны завернутые в десятки слоев
абстракций и выжирающих ОЗУ гигабайтами\note{Eclipse на запуске на пару минут
вырубает не самый тухлый i7}.
Метапрограммирование через генерацию кода способно решить обратную задачу:
получение исходного кода на \textit{embedded \emc}\ \note{\cpp, \java\ или любом
другом языке, в т.ч. на \py\ для самораскрутки системы}\ максимально
учитывающего все особенности используемой аппаратуры, окружения и конкретной
решаемой задачи. Общая идея\ ---
\begin{itemize}[nosep]
  \item 
\emph{шаблонизация}, 
  \item 
\emph{параметризация} и 
  \item 
\emph{наследование} \textbf{исходного кода}
\end{itemize}
написанного на языках программирования, которые в принипе не знают об ООП,
наследовании и шаблонах (ISO \emc, Makefile, МЭК 61131-3), или не способных их
полноценно реализовать\ \note{интерересно через сколько десятилетий наконец
додумаются встроить в компилятор \cpp\ интерпретатор (\lisp а?) для построения
кода в compile time, вместо сомнительных шаблонов?}. Вся абстрактная каша должна
оставаться на рабочей станции разработчика в высокоуровневом \py-коде,
результат\ --- низкоуровневый код на \emc/LLVM способный работать на сотнях байт
ОЗУ\note{типичное требование для прошивок аппаратуры, сделанных на дешевых
low-end микроконтроллерах, имеюших всего \emph{2+ Кило}байта ОЗУ}.

Идеальным результатом применения metapy будет код, не выполняющий ни одной
машинной инструкции, которая не является необходимой для инициализации
конкретной железки, или решения текущей задачи. Если код должен работать поверх
ОС, в идеале он должен использовать только нативный API и
\term{специфицированный} код
вместо сторонних библиотек и особенно мультиплатформенных фрейморков. В
реальности естественно приходится ограничиваться точечным применением, т.к. есть
legacy код, требования к читаемости выходного кода, обучение программистов
сложной методике, сложность реализации вывода (компиляция мета-моделей), и
невозможность переписать в виде метамоделей весь используемый набор сервисов и
библиотек.

% \clearpage
\subsecly{Гомоиконичные языки программирования}\label{homoiconic}

\begin{quotation}\noindent
\term{Гомоиконичность} (гомоиконносль, англ. homoiconicity, homoiconic)\\
свойство некоторых языков программирования, в которых \emph{представление
программ является одновременно структурами данных} определенных в типах самого
языка, \emph{доступных для просмотра и модификации}. Говоря иначе,
гомоиконичность\ --- это когда исходный \textit{код программы} пишется
\textit{как базовая структура данных}, и язык программирования знает, как
получить к ней доступ (в том числе в рантайме).
\end{quotation}

\noindent
В гомоиконичном языке \textit{метапрограммирование это основная методика}
разработки ПО, использующаяся в том числе и \textit{для расширения языка} до
возможностей, нужных конкретному программисту.

В качестве первого примера всегда приводится язык \lisp, который был создан для
обработки данных, представленных в форме вложенных списков.
Лисп-программы тоже записываются, хранятся и выполняются в виде списков; в
результате получается, что программа во время работы может получить доступ к
своему собственному исходному кода, а также автоматически изменять себя «на
лету». Гомоиконичные языки, как правило, включают полную поддержку
\term{синтаксических макросов}, позволяющие программисту определять новые
синтаксические структуры, и выражать \term{преобразования программ} в компактной
форме.

\subsecly{Серебряная пуля Брукса}

\noindent
очень вольный, отрывочный и местами противоречащий оригиналу перевод: 
\textit{Frederick P. Brooks} \textbf{No Silver Bullet\ --- Essence and Accident
in Software Engineering} 1986
\href{http://worrydream.com/refs/Brooks-NoSilverBullet.pdf}{.pdf}
\bigskip

Сущностью программирования является, прежде всего, не написание инструкций на
конкретном языке программирования, а выработка подробной структуры
взаимодействующих сущностей проблемной области, а также проверка внутренней
непротиворечивости этой структуры.
Следовательно, ни одно средство разработки ПО не сможет существенно снизить
сложность разработки, так как даже если, например, изобрести компьютерный язык,
оперирующий понятиями на уровне проблемной области, программирование все равно
останется сложной задачей, поскольку придется точно определять взаимосвязи между
объектами реального мира, устанавливать исключения, предусматривать все
возможные переходы между состояниями и т.д.

Что делает язык высокого уровня? Они изолируют программу от большей части ее
нецелевой сложности. Абстрактная программа состоит из концептуальных
конструкций: операций, типов данных, последовательностей и коммуникации.
Конкретная машинная программа связана с битами, регистрами, условиями, ветвями,
каналами, дисками и тому подобным. В той степени, в которой язык высокого уровня
воплощает конструкции, которые нужны в абстрактной программе, и избегает всех
нижестоящих, он устраняет целый уровень сложности, который вообще никогда не был
присущ программе. Безусловно, уровень нашего мышления о структурах данных, типах
и операциях неуклонно растет в сторону прикладной области благодаря
возможностям абстракции, предоставляемым ООП, но с постоянно уменьшающейся
скоростью\ --- развитие языков и абстракий движется все ближе и ближе к
прикладной сложности пользователей. Более того, в какой-то момент разработка
языка высокого уровня и фреймворков создает бремя владения инструментом, которое
увеличивает, а не уменьшает интеллектуальную сложность пользователя.

Многие люди ожидают, что достижения в области искусственного интеллекта
обеспечат революционный прорыв, который даст увеличение производительности и
качества программного обеспечения на порядок. При этом не стоит путать
"численный ИИ" как он широко известен сейчас, т.е. нейроные сети и машинное
обучение, с \term{семантическим ИИ}, выполняющим логический вывод \emph{на
основе сетей взаимосвязанных понятий и отношений между объектами}. Наиболее
широко известная технология семантического ИИ\ --- \term{экспертные системы}.

\begin{quotation}\noindent
Экспертная система\ --- это программа, которая содержит обобщенный механизм
логического вывода и базу правил и отношений, принимает входные данные и
предположения, генерурет гипотезы, выводимые из базы правил, дает выводы и
рекомендации, и предлагает объяснение свои результатов для пользователя (путем
отслеживания цепочки логического вывода). Механизмы вывода часто могут иметь
дело с нечеткими или вероятностными данными и правилами, в дополнение к чисто
детерминированной логике.
\end{quotation}

Как эта технология может быть применена к задаче разработки программного
обеспечения?

Работа, необходимая для генерации базы знаний\ --- это работа, которую в любом
случае необходимо будет не просто выполнить единожды, но и постоянно
поддерживать базу знаний в актуальном состоянии. Многие трудности стоят на пути
скорейшей реализации полезных экспертных системных советников для разработчика
программ. Важной частью нашего воображаемого сценария является разработка
простых способов перехода от спецификации структуры программы к автоматической
или полуавтоматической генерации кода, созданию правил тестирования и
диагностики, скриптов развертывания и средств мониторинга. Еще более трудной и
важной проблемой является получение знаний в двух направлениях: поиск четких,
самоаналитичных экспертов, которые знают, почему они делают что-то, и разработка
эффективных методов извлечения того, что они знают, и выражение их знаний в базу
правил. Необходимым условием для построения экспертной системы является наличие
эксперта.

Самым мощным вкладом экспертных систем, безусловно, должно быть предоставление
на службу неопытному программисту опыта и накопленной мудрости лучших
программистов. Это немалый вклад. Разрыв между лучшей практикой разработки
программного обеспечения и средней практикой очень велик\ --- возможно, больше,
чем в любой другой инженерной дисциплине. Инструмент, который распространяет
передовой опыт, был бы важен.

В течение почти 50 лет люди ожидали и писали об «автоматическом
программировании» или создании программ на основе формулировки спецификаций
проблемы. Некоторые сегодня пишут, как будто они ожидают, что эта технология
обеспечит следующий прорыв. Короче говоря, автоматическое программирование
всегда было эвфемизмом для программирования на языке более высокого уровня, чем
было доступно в настоящее время программисту.

Техника создания \term{генераторов кода} очень мощная, и она обычно дает хорошие
преимущества для программ \term{синтаксического разбора} и реализаций протоколов
обмена данными. 

Графическое программирование. Любимый предмет для докторской диссертации в
области разработки программного обеспечения представляют собой графическое или
визуальное программирование\ --- применение компьютерной графики к разработке
программного обеспечения. Предполагаемый успех, обеспечиваемоый таким подходом,
постулируется по аналогии с САПР для микросхем СБИС, в которых компьютерная
графика играет плодотворную роль. Иногда теоретик оправдывает такой подход,
рассматривая блок-схемы как идеальную среду разработки программ и предоставляя
мощные средства для их построения. Тем не менее ничего, более-менее
убедительного, не говоря о чем-то большем, еще не появилось в результате таких
усилий.

Блок-схема является очень плохой абстракцией структуры программного обеспечения.
В жалкой, многостраничной, топорной форме, к которой сегодня сводится
блок-схема, она оказалась бесполезной в качестве средства разработки ПО\ ---
программисты рисуют блок-схемы после написания описываемых ими программ, а не
раньше.

Самая сложная часть построения программной системы\ --- это решить, что именно
строить. Никакая другая часть концептуальной работы не является такой сложной,
как разработка подробных технических требований, включая все интерфейсы для
людей, машин и других программных систем. Никакая другая часть работы не наносит
больший вред полученной системе, если она сделана неправильно. Никакую другую
часть дизайна не сложнее исправить если допущены ошибки.

Следовательно, наиболее важной функцией, которую разработчик программного
обеспечения выполняет для клиента, является итеративное извлечение и уточнение
требований к продукту. По правде говоря, клиент не знает, чего он хочет. Клиент
обычно не знает, на какие вопросы нужно ответить, и он почти никогда не
задумывался о проблеме до деталей, необходимых для спецификации. Даже простой
ответ\ --- «заставить новую программную систему работать как наша старая система
ручной обработки информации»\ --- слишком прост. Никто не хочет именно этого.
Более того, сложные программные системы\ --- это то, что действует, движется, и
изменяется в процессе работы. Динамику этого процесса трудно представить. Таким
образом, при планировании любой деятельности по разработке программного
обеспечения необходимо предусмотреть обширную итерацию между клиентом и
разработчиком как часть определения системы.

Более того, клиенту, даже работающему в паре с инженером-програм\-мистом,
действительно невозможно полностью, точно и правильно указать точные требования
к современному программному продукту, прежде чем пытаться практически
использовать несколько версий продукта. Поэтому одним из наиболее многообещающих
из текущих технологических усилий, который затрагивает суть проблемы
программного обеспечения, является \emph{разработка подходов и инструментов для
быстрого прототипирования систем}, поскольку прототипирование является частью
итеративной спецификации требования.

Прототип программной системы\ --- это система, которая имитирует важные
интерфейсы и выполняет основные функции предполагаемой системы, но при этом
необязательно ограничивается теми же аппаратными ограничениями по скорости,
размеру или стоимости. Прототипы, как правило, выполняют основные задачи
приложения, но не пытаются реализовывать полный функционал, поддерживать все
интерфейсы и форматы, или иметь документацию. Цель прототипа\ --- реализовать
заданную концептуальную структуру, чтобы клиент мог проверить ее на
согласованность и удобство использования.

Даже простые приложения никогда сначала и до конца не создаются по
свецификациям. Инкрементальное развитие\ --- программное обеспечение
выращивается, а не строится по чертежам. Концептуальные структуры, которыми мы
оперируем сегодня, слишком сложны, чтобы их можно было точно определить заранее,
и даже слишком сложны, чтобы их можно было безошибочно построить даже за один
цикл эволюции программы. Если мы обратим внимание на природу, и изучим сложность
в живых существах, мы найдем конструкции, сложность которых потрясает. Один
только мозг запутан выше возможности картографирования, мощен за пределами
имитации, богат разнообразием, самозащитой и самообновлением. Секрет в том, что
он вырос, а не построен по ТЗ.

Так должно быть и с нашими программными системами. То есть систему сначала нужно
заставить работать, даже если она не делает ничего полезного, кроме вызова
правильного набора подпрограмм-заглушек. \textit{Подход неудобен для
проектирования сверху вниз, поскольку создание множества таких заглушек отнимает
силы и запутывает код, нужно постоянно помнить детали, отличать заглушку от
рабочего кода разной степени готовности, и какие множества входных параметров
допустимы. Намного проще строить систему снизу вверх, имея работающую систему
хотя бы частично. Нельзя реализовывать при этом функции ``на будущее'', если вы
не можете их немедленно протестировать, и использовать для дальнейшего роста}.

\subsecly{Языково-ориентированное программирование}

Языково-ориентированное программирование\ --- разработка, опирающаяся на
предметно-специфичный язык (англ. DSL\ --- Domain-Specific Lan\-guage).
Это парадигма программирования, заключающаяся в разбиении процесса разработки
программного обеспечения на две стадии
\begin{itemize}[nosep]
  \item 
разработки предметно-ориентированных языков (DSL) и
  \item 
описания собственно решения задачи с их использованием.
\end{itemize}

\noindent
ЯОП и применение DSL-языков предназначено для контроля сложности разработки ПО
за счет разделения
\begin{description}
\item[машинно-зависимой части] и\\
множество тонкостей по выделению памяти, реализации
многопоточности, связи с внешними библиотеками и сервисами операционной системы,
библиотеки оптимизированных подпрограмм для численных расчетов, синтаксис
низкоуровневых языков, и т.п.
\item[проблемной части]\ \\
с которой взаимодествует (часто неподготовленный) пользователь рассматривающий
программный пакет с прикладной точки зрения: входной язык должен быть
максимально приближен к прикладной области решения задач, иметь хорошо читаемый
натуралистичный синтаксис, прикладной язык должен иметь возможность расширения
пользователем
\end{description}

\noindent
Такое разделение позволяет уменьшить экспоненциальный рост сложности, и
разделить сложности разработки, поддержки, и зону ответственности между
разработчиками платформы (низкий уровень) и прикладными программистами и
интеграторами (проблемный уровень). Типичный пример\ --- система 1С.

Использование DSL вместо языков общего назначения существенно повышает уровень
абстрактности кода, что позволяет вести разработку быстро и эффективно и
создавать программы, которые легки в понимании и сопровождении; а также делает
возможным или существенно упрощает решение многих задач, связанных с
метапрограммированием (порождение программ, проверка корректности кода,
трансформации).

ЯОП выделяется гораздо более агрессивной направленностью на приближение
компьютера к человеку. Среди последователей ЯОП бытует мнение, что в
ресурсоёмких задачах \emph{хорошо спроектированный и реализованный DSL} делает
общение неподготовленного\note{с точки зрения навыков программиста и специалиста
по автоматизации ПО} человека с компьютером \emph{куда более удобным и
продуктивным, чем графический интерфейс пользователя}.

\secup


\part{Обзор и применение \hico}\secdown

Интро еще толком не прописал, поэтому вкратце:
интересует применение экспертных систем для генерации программ для встраиваемых
систем и IoT,

погуглил на тему представления знаний в таких системах, попалась
книжка Марвина Минского (в переводе \cite{minsky}) и пара ссылок с кратким
описанием принципа, подкупает полная поддержка ООП и объектного представления + логический вывод
\begin{itemize}[nosep]
  \item 
прототип решил делать поверх Python\note{чтобы не возиться с управлением
памятью, и использовать несколько удобных библиотек},
  \item 
командный язык а-ля Форт (стек и постфикс) для простоты,
  \item 
как основной инструмент хочется унификацию (как в Прологе) но более
дружественную к императивному программированию,
\clearpage
  \item 
и самое главное гомоиконичность\\
\end{itemize}
(а) чтобы система могла достраивать сама себя (bootstrap) и\\ 
(б) \emph{полностью динамическая интерактивная система а-ля Smalltalk/Self}\\
позволяющая в себя залезть/отладить/модифицировать в рантайме

\bigskip
\noindent
основное прикладное применение: \emph{генерация кода для микроконтроллеров по
шаблонам}\ \note{параметрические куски кода на embedded Си, которые немного
изменяются в зависимости от целевой системы и контекста в котором используются}
наследование дизайна прошивки: есть код прошивки для базового прибора, и
полсотни заказчиков, каждый хочет свои лыжи и гамак, С++ под корпоративным
запретом (и в 2-8К ОЗУ не разбежишься), в итоге исходники неконтролиремо
копипастятся и имеют море наслоений legacy

\secrel{Первые шаги}\secdown

Перед вами книга, посвященная созданию очень примитивных языков программирования
на \py. Здесь вы не найдете ленивых лямбд и жутких монад, живущих в лесу
Хомского, и прочей бурбулятристики про обощенный вывод типов.
И все же мне хочется пошагово показать создание собственной реализации языка
программирования, более близкого по ощущениям к Self, Smalltalk и \lisp.
В реализациях этих языков вы можете \emph{программно} строить части программ во
время исполнения, вмешиваться в процесс работы ядра языка, и добавлять в язык
различные возможности нужные конкретно вам, например смешивать фнукциональное,
императивное и логическое программирование.

\begin{quotation}\noindent
Метапрограммирование — вид программирования, связанный с созданием
\textit{программ, которые порождают другие программы} как результат своей работы
(в частности, на стадии компиляции их исходного кода), либо программ, которые
меняют себя во время выполнения (самомодифицирующийся код).
\end{quotation}

\begin{itemize}
  \item 
\url{https://www.youtube.com/watch?v=QKFrxEkVusg}
  \item 
\url{https://www.youtube.com/watch?v=bt6kU1kuHWA}
\end{itemize}

Такие богатейшие возможности \term{метапрограммирования} возможны благодара
тому, что эти языки \term{гомоиконичны}: их реализацации работают как
\emph{живая интерактивная система} используючая структуры данных как
представление программы и исполняемый код.

\begin{quotation}\noindent
\term{Гомоиконичность} (гомоиконносль, англ. homoiconicity, homoiconic)\\
свойство некоторых языков программирования, в которых \emph{представление
программ является одновременно структурами данных} определенных в типах самого
языка, \emph{доступных для просмотра и модификации}. Говоря иначе,
гомоиконичность\ --- это когда исходный \textit{код программы} пишется
\textit{как базовая структура данных}, и язык программирования знает, как
получить к ней доступ (в том числе в рантайме при работе программ у конечного
пользователя).
\end{quotation}

\clearpage
Реализация языков такого типа \ref{implement}\ строится на \emph{интерпретаторе
структур данных}. По какой-то странной причине почему-то принято
противопоставлять \term{интерпретатор} и \term{компилятор}. Могу вас
обрадовать: \textit{все современные интерпретаторные реализации языков
программирования в обязательном порядке включают компилятор}\note{как минимум в
байт-код}.

На самом деле
\emph{\term{интерпретатор} и \term{компилятор} не противоположны}:
интерпретатор может включать в себя компилятор в машинный код как
составную часть. Или наоборот Java считается компилятором, на самом деле
программы преобразуются в \textit{интерпретируемый} \term{байт-код}, который в
свою очередь еще раз компилируется в машинный код\note{JIT\ ---
\textit{необязательная} часть языка Java, см. JavaME на телефонах}.

Еще одна бредовая привычка пользоваться сочетаниями
``интерпретируемый/компилируемый язык'': эта фраза скрывает, чем \emph{язык}
отличвается его его \term{реализации} (я неоднократно и специально использовал
это слово).

\term{Язык программирования}\ --- это \emph{формальный набор
правил}, описывающих \term{синтаксис} (как программы выглядят), \term{семантику}
(что каждая часть значит в соответствии со стандартом языка), требования к
\term{рантайму/виртуальной машине} реализации языка, и \term{стандартную
библиотеку} набор функций и процедур, поставляемых в составе \term{реализации
языка}.

Например, язык Си всегда называют ``компилируемым языком'', но никто не
запрещает написать его интерпретатор. Васик очень часто называли
интерпретатором, но даже на ZX Spectrum был компилятор Lazer Basic. Строго
говоря для некоторых языков (в том числе \hico) нельзя сделать полный
компилятор: некоторые фичи языка могут потребовать перестройки программ в
процессе выполнения. Но даже в таких клинических случаях, как реалиции языка
\lisp, возможно использование техник \term{динамической компиляции} \ref{dyna}
в реальный машинный код. В \ref{llvm}\ мы также рассмотрим встраивание
\term{кросс-компилятора} для микроконтролеров в состав нашего интерпретатора
(техника \term{управляемой компиляции}).

\clearpage
\paragraph{Применение}\ методов программирования, описанных в этой книге:\\
\bigskip

\begin{itemize}[nosep]
\item \emph{обработка текстовых форматов данных}\\
	файлы САПР, исходные данные для расчетных программ
\item командный интерфейс для устройств на микроконтроллерах\\
	управление человеко-читаемыми командами, \emph{передача пакетов данных
	любой структуры и типов}
\item реализация специализированных скриптовых языков
\item обработка исходных текстов программ\\
	модификация, трансляция на другие языки программирования,\\ 
	\emph{универсальный язык независимых от языка шаблонов и метапрограммирования}
	для ЯП с ограниченными или отсутствующими макросами
\end{itemize}

\secrel{Установка}\label{install}

\url{https://github.com/ponyatov/metaL/releases/latest}

\begin{verbatim}
~$ git clone [-b master] https://github.com/ponyatov/metaL.git
~$ cd metaL
~/metaL$ python ./metaL.py
\end{verbatim}

Интерпретатор написан на диалекте \py 2, также в системе должны быть установлены
библиотеки

\begin{verbatim}
~$ sudo pip install --upgrade pip
~$ sudo pip install ply
\end{verbatim}

Для использования веб-интерфейса

\begin{verbatim}
~$ sudo pip install flask wtforms
\end{verbatim}



\secup


\secup

\part{Реализация в деталях}\label{implement}\secdown

\secrel{Концепция фреймов Марвина Мински}\label{minsky}\secdown

\clearpage
\cite{minsky} Марвин Минский \textbf{Фреймы для представления знаний}

\begin{itemize}
%   \item 
% \url{https://royallib.com/read/minskiy_marvin/freymi_dlya_predstavleniya_znaniy.html#0}
  \item 
\url{https://ponyatov.quora.com/Minsky-Frames-Database-metaL}\\(см. видео в
начале)
\end{itemize}

\secrel{Базовый Frame}

В качестве модели представления (мета)программ было выбрано расширенное
представление фреймов Мински. Оригинальные фреймы не имели очень важного для
метапрограммирования функционала: \textit{способности хранить упорядоченные
объекты}. Эта фича необходима для представления любых
программ\note{последовательного набора инструкций, или рекурсивно вложенных
структур}, в качестве примера см. деревья разбора/AST и реализацию атрибутных
грамматик \cite{dragon2}. С другой стороны, фреймы имеют практически полное
соответстивие объектной парадигме, в т.ч. объектам \py.

Если мы попытаемся описать дерево программы через граф объектов (фреймов), мы
сталкиваемся с необходимостью иметь \emph{упорядоченные контейнеры}, например
для хранения операндов в выражении деления. Одновременно нам необходим
\emph{ассоциативный массив} для хранения и обработки \term{атрибутов}\ при
преобразованиях кода с использованием \term{атрибутных грамматик}.

\clearpage
\begin{itemize}
  \item 
выделенная иерархия классов применяется для отделения логики фреймов от логики
работы объектной системы в Python\ \note{хотя в принципе динамическая природа
\py\ позволяет реализовать все на встроенных механизмах его объектного движка},
  \item 
явные манипуляции с фреймовыми структурами демонстируют принципы реализации на
низкоуровневых языках с жесткой типизацией, AOT-компиляцией и соответственно
невозможностью произвольно менять структуру класса или единичного объекта в
рантайме (\cpp, \java)
  \item 
добавление некоторых фич, характерных для функциональных и логических языков 
программирования \note{унификация/backtracking и структурный pattern matching}
дает возможности, крайне полезные для метапрограммирования и реализации
интеллектуальных систем (базы знаний, экспертные системы, \term{семантический
ИИ}).
\end{itemize}

\clearpage
\lst{lst/frame.py}{language=Python}
 
Также в большинстве случаев у нас есть необходимость хранить для любого
элемента данных два поля:
\begin{description}

\item[type]\ тэг класса/типа явно указывающий на тип фрейма.\\
Мы принципиально не можем оперировать двумя фреймами в выражении типа
\verb|<string:> + <number:>| без их приведения к одному типу, причем это
приведение часто зависит от контекста, в каком именно смысле мы это выражение
используем (привет долбанутый JavaScript)

\item[value]\ имя фрейма или атомарное значение\note{имена type/value
фиксированы требованиями библиотеки PLY},\\хранимое в типе языка
реализации (\py): нам нужно именовать объекты, хранить значение строк и числовых
данных, поэтому также необходим подкласс фреймов для представления таких
значений-примитивов \ref{prim}.

\item[attr]\verb|{}|\ ассоциативный массив для хранения атрибутов/слотов\\
хранит любые другие фреймы, и использует строки для адресации элементов по имени
атрибута

\item[nest]\verb|[]|\ упорядоченный массив\\
одновременно может работать как список, стек или очередь, адресуется целыми
числами

\end{description}


\secrel{Дамп произвольного объекта из словаря}

Часно нужно мониторить конкретные объекты в словаре, вы выводя его полный дамп,
и не используя командную строку. Для этого можно использовать прямую ссылку:
\url{http://127.0.0.1:8888/dump/W}

\lst{web/dump.html}{language=html,title=templates/dump.html}
\lst{web/dump.css}{language=html,title=static/dark.css}
\lst{web/dump.py}{language=Python}


\secrel{Примитивные типы}\label{prim}\secdown

Примитивный (встроенный, базовый) тип\ --- тип данных, предоставляемый языком
программирования как базовая встроенная единица языка. Обычно о примитивных
типах говорят с точки зрения компилятора, предоставляющего набор типов,
реализующихся на машинном уроне.

В нашем случае фреймы обеспечивают представление верхнего уровня для таких
типов, но при этом предоставляют тот же набор интерфейсов, что и остальные типы:
\emph{в \hico\ примитивные типы} (число или строка) \emph{могут иметь
произвольные вложенные элементы и атрибуты, и поддерживают общие для всех
фреймов операции}.

Такой подход перевешивает по уровню абстракции даже подход \py\ ``все есть
объект'', и значительно замедляет прямые попытки делать на \hico\ какие-либо
вычисления. На самом деле это \textit{кривые} попытки\ --- если вы это делаете,
значин сами себе злобные Буратины, для этого предназначены другие методы
\ref{dyna}. Не нужно забывать что \emph{\hico\ это язык для
метапрограммирования} и приложений искуственного интеллекта для этого самого.

Например если вы захотите написать САПР и использовать числа как единицы
изменений размеров, вам придется создавать целую пачку специализированных
классов для представления единиц, назначения допусков, используя примитивные
типы как безразмерные величины, к которым искуственно добавляются дополнительные
атрибуты. У \hico\ несколько иной подход\ --- число может представлять любую
величину, достаточно добавить к нему атрибуты или указать единицы измерения как
вложенный элемент (по вашему выбору).

По факту, в \hico\ базовыми типами являются типы языка реализации (\py), Тем не
менее в случае (динамической) компиляции примитивные типы будут преобразованы в
машинные, если набор атрибутов не заставит процесс генерации кода использовать
дополнительные обертки.

\clearpage
\lst{lst/prim.py}{language=Python}

\secrel{Строка}\label{string}

\secrel{Cимвол}\label{symbol}

\lst{frames/symbol.py}{language=Python}

\secrel{Числа}\label{number}\secdown

\lst{lst/number.py}{language=Python}

\secup


\secup


\secrel{Контейнеры}\secdown

\lst{frames/container.py}{language=Python}

\secrel{Стек}\label{stack}

\lst{frames/stack.py}{language=Python}

\secrel{Словарь}\label{fdict}

\lst{forth/words.py}{language=Python}

\noindent
Для добавления в словарь элементов, определенных на \py\ нужно добавить пару
методов, переопределяющих операторы:
\begin{description}
\item[{dict[slot]=function}] создает словарную статью с произвольным именем
\item[dict << function] создает словарную статью используя имя функции
\item[VM(function)] обертывает функцию во фрейм \ref{vm}
\end{description}

\medskip
\lst{frames/dict.py}{language=Python}

\secup


\clearpage
\secrel{\class{Active}: исполняемые объекты}\secdown

\lst{frames/active.py}{language=Python}

\medskip\noindent
Очень важным подклассом фреймов являются \term{исполняемые объекты}. Именно они
превращают граф фреймов из пассивного описания объектов и их отношений в
\emph{выполняемую программу}, и придают фреймовой системе свойства
\term{гомоиконичного языка программирования}.

\clearpage
\secrel{\class{CMD}: команда виртуальной машины}\label{cmd}

\lst{frames/cmd.py}{language=Python}

\medskip\noindent
Команда обертывает функцию написанную на \py\ во фрейм, отображая ее в
пространство имен \metal.

\clearpage
\secrel{\class{VM}: виртуальная \F-машина}\label{vm}

\lst{frames/vm.py}{language=Python}

\medskip\noindent
Фрейм виртуальной машины описывает сущность, соответствующую абстрактному
компьютеру: он имеет память программ и память данных\note{в общем случае в одном
адресном пространстве}, и стек. \class{FVM}\ является частным случаем \F-машины
\ref{forth}

\secrel{\class{Context}: контекст}\label{context}

\lst{frames/context.py}{language=Python}
 
\medskip\noindent
Контекст\ --- текущее состояние вычисления, которое также можно рассматривать
как нить или процесс операционной системы: состояние стека, памяти, и регистров
виртуальной машины \ref{vm}.

\secup


\secup

\clearpage
\secrel{Язык \metal: исправленный \F}

Хотя мы стараемся уйти от использования языка программирования как основного
средства разработки \ref{nolang}, в любом случае нам нужен способ ввода данных и
систему, и управления вычислениями. Несмотря на десятки лет  

\secrel{Раскрутка языка (bootstrap)}\label{circ}\secdown

В этой книге нам нужно показать всю мощь языка специально заточенного под
метапрограммирования. Лучшим способом для этого является его \term{bootstrap},
или \term{раскрутка}: написать \term{метациркулярную} реализацию языка
программирования\ --- \emph{на нем самом}.

\secrel{Метациркулярный интерпретатор}

\begin{quotation}
Метациркулярный интерпретатор является интерпретатором, написанным в (возможно,
более базовой) реализации того же языка. Обычно это делается для того, чтобы
экспериментировать с добавлением новых функций на язык или созданием другого
диалекта.
\end{quotation}

В целях демонстрации того, как работает язык программирования, в литературе
часто применяют этот метод: некоторые части интерпретатора описываются на том же
языке программирования. Это позволяет не только показать внутреннее устройство,
но и служит реальным примером применения.

Если в комплект поставки включить полную метациркулярную реализацию языка,
пользователь также может адаптировать язык под свои нужды, или написать свой
клон, но для этого должно выполняться одно очень важное, критическое условие\
--- \emph{документация должна поставляться} не как руководство
пользователя, а \emph{как учебник по написанию собственной версии языка}.

\bigskip
Понять метациркулярность \textit{компилятора} очень просто: у нас есть исходный
код компилятора для некоторого языка программирования, и исполняемый файл этого
компилятора, оба версии N. Исходный код модифицируется, подается на вход
\file{компилятора-N}, в результате получем исполняемый код
\file{компилятора-N+1}. Для тестирования новой версии мы еще раз подаем исходный
код N+1 на вход \file{компилятора-N+1}, и он собирает сам себя. Такой способ в
частности применяется при сборке GCC из GNU Compilers Collection.
Собственно говоря, это единственный способ написать самодостаточный компилятор
такого системного языка как \emc: стартовая версия компилируется другим
(коммерческим) компилятором (раньше писали на ассемблере), а затем
происходит \term{раскрутка компилятора}.

\clearpage
Для интерпретаторов динамических языков используется другой способ \cite{plai},
похожий на то как мы написали всю внутреннюю механику \hico\ на \py:
реализация нового языка-N+1 представляется в виде множества явно выделенных
структур данных, которые интерпретируются исполняющим кодом на языке-N.

\bigskip
Метапрограммирование, а точнее \term{кодогенерация}, предлагает еще одну
альтернативу: ядро интерпретатора, написанное на языке \py, выполняет
метапрограмму на языке \hico, который \emph{генерирует исходный код} реализации
интерпрератора на языке Java\note{и прочее барахло в файлах проекта для
Android}, которое в итоге будет работать как интерпретатор языка \F\ на
мобильном телефоне. Это самый обкуренный пример, специально переусложненный
для демонстрации, на самом деле бы пока ограничимся только цепочкой
\py$\rightarrow$\hico$\rightarrow$\py$\rightarrow$\hico$_{N+1}$.

\secrel{Метамодель языка \hico\ с генерацией кода}

\hico\ запускается с загрузкой инициализационного файла \file{hico.ini}
\ref{ini}, который по умолчанию содержит полную модель системы и множество
других определений (для встроенного программирования, и для самораскрутки
системы). Запустив систему, вы получаете возможность делать \term{разработку
через клонирование}: при контрактной разработке передается система \hico,
дополненная функционалом, заказанным клиентом.

Такой подход особенно хорош если у вас множество заказчиков, которым вы
поставляете примерно одну и ту же систему, но с различными модификациями. Вы
можете наследовать значительные блоки исходного кода и дизайна\note{структуры
данных, реализация алгоритмов, компоненты, документацию \ref{doc},..},
прописывая для каждого клиента только те блоки, которые вы ему поставляете, а
кодогенерация сама будет отслеживать обновления и зависимости.

\clearpage
\lst{meta/header.ini}{language=Python}
Здесь вы сразу видите два варианта \term{строчных комментариев}: в стилях \py\ и
\F. Одновременно \file{hico.ini} также является последним этапом интеграционного
тестирования, проверяющим работу всех фич языка.

\secrel{Файлы проекта}\label{circfiles}

\lst{meta/files.ini}{}
\lst{meta/eclfiles.ini}{}
\lst{meta/ebldfiles.ini}{}
\lst{meta/vimfiles.ini}{}

\secup
\secrel{Web-интерфейс /Flask/}\label{web}\secdown

Система по умолчанию запускается в консольном режиме, и не требует никаких
сторонних библиотек. Это удобно при запуске \metal\ в консольном окне
\eclipse, или в командной строке. Если вы хотите удобства, например иметь
несколько окон, показывающих состояние различных объектов в словаре системы,
стоит перейти к Web-интерфейсу, обслуживаемому \term{web-фрейморк}ом Flask. Он
не так известен как Django, это минималистичен и не тянет за собой массу
библиотек, базу данных, и не требует для работы настройку виртуального
окружения.

\medskip
\lst{web/web.py}{language=Python}
\begin{description}[nosep]
\item[\file{WEB()}] весь веб-интерфейс завернут в функцию\\ если у вас не
установлены библиотеки Flask, до попытки ее запуска никаких проблем не
возникнет, консольный режим будет работать
\item[\file{IP}] сервис запускается на \file{localhost}.\\ \emph{не запускайте
\metal\ на открытых IP}, так как веб-интерфейс запускается в отладочном режиме, и
становится дыркой в безопасности.
\item[\file{PORT}] порт сервиса\\ при запуске от пользователя в \linux\ доступны
порты выше 8000
\item[\file{web}] Flask-приложение
\item[\file{SECRET\_KEY}] используется в защите веб-форм от cross-site атак\\
инициализируется примитивно\ --- случайной строкой
\item[\file{.route}] роутинг задает маршрутизацию веб-запросов\\ привязывает
путь к ресурсу в URL к прикладному коду, оформленному как функция-обработчик
запроса
\item[\file{index()}] обработчик \file{http://127.0.0.1:8888 /}\\
в простейшем случае нам нужно обрабатывать запросы только к корневому URL, и
отдать кусок html, в примере отдается дамп словаря в plain text завернутый в тег
\verb|<pre>|
\item[.run(ip,port,debug=True)] запуск веб-сервиса в отладочном режиме\\ при
появлении исключений при запуске ваших команд будет выведен дамп ошибки, и
сервис не прекратит работу аварийно
\end{description}

\smallskip
\fig{web/first.png}{height=.5\textheight}

\begin{verbatim}
 * Serving Flask app "metaL" (lazy loading)
 * Environment: production
   WARNING: Do not use the development server
            in a production environment.
   Use a production WSGI server instead.
 * Debug mode: on
 * Running on http://127.0.0.1:8888/ (Press CTRL+C to quit)
 * Restarting with stat
 * Debugger is active!
 * Debugger PIN: 299-303-273
127.0.0.1 - - [19/Mar/2019 14:58:45] "GET / HTTP/1.1" 200 -
127.0.0.1 - - [19/Mar/2019 15:01:24] "GET / HTTP/1.1" 200 -
127.0.0.1 - - [19/Mar/2019 15:01:25] "GET / HTTP/1.1" 200 -
\end{verbatim}

\secrel{Экспорт настроек в адресное пространство Форта}

Для настройки запуска веб-интерфейса через .ini файлы вынесем настройки в
адресное пространство \F-системы. Перез запуском команды \file{WEB} вы можете
поменять их.

\medskip
\lst{web/iport.py}{language=Python}

\secrel{Шаблоны и CSS}

При разработке сайтов и веб-приложений все страницы должны выводиться в
контролируемом оформлении (цвета, шрифты, расположение элементов),
и в одном стиле. За это отвечают каскадные листы стилей\ --- CSS.

Для веб-приложений, и современных сайтов с \term{активным бэкендом}\note{когда
на сервере находится выполняемый код, осуществуляющий связку веб-представления с
базами данных, и прикладным бизнеес-кодом}\ для формирования html-вывода
применяются \term{шаблоны}: образцы файлов страниц, в тексте которых указаны
места куда будет подставляться вывод генерируемый кодом на \py.

В файловой системе Flask-приложения создаются два каталога:
\begin{description}[nosep]
\item[static/] для хранения статических файлов
\verb|web.send_static_file()|
\item[templates/] для шаблонов Jinja \verb|web.render_template()|
\end{description}

\lst{web/templ.py}{language=Python}

Веб-приложение снаружи выглядит как набор множества html-страниц. Использование
шаблонов дает возможность поместить весь повторяющийся код страниц в одно место,
и подставлять изменяющуюся часть html динамически в момент отправки страницы
браузеру клиента.

\clearpage
\lst{web/templ.css}{title=static/dark.css}
\lst{web/templ.html}{title=templates/index.html}
\fig{web/templ.png}{width=\textwidth}


\secrel{Формы}\label{webform}

Для защиты CSRF\note{атаки с помощью межсайтового скриптинга}\ никогда не
пользуйтесь самодельными обработчиками ввода, для этого в любом фреймворке есть
поддержка форм.

\lst{web/form.html}{language=html,title=templates/index.html}
\lst{web/form.css}{language=html,title=static/dark.css}
\lst{web/form.py}{language=Python}


\secrel{Дамп произвольного объекта из словаря}

Часно нужно мониторить конкретные объекты в словаре, вы выводя его полный дамп,
и не используя командную строку. Для этого можно использовать прямую ссылку:
\url{http://127.0.0.1:8888/dump/W}

\lst{web/dump.html}{language=html,title=templates/dump.html}
\lst{web/dump.css}{language=html,title=static/dark.css}
\lst{web/dump.py}{language=Python}


\secrel{Графическая визуализация фреймов на D3js}\label{d3viz}

\lst{web/viz.html}{language=html,title=templates/viz.html}
\lst{web/viz.css}{language=html,title=static/dark.css}
\lst{web/viz.py}{language=Python}


\secrel{Развертывание на халявном хостинге}

\url{https://www.pythonanywhere.com/}

\bigskip\noindent
Для выпендрёжа и тестирования на мобильном телефоне развернем \metal\ на одном
из бесплатных \py-хостингов.

\begin{verbatim}
Создайте веб-приложение Python 2.7 (Flask 1.0.2)
Укажите стартовый файл сервиса: /home/metaL/metaL/metaL.py
Working directory: /home/metaL/ [Go to directory]
/home/metaL/metaL [Open Bash console here]

~/metaL$ rm -rf *
$ git init
$ git remote add gh https://github.com/ponyatov/metaL.git
$ git pull -v gh master
\end{verbatim}

\url{http://metal.pythonanywhere.com/}


\clearpage
\secrel{Google Cloud Platform}\label{gcp}\secdown

\href{https://cloud.google.com/appengine/docs/standard/python/getting-started/python-standard-env}{Getting
Started with Flask on App Engine Standard Environment}

\begin{itemize}[nosep]

  \item 
Перейдите в \href{https://console.cloud.google.com}{консоль Google Cloud}

  \item 
\url{https://cloud.google.com/sdk/docs/}

\end{itemize}

\medskip\noindent
структура проекта:
\medskip

\begin{description}[nosep]
\item[app.yaml]: конфигурация приложения Google App Engine
\item[metaL.py]: ваше приложение
\item[static/]: каталог для хранения статических файлов
\begin{description}[nosep]
\item[dark.css]: темная тема приложения
\end{description}
\item[template/] шаблоны HTML страниц
\begin{description}[nosep]
\item[index.html]: главная страница: лог и форма командной строки 
\item[dump.html]: дамп объекта из словаря
\item[viz.html]: визуализация через D3js
\end{description}
\end{description}

\clearpage
При развертывании приложения мы можете указать в этом файле настройки Python
Application Engine, в том числе сторонние библиотеки, установленные в каталог
\file{lib/}:
\lst{../appengine_config.py}{title=appengine\_config.py,language=Python}

Файл указывает какие библиотеки требуются приложению:
\lst{../requirements.txt}{title=requirements.txt,language=Python}

\clearpage\noindent
Для разработки под Google Cloud необходимо использование \file{virtualenv}

\begin{verbatim}
$ sudo pip install --upgrade pip virtualenv
~/metaL$ virtualenv --python python2 env
\end{verbatim}

\secup


\clearpage
\secrel{WebAssembly}\secdown

Интересная заморочка для кросс-разработки игр, запускающихся в браузере, на
\metal. В этом разделе мы сделаем кодогенератор, который запускается по
запросу через веб-интерфейс, и отдает клиенту образ сгенерированного \wasm-кода.
Изменяя набор игровых метамоделей внутри \metal, мы можем менять логику игры, но
\wasm\ пока не поддерживает обновление кода, поэтому рестартовать игру придется
вручную.

\bigskip
\url{https://habr.com/ru/company/jugru/blog/441140/}

\secrel{WasmFiddle}

\noindent
\url{https://wasdk.github.io/WasmFiddle/}\\
\url{https://habr.com/ru/post/342180/}
\bigskip

Попробовать технологию без установки можно online.

\bigskip
\lst{web/wasm/none.c}{language=C}
% \bigskip

Слева вверху исходный код, слева внизу текстовое представление результата
компиляции по кнопке Build, справа вверху \js\ код для запуска и справа
внизу результат запуска по кнопке Run.

WebAssembly это бинарный формат, для удобства и отладки существует механизм
текстового представления один-в-один в виде текста в формате WAT.

\lst{web/wasm/none.wat}{}

\secrel{Интерфейс WASM/\js}

\noindent
\fig{web/wasm/files.png}{height=.4\textheight}
\fig{web/wasm/api.png}{height=.7\textheight}

\medskip\noindent
WebAssembly может пользоваться любыми API, но это возможно только через \js: код
WASM может вызывать код на \js, и наоборот.

\secrel{Управление памятью}

Модель памяти WebAssembly очень проста. Это плоский «кусок» памяти, в котором
находится код программы, глобальные переменные, стек и куча. Есть возможность
сделать так, чтобы память была расширяемой, то если если при очередном выделении
памяти нам не хватает места, то верхняя граница памяти автоматически
увеличивается.
Весь блок памяти доступен из JavaScript как на чтение так и на
запись как массив байт.

\fig{web/wasm/memory.png}{width=\textwidth}


\secrel{\ems: старт на \cpp}\secdown

\noindent
\url{https://tproger.ru/translations/introduction-to-webassembly/}

\bigskip
Начинать проще с высокоуровневого тулчейна для \emc/\cpp\ --- для него проще
найти tutorialы, и собрать необходимые библиотеки. Недостатком является
длительная установка, и настройка рабочей среды. 

\secrel{Локальная установка}

\noindent
SDK \ems\ предоставляет обширный набор инструментальных средств для разработки
на \cpp\ для фронтенда. \ems\ не поставляется в виде готовых пакетов для Debian
\linux, и использует собственную систему установки со встроенной поддержкой
обновлений до новых версий SDK:
\begin{lstlisting}
$ cd ~
$ git clone --depth 1 \
	https://github.com/emscripten-core/emsdk.git
$ sudo apt install git cmake nodejs python2.7
$ cd emsdk

~/emsdk$ git pull
~/emsdk$ ./emsdk update
~/emsdk$ ./emsdk install latest
~/emsdk$ ./emsdk activate latest
~/emsdk$ source ./emsdk_env.sh
\end{lstlisting}

\secrel{Первые программы}

\noindent
\url{https://tproger.ru/translations/webassembly-tutorial-first-steps/}\\
\url{https://tproger.ru/translations/introduction-to-webassembly/}

\bigskip
\lst{web/wasm/Makefile}{}

\bigskip
\lst{web/wasm/none.c}{language=C}
На выходе получаем бинарный файл, содержащий 
\begin{lstlisting}[title=none.wasm]
0000000 060400 066563 000001 000000 007400 062006 066171
...
\end{lstlisting}

\lst{web/wasm/hello.c}{language=C}

При компиляции комплятору можно передать различные флаги:
\begin{description}

\item[-o путь\_к\_выходному\_файлу] указывает путь к файлу, который надо
сгенерировать, обычно это либо файл wasm, либо js-файл, коорый загружает
скомпилированный модуль wasm, либо html-страница, на которой загружается модул
wasm

\item[-g] генерирует отладочную информацию

\item[-s option=value] устанавливает настройки компиляции. Например, некоторые
параметры компиляции:

\item[-s WASM=1] эта опция указывает компилятору сгенерировать .wasm

\item[-s ONLY\_MY\_CODE=1] указывает компилятору не включать код из стандартной
библиотеки \emc/\cpp\ в компилируемый модуль wasm\ --- он будет включать только
неспоредственно тот код, который мы сами пишем

\item[-s EXPORTED\_FUNCTIONS='{[}...{]}'] определяет набор функций, который
должны быть экспортированы из wasm

\item[-s SIDE\_MODULE=1] эта опция указывает компилятору, что надо создать
только модуль wasm

\item[-O{[}уровень\_оптимизации{]}] указывает, какой уровень оптимизации следует
использовать при компиляции. Зачастую используется третий высший уровень, то
есть -O3

\end{description}

\secup

\secrel{WebAssembly Binary Toolkit}\label{wabt}

После того как вы немного ознакомились с приненением \wasm, можно от верхнего
уровня спустится на уровень текстового и бинарного представления кода
WebAssembly. Для работы на низком уровне предназначен

\url{https://github.com/WebAssembly/wabt}


\secup

\secup

\secrel{Элементы языка \prolog}\secdown

\url{http://yieldprolog.sourceforge.net/tutorial1.html}

\secrel{Магия алгоритма унификации}\secdown

\secrel{Генераторные функции и yield}\label{yield}

Ключевое слово \file{yield}\ в \py\ превращает любую функцию, в которой оно
используется, в функцию-\term{генератор}. Вызов генератора вместо
выполнения функции возвращает объект-\term{итератор}. Если его использовать в
качестве параметра цикла \file{for}, или явно вызывать встроенй метод
\verb|__next__()|, то вы сможете использовать \term{ленивые вычисления}\ в
обычной императивной программе на \py.

\begin{quotation}\noindent
\term{Ленивые вычисления} (англ. lazy evaluation, также отложенные вычисления)\
--- применяемая в некоторых (функциональных) языках программирования стратегия
вычисления, согласно которой вычисления следует откладывать до тех пор, пока не
понадобится их результат.
\end{quotation}

В рамках \py\ полная реализация ленивый вычислений недоступна \ref{lazy}, тем не
менее использование генераторов позволяет вычислять функции в бесконечном цикле,
возвращая промежуточные результаты. Также на генераторных функциях построен
механизм \term{логического вывода в возвратами}, используемый в языке \prolog,
который мы рассмотрим далее.


\secup
\secup
\secrel{Динамическая компиляция}\label{dyna}\secdown

\secrel{LLVM}\label{llvm}\secdown
\url{https://www.youtube.com/watch?v=q6uF3a-SJUU}
\secup


\secup

\secup

\secrel{Применение для встраиваемых систем}\label{mcu}\secdown
\secrel{Микроконтроллеры}\secdown

\secrel{Общая архитектура RISC микроконтроллеров}\secdown
\secrel{Процессорное ядро}\label{cpucore}

RISC

\secrel{Управление памятью}

Модель памяти WebAssembly очень проста. Это плоский «кусок» памяти, в котором
находится код программы, глобальные переменные, стек и куча. Есть возможность
сделать так, чтобы память была расширяемой, то если если при очередном выделении
памяти нам не хватает места, то верхняя граница памяти автоматически
увеличивается.
Весь блок памяти доступен из JavaScript как на чтение так и на
запись как массив байт.

\fig{web/wasm/memory.png}{width=\textwidth}


\secrel{WDT: сторожевой таймер}

Cторожевой таймер предназначен для защиты от аппаратных сбоев, приводиящих к
зависанию МК. Реализуется в виде таймера, срабатывание которого приводит к
аппаратному сбросу.

\secup

\secrel{Одурино}\label{arduino}

\secrel{MSP430}\label{msp}

16-битные МК средней мощности для устройств с батарейным питанием, неплохо
подходят как промежуточный этап перехода к полноценным микроконтроллерам.

\secrel{ARM Cortex-Mx /STM32/}\label{cortex}\label{stm}

\secup

\secrel{embedded Linux}\label{linux}

\secrel{Событийная архитектура вместо ОСРВ}\label{event}\secdown

\secrel{QP\texttrademark\ Real-Time Embedded Frameworks}\label{qp}

{\Huge \href{https://www.state-machine.com/}{$QuantumL^{e_a}Ps$}\ \cite{psicc2}}

\secup

\secrel{CODEin: таблетка от legacy}\label{codein}\secdown

\noindent
Огромной проблемой поддержки уже существующих проектов является практически
полное отсутствие бесплатных средств интерактивного анализа исходного кода.
Каким бы суперкрутым метапрограммистом вы не были, вы обязательно вляпаетесь в
legacy код, прибитый ржавыми гвоздями к огромной табличке: ``Ничего не трогать!
Только поправить''.

\secrel{emCin: загрузка кода на embedded \emc}\label{codein}

Даже если каким-то чудом вам удастся найти триалку коммерческого интерактивного
анализатора кода или \href{https://www.sourcetrail.com/}{Sourcetrail}, внезапно
выяснится что:
\begin{itemize}[nosep]
  \item диалект \emc\ исходников вашей прошивки не поддерживается,
  \item при попытке загрузки проекта вываливается 100500 сообщений об ошибках
  синтаксиса
  \item анализатор умеет работать только с \cpp,
  \item ничего не знает про особенности кодирования под микроконтроллеры, и
  \item неспособен загрузить исходный код ядра Linux с учетом всей пары сотен
  настроек конфигурации, и уж тем более
  \item не имеет никакого понятия о Makefile, autohell, файлах проектов IAR и
  двух десятках сборочных .batников, наклёпанных кем-то из пяти ваших
  предшественников.
\end{itemize}

\clearpage
Короче, вы попали.
\bigskip

Можете не надеяться что, прочитав эту главу, вы сможете наклепать
супер-пупер-анализатор с рефакторингом и автогенератором тестов. Чтобы написать
инстумент, способный хотя бы обеспечить приличную навигацию по исходному коду,
нужна команда разработчиков, по квалификации стремящаяся к JetBrains.

Максимум что я могу вам предложить\ --- сделайте пару пробных шагов, вдруг вы
загонитесь, и запустите проект по разработке CodeShit Studio.


\secup

\secrel{mets: генератор метаОС\\для встраиваемых систем}\label{os}\secdown

Существует огромное количество операционных систем\note{встраиваемых, реального
времени, специализированных, общего назначения, распределенных, защищенных,
исследовательских, монолитных и микроядерных,\ldots}, и систем похожих на них по
поведению \ref{qp}, в большей или меньшей степени разделяющих одни и те же
компоненты, библиотеки, и принципы дизайна.
В этом разделе предлагается альтернативных подход на базе принципов
метапрограммирования: автоматизированное построение средств runtime-поддержки
для конкретного проекта, и адаптивное конфигурирование в зависимости от
набора фич, которые вам необходимы.

\secrel{Загрузчик}\label{boot}

\secrel{Отладка под эмулятором}\label{qemudebug}\secdown

QEMU обеспечивает не только запуск \term{bare metal} программ в эмулиремой
виртуальной машине, но и полноценный отладочный интерфейс для GNU \file{gdb}.

\lst{game/debug0.mk}{title=\file{/game/Makefile},language=make}

\clearpage
PHONY-цель \file{go}\ идет в \file{Makefile} самой первой, и является главной
целью при запуске \verb|make [go]|. \file{game.elf}\ будет автоматически
перекомпилирован если вы меняли или удаляли какие-то зависимые файлы (с
исходным кодом), затем будет запущен QEMU в режиме \term{отладочного сервера}.
Режимы работы задаются с помощью опций командной строки:

\begin{description}[nosep]
\item[-kernel game.elf] QEMU умеет загружать напрямую ядра операционных систем и
bare metal программы, в которых есть загрузочный заголовок Multiboot
\ref{multiboot}. Для запуска игр скомпилированных в ELF файлы вам не нужны
образы дисков\ --- все ресурсы игры вкомпилированы внутрь исполняемого
кода, и загрузчик ОС сам позаботится чтобы они были помещены в правильные
области ОЗУ.
\item[-nographic] отключить открытие окна эмулятора видеовывода VGA
\item[-s] включить \term{gdb-сервер}
\item[-S] ожидать подключения отладчика до запуска программы
\end{description}

\clearpage
\begin{framed}\noindent
Умение пользоваться \term{отладчиком}\ --- \emph{критический навык} для
любого разработчика встраиваемых систем. 
\end{framed}
\noindent
Вы это поймете как только соскочите с Одурины \ref{arduino}\ на любую другую
среду разработки для микроконтроллеров. Вместо того чтобы долбить команды
отладочного вывода на UART, через (аппаратный) отладчик вы получаете полноценный
интерактивный доступ ко всем внутренностям микроконтроллера или любой программы,
запущенной под операционной системой.
\note{
Если какой-то великий специалист пытается вам усиленно доказывать, что для
разработки отладчик не нужен, и достаточно UART-команд устройства и отладочного
вывода, поздравляю\ --- вы наняли само\-учку-дяйвайщика. Такой ``специалист''
подобен электронщику который настраивает схему с помощью китайского тестера и
лампочки вместо многоканального осциллографа и логического анализатора. Самое
смешное, что таких инженеров массово выпускают профильные технические кафедры
российских ``ВУЗ''ов, в нагрузку 146\% вы также получаете незнание о
существовании систем контроля версий, документирования кода, багтрекинга,
тестирования, и ведения проектной документации.
}

% \clearpage
\url{http://www.linuxcenter.ru/lib/books/linuxdev/linuxdev9.phtml}

\url{https://eax.me/gdb/}

\medskip\noindent
Запуск отладчика состоит из двух частей:
\begin{description}[nosep]
\item[qemu] запускается в фоновом процессе, ожидая подключения на порту TCP
\file{127.0.0.1:1234} как \term{gdb-сервер}
\item[gdb] \emph{отладчик является (сетевым) \term{gdb-клиентом}}
опция \verb|-x game.elf.gdb| указывает скрипт с командами инициализации
отладочной сессии:
\end{description}
\lst{game/gdb0.gdb}{title=\file{/game/game.elf.gdb}}

\clearpage
\secrel{Запуск отладчика с графической оболочкой}

В \linux\ доступна интересная \term{отладочная оболочка} \file{ddd}, которая
обеспечивает не только работу любого gdb-клиента в графическом режиме, но и
\emph{отображение структур данных отлаживаемой программы в виде диаграмм}.

\lst{game/debug1.mk}{title=\file{/game/Makefile},language=make}

\clearpage
\fig{game/debug1A.png}{width=\textwidth}
\clearpage
\fig{game/debug1B.png}{width=\textwidth}
\clearpage
\fig{game/debug1C.png}{width=\textwidth}

\secrel{Отладка прошивки на микроконтроллере}

\secup

\secrel{Управление памятью}

Модель памяти WebAssembly очень проста. Это плоский «кусок» памяти, в котором
находится код программы, глобальные переменные, стек и куча. Есть возможность
сделать так, чтобы память была расширяемой, то если если при очередном выделении
памяти нам не хватает места, то верхняя граница памяти автоматически
увеличивается.
Весь блок памяти доступен из JavaScript как на чтение так и на
запись как массив байт.

\fig{web/wasm/memory.png}{width=\textwidth}


\secrel{Ввод/вывод и драйвера устройств}\secdown
\clearpage
\secrel{Последовательные порты}\secdown

\emph{Драйвер последовательного порта\ --- первый по необходимости}. Вы можете
игнорировать все остальные драйвера аппаратуры, но \emph{доступ к состоянию
системы по UART \ref{uart}} и командный пользовательский интерфейс \ref{cli} при
разработке ОС \emph{вы обязаны реализовать принципиально}. 
Эту особенность можно проследить даже по истории ЭВМ: было множество вариантов
элементов управления, но выжил, активно применяется до сих пор\note{особенно в
микроконтроллерах}, и подминает под себя даже USB\note{конвертер usb/serial вы
сейчас найдете на любой самой примитивной железке\ --- на плате, или в виде
кабеля или отдельного модуля} только один: USART.

\secup

\clearpage
\secrel{Файлы проекта}\label{circfiles}

% \lst{meta/files.ini}{}
% \lst{meta/eclfiles.ini}{}
% \lst{meta/ebldfiles.ini}{}
% \lst{meta/vimfiles.ini}{}

Начиная новый программный проект, мы каждый раз снова и снове делаем одни и те
же действия, даже используя визард в вашей любимой IDE: создать структуру
каталогов, прописать маски производных файлов для git, задать
мультиплатформенную кодировку для файлов и т.д. При этом от проекта к проекту
некоторые файлы немного меняются, особенно это относится к Makefile и другим
файлам, хранящим список компилируемых модулей или опции компиляции.
Для встраиваемых систем самым геморным будет постоянно отслеживать список файлов
и своевременно обновлять Makefile. И в итоге, для
автогенерируемого кода нам нужно автоматизировать и создание Makefile отслеживая
зависимости между множеством файлов.


\secrel{Сетевой стек}

\secrel{Стек USB}

\secup
\secrel{Планировщик задач}
\secrel{Управление пользователями}

\secrel{GUI}


\secup

\secup

\secrel{\A\ органайзер}\label{android}\secdown

\noindent
\begin{tabular}{l|p{8.3cm}}
\tfig{android/android_plan.png}{height=.45\textheight} &
Мобильниый телефон\ --- компьютер, который всегда с собой. Но \emph{удобных
средств программирования}, работающих на мобильном \textit{телефоне},
практически нет.
Разработка под \A\ и реализация on-device системы программирования\ --- тема
отдельной большой книги. Для начала можно попробовать написать \emph{органайзер,
программируемый пользоваталем}
\\ \end{tabular}

\medskip\noindent
На планшете есть несколько сред разработки для \emc, \py\ и \js, но на
телефоне они совершенно неюзабельны.

\clearpage
\begin{description}
\item[PIM] Personal Information Manager

Функции, выполняемые органайзером\note{персональным информационным
менеджером}:
\begin{itemize}[nosep]
  \item 
\emph{планирование задач} для контроля за их самостоятельным и сторонним
выполнением (ToDo list, task-трекер, мобильный CRM);
  \item 
планирование событий, привязанных к датам и времени (праздники или встречи);
  \item 
\emph{напоминальники и зудильники} об определённых пользователем событиях;
  \item 
управление контактами (адресно-телефонная книга);
  \item 
записная книжка и листки-липучки;
  \item 
личные записи (дневник);
  \item 
интеграция с электронной почтой и мессенджерами
  \item 
\emph{персональная база знаний}.
\end{itemize}

\item[PPS] Personal Planning System, система персонального планирования\\
специально заточенная на трекинг задач, с точки зрения конкретного
человека.
\end{description}

\noindent
Несмотря на десятки лет усилий, даже такие гиганты как Google и Microsoft не
смогли решить проблему создания полноценного органайзера, в который by design
должен был превратиться смартфон:
\begin{itemize}
  \item доступны только примитивные типы задач, при этом на практике нужно
  множество вариантов, от простого будильника, встречи, действий привязанных по
  месту, периодические задачи с разным масштабом\note{от 30 минут для отдыха
  глаз, до года для дней рожденья}, до задач характерных для систем
  groupware
  \item полностью отсутствует функционал трекинга проектов: групповые
  задачи, зависимости задач по времени, исполнителям и ресурсам, делегирование и
  контроль,\ldots
  \item отсутствие средств индивидуальной адаптации, включая средства
  программирования пользователем, и доступ к внешним приложениям, библиотекам и
  сенсорам
\end{itemize}

\noindent
Для таких внутренне сложных приложений как органайзер, планировщик или трекер
задач, подход традиционных приложений с пользовательским GUI не подходит. Чем
больше автор приложения усложняет его, добавляя все новые и новые функции, тем
сложнее для пользователя становится его освоение. При этом даже для
очень переусложненного органайзера обязательно встретится случай использования,
под который не подходит ни один из предусмотренных в приложении вариантов.
Например потребуется задача, которая должна срабатывать одновременно по времени,
местоположению, и условию выполнения другой задачи, назначеная другому
пользователю в вашей рабочей группе.

Если наборот в дизайне органайзера пойти от архитектуры, построенной на
минимальном ядре, расширяемом пользователем через написание скриптов, мы снижаем
начальный порог вхождения (за счет уменьшения необходимых усилий для освоения),
одновременно предоставляя пользователю механизм персональной адаптации.

\medskip\noindent
\begin{tabular}{l p{8.7cm}}
\tfig{android/basact.png}{height=.6\textheight} &
Для создания проекта в Android Studio лучше всего подходит Basic Activity, так
как из коробки в графическом интерфейсе приложения есть элемент добавления новой
задачи.

\medskip
\begin{tabular}{l l}
Name & HICO \\
Package name & \file{io.github.ponyatov.hico} \\
Save location & \file{/home/ponyatov/hico/Android} \\
Language & Java \\
Minimum API Level & API14\\
\end{tabular}
\\
\end{tabular}

\bigskip\noindent
\begin{tabular}{l p{9.5cm}}
&\\
\tfig{android/planning.png}{height=.2\textheight} &
Для начала стоит заменить иконку приложения, также создав комплект round icon
для совместимости со новыми версиями \A\ использующими ``пузырчатый'' интерфейс.
\\
\end{tabular}

\bigskip\noindent
Для подготовки иконок нужно иметь некоторые навыки для работы с графикой и
каким-то из графических редакторов, типа \href{https://www.gimp.org}{GIMP}. Если
у вас уже есть черновик иконки, можете воспользоваться визардом по адресу\\
\url{http://jgilfelt.github.io/AndroidAssetStudio/icons-launcher.html}\\
Загрузив в него черновик иконки, и включив эффект Shape:Bewel, вы сможете
скачать \file{ic\_launcher.zip} содержащий готовый набор иконок для разных
разрешений экрана. Копировать их в проект придется по одному файлу вручную, так
как в новых версиях Android Studio изменилась схема именования каталогов с
ресурсами: с \file{drawable\_}\ на \file{mipmap\_}.

\bigskip
\begin{lstlisting}
~/hico$ find Android/ -type f \
	-regex .+ic_launcher.png$ -exec file {} +
	
Android/app/src/main/res/mipmap-xhdpi/ic_launcher.png:   
	PNG image data, 96 x 96, 8-bit/color RGBA, non-interl
Android/app/src/main/res/mipmap-xxhdpi/ic_launcher.png:  
	PNG image data, 144 x 144, 8-bit/color RGBA, non-inte
Android/app/src/main/res/mipmap-mdpi/ic_launcher.png:    
	PNG image data, 48 x 48, 8-bit/color RGBA, non-interl
Android/app/src/main/res/mipmap-xxxhdpi/ic_launcher.png: 
	PNG image data, 192 x 192, 8-bit/color RGBA, non-inte
Android/app/src/main/res/mipmap-hdpi/ic_launcher.png:    
	PNG image data, 72 x 72, 8-bit/color RGBA, non-interl
\end{lstlisting}

\clearpage

\fig{android/firstrun.png}{height=.85\textheight}
\fig{android/fixcolors.png}{height=.85\textheight}
\fig{android/addbutton.png}{height=.85\textheight}

\noindent
Цвета по умолчанию получаются слишком кислотные, стоит их приглушить поправив
файлы, содержащие цвета и стили для проекта:

\begin{lstlisting}[title=Android/app/src/main/res/values/colors.xml]
<?xml version="1.0" encoding="utf-8"?>
<resources>
    <color name="colorPrim">#008577</color>
    <color name="colorPrimDark">#00473B</color>
    <color name="colorAccent">#880022</color>
</resources>
\end{lstlisting}

\begin{lstlisting}[title=Android/app/src/main/res/values/styles.xml]
<resources>
<style name="AppTheme" parent="Theme.AppCompat">
<item name="colorPrimary">@color/colorPrimDark</item>
<item name="colorPrimaryDark">@color/colorPrimDark</item>
\end{lstlisting}

\secup


\secrel{Документирование и CBL}\label{docu}\secdown

\noindent
CBL\ --- Computer Based Learning, \term{автоматизированное обучение}

\begin{itemize}
\item документирование аппартно-программных разработок
\item передача знаний в рабочих группах
\item решение проблем описанных в \ref{codein}
\item пользовательская документация, справка
\end{itemize}
Документирование предполагает пассивный режим: есть мануалы и справка, но
пользователю они доступны в режиме чтения. При добавлении подсистемы CBL
появляется трекинг обучения конкретного пользователя
\begin{itemize}
  \item активное адаптивное обучение пользователей
  \item передача знаний в группах разработчиков и системных интеграторов
\end{itemize}

\secrel{Компоненты обучающий системы}
\secrel{Метод Монте-Карло}\label{lrnkarlo}

\clearpage
Случайный показ произвольного объекта документации, включая узлы, которые на
него ссылаются, и узлы, на которые ведут исходящие ссылки. В результате
формируется \term{начальная точка просмотра} включающая взаимосвязи.
По каждому элементу пользователю предъявляется статистика покрытия, для
выявляения элементов документации, по которым пользователь прошел недостаточное
обучение.

\begin{itemize}
  \item 
Частота предъявления элементов выбирается в соответствии со статистикой
обучения, приоритет отдается областям \emph{смежным с наиболее посещаемыми} (они
интересуют пользователя так как статистика накручивается в процессе работы в
поисках ответов на проблемы).
  \item 
Одновременно в выборку \emph{добавляются области с
минимальным посещением, чтобы обеспечить равномерное покрытие} документации
просмотрами.
\end{itemize}


\secup



\addcontentsline{toc}{section}{Литература}
\begin{thebibliography}{99}

\clearpage
\bibitem{py}\ \bibfig{bib/python.jpg}\\
\textbf{Язык программирования Python}\\
Россум, Г., Дрейк, Ф.Л.Дж., Откидач, Д.С.,\ldots\\
\url{http://rus-linux.net/MyLDP/BOOKS/python.pdf}

\clearpage
\bibitem{dragon} \bibfig{bib/dragon2.png}\ \emph{Purple Dragon Book /2nd ed/}\\
\textbf{Компиляторы: принципы, технологии и инструментарий} 2 изд.\\
Альфред В. Ахо, Моника С. Лам, Рави Сети, Джеффри Д. Ульман.\\
М.: Вильямс, 2008.\\ ISBN 978-5-8459-1349-4.\\
\url{https://www.ozon.ru/context/detail/id/148568229/}

\clearpage
\bibitem{sicp} \bibfig{bib/sicp.jpg}\ \textbf{\emph{SICP}\\
\href{https://drive.google.com/file/d/0B0u4WeMjO894X3lnWmhjUktKRk0/view?usp=sharing}{Структура
и интерпретация компьютерных программ}}\\
Харольд Абельсон, Джеральд Сассман\\
ISBN 5-98227-191-8\\
EN: \url{web.mit.edu/alexmv/6.037/sicp.pdf}\\
\url{https://www.ozon.ru/context/detail/id/5322055/}

\clearpage
\bibitem{bratko}\ \bibfig{bib/bratko.jpg}\\
\textbf{Программирование на языке Пролог\\для искусственного интеллека}\\
Иван Братко\\
Мир, 1990\\ ISBN 5-03-001425-Х, 0-201-14224-4

\end{thebibliography}


\end{document}
