\secrel{Универсальная IDE на wxPython}\label{wx}\secdown

\noindent
Использование приложений с нативным GUI стало немодно: веб-интер\-фейс
обеспечивает б\'{о}льшую гибкость, и возможность
интеграции с внешними сетевыми сервисами (GitHub, Google). Если рассматривать
требования браузера по памяти по отрисовке GUI, и быстродействия интерфейса,
вопрос о применении \term{нативного GUI} остается открытым.

\secrel{Установка под \win}

\begin{verbatim}
$ python-2.7.16.msi
$ pip install --upgrade pip
$ pip install --upgrade ply
$ wxPython3.0-win32-3.0.2.0-py27.exe
\end{verbatim}

\noindent
Поддержка версии 2.7 закрыта, поэтому часть пакетов придется ставить не через
\file{pip}, а индивидуальными инсталляторами.

\secrel{Базовый мультиоконный GUI}

\clearpage
\lst{gui/wx00.py}{language=Python,title=gui/wxide.py}

\begin{description}[nosep]
\item[ide] приложение wxWidgets, объект отслеживает логику GUI
\item[ideWindow] класс окна, формирует типовое окно IDE (редактор + меню)
\item[ideConsole] главное окно\ --- командная консоль
\end{description}

\fig{gui/wx00.png}{height=.5\textheight}

\lst{gui/wx01.py}{language=Python}
\fig{gui/wx01.png}{height=.3\textheight}

\clearpage
\lst{gui/wx02.py}{language=Python}
\fig{gui/wx02.png}{height=.5\textheight}

\clearpage
При запуске будем открывать файл командного скрипта по умолчанию, и хранить
привязку окна IDE к текстовому файлу:
\lst{gui/wx03.py}{language=Python}
\lst{gui/wx04.py}{language=Python}
\lst{gui/wx05.py}{language=Python}
Пока ограничимся привязкой каждого окна редактора к текстовому файлу: этот
функционал в любом случае необходим для чтения уже существующего кода, и
обеспечения совместимости с системами версионирования.
\clearpage
\lst{gui/wx06.py}{language=Python}
\fig{gui/wx06.png}{height=.5\textheight}

\clearpage
Для интерактивной работы в IDE необходим импорт \F-машины из \metal:

\secup