\secrel{Примитивные типы}\label{prim}\secdown

Примитивный (встроенный, базовый) тип\ --- тип данных, предоставляемый языком
программирования как базовая встроенная единица языка. Обычно о примитивных
типах говорят с точки зрения компилятора, предоставляющего набор типов,
реализующихся на машинном уроне.

В нашем случае фреймы обеспечивают представление верхнего уровня для таких
типов, но при этом предоставляют тот же набор интерфейсов, что и остальные типы:
\emph{в \hico\ примитивные типы} (число или строка) \emph{могут иметь
произвольные вложенные элементы и атрибуты, и поддерживают общие для всех
фреймов операции}.

Такой подход перевешивает по уровню абстракции даже подход \py\ ``все есть
объект'', и значительно замедляет прямые попытки делать на \hico\ какие-либо
вычисления. На самом деле это \textit{кривые} попытки\ --- если вы это делаете,
значин сами себе злобные Буратины, для этого предназначены другие методы
\ref{dyna}. Не нужно забывать что \emph{\hico\ это язык для
метапрограммирования} и приложений искуственного интеллекта для этого самого.

Например если вы захотите написать САПР и использовать числа как единицы
изменений размеров, вам придется создавать целую пачку специализированных
классов для представления единиц, назначения допусков, используя примитивные
типы как безразмерные величины, к которым искуственно добавляются дополнительные
атрибуты. У \hico\ несколько иной подход\ --- число может представлять любую
величину, достаточно добавить к нему атрибуты или указать единицы измерения как
вложенный элемент (по вашему выбору).

По факту, в \hico\ базовыми типами являются типы языка реализации (\py), Тем не
менее в случае (динамической) компиляции примитивные типы будут преобразованы в
машинные, если набор атрибутов не заставит процесс генерации кода использовать
дополнительные обертки.

\clearpage
\lst{frames/prim.py}{language=Python}

\secrel{Строка}\label{string}

\secrel{Cимвол}\label{symbol}

\lst{frames/symbol.py}{language=Python}

\secrel{Числа}\label{number}\secdown

\lst{lst/number.py}{language=Python}

\secup


\secup
