\clearpage
\secrel{\class{Active}: исполняемые объекты}\secdown

\lst{frames/active.py}{language=Python}

\medskip\noindent
Очень важным подклассом фреймов являются \term{исполняемые объекты}. Именно они
превращают граф фреймов из пассивного описания объектов и их отношений в
\emph{выполняемую программу}, и придают фреймовой системе свойства
\term{гомоиконичного языка программирования}.

\clearpage
\secrel{\class{CMD}: команда виртуальной машины}\label{cmd}

\lst{frames/cmd.py}{language=Python}

\medskip\noindent
Команда обертывает функцию написанную на \py\ во фрейм, отображая ее в
пространство имен \metal.

\clearpage
\secrel{\class{VM}: виртуальная \F-машина}\label{vm}

\lst{frames/vm.py}{language=Python}

\medskip\noindent
Фрейм виртуальной машины описывает сущность, соответствующую абстрактному
компьютеру: он имеет память программ и память данных\note{в общем случае в одном
адресном пространстве}, и стек. \class{FVM}\ является частным случаем \F-машины
\ref{forth}

\secrel{\class{Context}: контекст}\label{context}

\lst{frames/context.py}{language=Python}
 
\medskip\noindent
Контекст\ --- текущее состояние вычисления, которое также можно рассматривать
как нить или процесс операционной системы: состояние стека, памяти, и регистров
виртуальной машины \ref{vm}.

\secup
