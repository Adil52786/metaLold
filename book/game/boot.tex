\clearpage
\secrel{Загрузчик Multiboot}\label{multiboot}

Спецификация \term{Multboot} описывает интерфейс между ядром операционной
системы и загрузчиком, следуя которому любой загрузчик может применяться
универсально для загрузки любых операционных систем, если ядра этих ОС
поддерживают спецификацию Multiboot.

Использование multiboot позволит сразу исключить лишний этап при написании своей
ОС для i386: создание загрузчика со всеми хождениями по граблям с чтением с
устройств, переключением в защищенный режим, инициализацией оборудования и т.п.
\file{QEMU} поддерживает спецификацию multiboot и может загружать ядро ОС
напрямую из файла через параметр \verb|-kernel| без образов дисков и сборки
файловых систем для эмулятора. На реальном железе вы можете использовать любой
другой загрузчик, типа \file{syslinux} \ref{syslinux} или \file{GRUB}.

\clearpage
Чтобы скомпилировать ядро ОС, в бинарном файле в самом начале должна находиться
структура:
\bigskip

\noindent
\url{https://habr.com/ru/post/351568/}

\noindent
\url{https://web.archive.org/web/20070210091856/http://www.osdcom.info/content/view/33/39/}

\bigskip

\noindent
\begin{tabular}{l l l}
магическое число & u32 & 0xE85250D6 \\
архитектура	& u32 & 0 для i386, 4 для MIPS \\
длина заголовка & u32 & общий размер заголовка включая тэги \\
контрольная сумма & u32	& -(магическое число + архитектура + длина заголовка) \\
тэги & \ldots & \\
завершающий тэг & & (u16, u16, u32) = (0, 0, 8) \\
\end{tabular}

\clearpage
\lst{game/boot00.mk}{title=\file{game/Makefile}}
\lst{game/boot00.objdump}{}
