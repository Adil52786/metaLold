\secrel{Разработка игр}\label{gamedev}\secdown\secdown

В качестве основы для этого раздела взята библия разработчиков игр\ -- книга
Андре Ла Мота \cite{lamot}. Это первая и единственная книга на русском языке,
выпущенная издательством ``Питер'' в 1995 году, кардинально и полностью
\emph{раскрывающая всю механику реализации игр до аппаратного уровня}\ --- не
скрывая ничего за слоями абстракции ОС, DirectX, фрейморков и библиотек. Именно
этим эта книга ценна до сих пор, несмотря на то что приведенный в книге исходный
код и инструменты устарели еще 20 лет назад.

А вот с выбором целевой платформы возникают некоторые сомнения. Ориентироваться
на DOS и реальный режим процессоров 8086 смысла уже нет: запуск такого ПО
потребует обязательного использования эмулятора DOSBOX, и при этом есть
определенные проблемы с поиском легальных средства разработки для реального
режима 8086. В этом разделе мы рассмотрим \file{QEMU-i386} и низкоуровневый код
как наиболее универсальный вариант, который вы при желании можете запустить на
реальном железе.

\secrel{Сборка кросс-компилятора для QEMU-i386}\label{cross386}

Для компиляции игр для платформы i386 мы можем использовать штатный \term{GNU
toolchain} для \file{x86\_64}, но воспользуемся скриптом сборки
\term{кросс-компилятора} из раздела \ref{mcu}:

\lst{game/cross0.mk}{title=\file{/game/Makefile},language=make}
\secrel{Загрузчик}\label{boot}

\secrel{Видеосистема}\secdown

\secrel{Инициализация VESA framebuffer}

\secup


\secup\secup
