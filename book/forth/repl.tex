\secrel{Интерпретатор}

\lst{forth/repl.py}{language=Python}

\noindent
REPL реализуется в виде бесконечного цикла:
\begin{itemize}[nosep]
  \item вывод стека перез запросом команды
  \item чтение команды, \emph{при нажатии Ctrl-D срабатывает исключение EOF}
  \item запуск \verb|INTERPRET|
\end{itemize}
и запускается непосредственно из модуля исходного кода.

\bigskip
\lst{forth/inter.py}{language=Python}

\noindent
\verb|INTERPRET| получает строку из стека, загружая ее в лексер
\verb|lexer.input()|

\noindent
После инициализации лексера запускается бесконечный цикл
\begin{description}[nosep]
\item[WORD] \verb|( -- token )| в \pyf\ \emph{отличается от классического \F}\\
получает следующий токен от лексера,\\и не принимает на вход
символ-разделитель, как это сделано в \F\\
\verb|return| возвращает \verb|False| если лексер закончил разбор\\
\lst{forth/word.py}{language=Python}
\item[FIND] \verb$( token -- object|token )$\\
поиск в словаре выполняется по имени любого объекта, возвращаемого методом
\verb|str()|\\
если объект в словаре не найден, на стеке возвращается оригинальный токен\\
\verb|return| возвращает флаг успешности поиска, который используется в \py\
коде\\
\lst{forth/find.py}{language=Python}
\item[EXECUTE] \verb|( object -- ... )|\\
объект выполняется, примитивные типы вычисляются сами в себя, базовый
\verb|Frame| вызывает выполенение вложенных элементов (рекурсивно)
\end{description}

\bigskip
\lst{forth/execute.py}{language=Python}
