\secrel{\F\ как примитивный язык CLI}\label{forth}\secdown

\begin{quotation}\noindent
\term{CLI}\ --- Command Line Interface, \term{интерфейс командной строки}\ ---
разновидность текстового интерфейса между человеком и компьютером, в котором
инструкции компьютеру даются путём ввода с клавиатуры текстовых строк (команд).
Также известен под названием \term{консоль}.
\end{quotation}

\noindent
CLI всегда активно использовался профессиональными программистами и операторами,
и все еще остается основным интерфейсом на встраиваемых системах, и на рабочих
станциях UNIX/Linux.

\begin{quotation}\noindent
\term{REPL} (от англ. Eead-Eval-Print Loop\ --- цикл чтение\ -- вычисление\ --
вывод)\ --- форма организации простой интерактивной среды программирования в
рамках средств интерфейса командной строки. Чаще всего этой аббревиатурой
характеризуется интерактивная среда языка программирования \lisp, однако такая
форма характерна и для минимальных интерактивнйых сред многих других языков,
реализованных в виде интерпретатора.
\end{quotation}

\clearpage
\noindent
В такой среде пользователь может вводить выражения, которые среда тут же будет
вычислять, а результат вычисления отображать пользователю. Названия элементов
цикла связаны с примитивами \lisp а:

\begin{description}
\item[read] читает одно выражение и преобразует его в соответствующую структуру
данных в памяти (строку или \term{синтаксическое дерево});
\item[eval] принимает одну такую структуру данных и вычисляет соответствующее ей
выражение; \textit{большинство интерпретаторов на этом этапе запускают
внутренний компилятор}, который гененерирует \term{байт-код} \ref{bc}
\term{виртуальной машины} \ref{vm}, или реальный машинный код
\item[print] принимает результат вычисления выражения и печатает его
пользователю.
\end{description}

\noindent
Чтобы реализовать REPL-среду для некоторого языка, достаточно реализовать три
функции: чтения, вычисления и вывода, и объединить их в бесконечный цикл.
REPL-среда очень удобна при изучении нового языка и экспериментов, так как
предоставляет пользователю быструю обратную связь.

% \clearpage
\begin{quotation}\noindent
\emph{REPL/CLI необходим для интерактивного командного управления} различным
оборудованием через последовательный порт, USB/Serial или в сетевой сессии
telnet/ssh. В большинстве случаев такая \emph{языковая оболочка крайне полезна и
для реализации протоколов обмена данными в режиме машина/машина (m2m)}\ ---
языковые вредства интерпретатора позволяют \emph{неограниченно расширять
протокол обмена и логику команд}, который при этом остается человеко-читаемым,
что очень удобно для отладки и логирования.
\end{quotation}

\clearpage
Язык \F\ \cite{starting} является самым примитивным языком программирования, про
который можно сказать, что он \textit{не имеет синтаксиса}: его \term{парсер}
крайне примитивен, и реализуется всего в несколько машинных команд на любой
платформе.

\F\ изначально был создан
\href{https://ru.wikipedia.org/wiki/%D0%9C%D1%83%D1%80,\_%D0%A7%D0%B0%D1%80%D0%BB%D1%8C%D0%B7\_(%D0%BF%D1%80%D0%BE%D0%B3%D1%80%D0%B0%D0%BC%D0%BC%D0%B8%D1%81%D1%82)}{Чальзом
Муром}\ как раз для задач управления оборудованием Национальной
радиоастрономической обсерватории (NRAO). И он \textit{так и остался непобедим в
этой роли} до настоящего времени, \textit{для применения в качестве CLI} на
самых маломощных компьютерах на базе \term{микроконтролер}ов.

У этого языка в оригинальном варианте крайне неприятный постфиксный синтаксис и
никоуровневая модель виртуальной машины, делающие бессмысленными попытки
его использования в качестве универсального языка программирования. С другой
стороны, этот же \textit{синтаксис оказывается удобным для однострочных
команд}: не требуется елозить по тексту и править множество вложенных скобок.

\clearpage
Как вы увидите дальше, \emph{главное достоинство \F а\ --- простота реализации}.
Второе достоинство, важное для встраиваемых систем: \emph{классическая
\F-система для своей работы требует \textbf{всего несколько Кб (!)} ОЗУ}, но мы
пока не будем углубляться на такой низкий уровень программирования.

\bigskip
Традиционная \F-система состоит из двух частей: исполняющая среда,
\term{виртуальная \F-машина}, или FVM\note{по аналогии с JVM}, и
\term{транслятор}. Транслятор занимается обработкой исходного текста команд и
программ, FVM выполняет готовый программный код.

Эти две части могут работать и независимо. Для встраиваемых систем и
коммерческих разработок можно использовать только FVM для выполнения
скомпилированной ранее программы. Если вы не хотите использовать \F\ для чтения
конфигурационных файлов или реализации командного протокола, то транслятор не
нужен.

Но нам эти детали не важны, так как мы планируем не затаскивать реализацию FVM
на целевую систему (микроконтроллер), а использовать более мощные методики,
позволяющие стыковать магические технологии с mainstream средствами и
компиляторами, применяемыми для коммерческой разработки.

Также нам не нужна полнофункциональная \F-система и на рабочей станции, мы
возьмем от нее только общий принцип минимального интерпретатора, и значительно
отойдем от классического Форта. Чтобы не путаться, для этого огрызка стоит взять
другое имя, я назвал его \pyf.

\secrel{Принцип работы минимального интерпретатора}

Для минимального интерпретатора нужны три вещи:
\begin{description}
\item[REPL] ввод строки от пользователя, и ее примитивный парсинг \ref{ply}\\
\F\ считает любую последовательность символов, разделенную пробелами,
единичным \term{словом}-командой. Поэтому разбор оказывается очень простым:
достаточно посимвольно просканировать \term{входной поток} до любого пробельного
символа (пробел, табуляция, конец строки).
\item[стек данных] для хранения промежуточных данных \ref{fstack}\\
Отсутствие лишних временных переменных удобно: не нужно придумывать названий
переменных для временных данных, или скроллить чтобы посмотреть названия
автоматически назначенных (как в REPL математических пакетов). Кроме того,
синтаксис без нескольких вложенных скобок и с короткими командами
\textit{оказывается крайне удобным на мобильных устройствах}: линейную
последовательность команд проще редактировать.
\item[словарь]\ \ref{fdict}\\
каждое выделенное парсером слово, не являющееся константой-\term{лите\-ралом}
(число, строка), ищется в словаре\note{цепочке словарей}, \emph{найденный объект
выполняется}
\end{description}

\secrel{Стек данных}\label{fstack}

Итак, что у нас остается? Прежде всего это \emph{стек} \ref{stack}: он
\textit{удобен для манипуляций, запускаемых короткими командами}, и
\emph{освобождает нас от указаний где сохранять промежуточные данные}.

\medskip
\lst{forth/stack.py}{language=Python}

\secrel{Словарь}\label{fdict}

\lst{forth/words.py}{language=Python}

\noindent
Для добавления в словарь элементов, определенных на \py\ нужно добавить пару
методов, переопределяющих операторы:
\begin{description}
\item[{dict[slot]=function}] создает словарную статью с произвольным именем
\item[dict << function] создает словарную статью используя имя функции
\item[VM(function)] обертывает функцию во фрейм \ref{vm}
\end{description}

\medskip
\lst{frames/dict.py}{language=Python}


\secrel{PLY: библиотека синтаксического разбора для \py}\label{ply}\secdown

\clearpage
Библиотека PLY позволяет писать \term{парсеры} для достаточно сложных языков.
Если вам для какой-то задачи потребуется применение \term{инфиксного
синтаксиса}, типа разбора арифметических выражений, вы сможете без особых усилий
добавить для них \term{синтаксический анализатор}.

\begin{description}
\item[Лексер] обрабатывает \term{входной поток} единичных символов из файла или
строки, группируя их в \term{токены}. Каждый токен имеет тип \verb|.type|,
значение \verb|.value|, и дополнительные поля типа имени файла исходного кода,
или позиции токена (строка, столбец).
\item[Парсер синтаксиса]\ читает \term{поток токенов}, или работает напрямую с
символами и текстовыми строками, в зависимости от того, какие алгоритмы
разоработа используются.
\end{description}

\clearpage
\lst{syntax/ply/lex.py}{language=Python}
\lst{syntax/ply/yacc.py}{language=Python}

\term{Парсер} может состоять из обоих компонентов, или только из одного, в
зависимости от сложности синтаксиса входного языка, и того, нужно ли нам
\term{распознавать} рекурсивно вложенные синтаксические конструкции, или
достаточно только определить тип \term{лексем} (для подстветки синтаксиса).

\clearpage
Каждое правило лексера задается в виде функции, её docstring задает регулярное
выражение, которому должна удовлетворять группа символов, чтобы быть
распознанной. Функция получает на вход параметр \verb|t| содержащий
\textit{состояние лексера}; \verb|t.value| содержит распознанную группу символов
в виде строки, которую мы возвращаем из функции через вызов конструктора фрейма
соответствующего типа.

\bigskip
\noindent
\verb|ply.lex.lex()| проходит по исходному коду текущего модуля \py, находит
функции соответствующие шаблону правил лексера, и синтезирует функцию-лексер.



\secrel{\F\ лексер}\label{plyforth}

Несмотря на то что диалекты \F/\pyf\ требуют только реализацию \term{лексера},
есть смысл немного сэкономить усилия, и не заморачиваться с написанием
традиционного посимвольного разбора ``до пробела''. Еще одно достоинство
использования PLY: с ее помощью мы можем автоматически определять тип для каждой
лексемы, в частности разпознавать примитивные типы \ref{prim}\ и вызывать
соответствующие конструкторы.

\begin{description}%[nosep]
\item[tokens] список токенов, которые может распознавать парсер\\
так как мы специально обеспечили возможность использования фрей\-мов-примитивов
в качестве \term{литералов}, в этом списке должны быть перечислены тэги (c
маленькой буквы)
\item[t\_ignore] символы которые не будут участвовать в разборе (пробелы) 
\item[t\_number()] правило лексера распознающее числа
\item[t\_symbol()] правило распознающее символы как имена форт-слов
\item[t\_error()] обработка синтаксических ошибок: нераспознанные символы
\end{description}

\secup

\secrel{Интерпретатор}

\lst{forth/word.py}{language=Python}
\lst{forth/find.py}{language=Python}
\lst{forth/execute.py}{language=Python}
\lst{forth/repl.py}{language=Python}

\secup
