\secrel{\F\ как примитивный язык CLI}\label{forth}\secdown

\begin{quotation}\noindent
\term{CLI}\ --- Command Line Interface, \term{интерфейс командной строки}\ ---
разновидность текстового интерфейса между человеком и компьютером, в котором
инструкции компьютеру даются путём ввода с клавиатуры текстовых строк (команд).
Также известен под названием \term{консоль}.
\end{quotation}

\noindent
CLI всегда активно использовался профессиональными программистами и операторами,
и все еще остается основным интерфейсом на встраиваемых системах, и на рабочих
станциях UNIX/Linux.

\begin{quotation}\noindent
\term{REPL} (от англ. Eead-Eval-Print Loop\ --- цикл чтение\ -- вычисление\ --
вывод)\ --- форма организации простой интерактивной среды программирования в
рамках средств интерфейса командной строки. Чаще всего этой аббревиатурой
характеризуется интерактивная среда языка программирования \lisp, однако такая
форма характерна и для минимальных интерактивнйых сред многих других языков,
реализованных в виде интерпретатора.
\end{quotation}

\clearpage
\noindent
В такой среде пользователь может вводить выражения, которые среда тут же будет
вычислять, а результат вычисления отображать пользователю. Названия элементов
цикла связаны с примитивами \lisp а:

\begin{description}
\item[read] читает одно выражение и преобразует его в соответствующую структуру
данных в памяти (строку или \term{синтаксическое дерево});
\item[eval] принимает одну такую структуру данных и вычисляет соответствующее ей
выражение; \textit{большинство интерпретаторов на этом этапе запускают
внутренний компилятор}, который гененерирует \term{байт-код} \ref{bc}
\term{виртуальной машины} \ref{vm}, или реальный машинный код
\item[print] принимает результат вычисления выражения и печатает его
пользователю.
\end{description}

\noindent
Чтобы реализовать REPL-среду для некоторого языка, достаточно реализовать три
функции: чтения, вычисления и вывода, и объединить их в бесконечный цикл.
REPL-среда очень удобна при изучении нового языка и экспериментов, так как
предоставляет пользователю быструю обратную связь.

% \clearpage
\begin{quotation}\noindent
\emph{REPL/CLI необходим для интерактивного командного управления} различным
оборудованием через последовательный порт, USB/Serial или в сетевой сессии
telnet/ssh. В большинстве случаев такая \emph{языковая оболочка крайне полезна и
для реализации протоколов обмена данными в режиме машина/машина (m2m)}\ ---
языковые вредства интерпретатора позволяют \emph{неограниченно расширять
протокол обмена и логику команд}, который при этом остается человеко-читаемым,
что очень удобно для отладки и логирования.
\end{quotation}

\clearpage
\noindent
Язык \F\ \cite{starting} является самым примитивным языком программирования, про
который можно сказать, что он \textit{не имеет синтаксиса}: его \term{парсер}
крайне примитивен, и реализуется всего в несколько машинных команд на любой
платформе.

\F\ изначально был создан
\href{https://ru.wikipedia.org/wiki/%D0%9C%D1%83%D1%80,\_%D0%A7%D0%B0%D1%80%D0%BB%D1%8C%D0%B7\_(%D0%BF%D1%80%D0%BE%D0%B3%D1%80%D0%B0%D0%BC%D0%BC%D0%B8%D1%81%D1%82)}{Чальзом
Муром}\ как раз для задач управления оборудованием Национальной
радиоастрономической обсерватории (NRAO). И он \textit{так и остался непобедим в
этой роли} до настоящего времени, \textit{для применения в качестве CLI} на
самых маломощных компьютерах на базе \term{микроконтролер}ов.

\secup
