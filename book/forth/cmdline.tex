\secrel{Обработка командной строки\\чтение команд из файлов}\label{cmdline}

\lst{forth/cmdline.py}{language=Python}

Нам неудобно тащить всю мощь \hico\ в виде единственного файла \file{hico.py}.
Ядро языка поставляется в исходных кодах, и оно должно
оставаться компактным, читаемым, и при этом пользователь должен иметь
возможность экспериментировать \ref{circ}.

Также пользователь при запуске системы привыкает к ее определенному состоянию.
До того как мы реализовали \term{гибернацию} \ref{hyb} мы можем пойти более
простым путем, используя \term{файлы инициализации} при запуске системы,
устанавливающие ее начальное состояние.

В исходном коде выше вы видите получение полного пути фала инициализации
\file{hico.ini} лежащего в том же каталоге, что и ядро интерпретатора. Перед
появлением командной строки в интерпретатор последовательно подается исходный
код из \file{.ini} и необязательных файлов, указанных в параметрах командной
строки. При желании вы можете сами добавить загрузку файла \file{~/.hico}\ ---
традиционный способ хранения настроек программ для конкретного пользователя в
домашнем каталоге.

