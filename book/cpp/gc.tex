\secrel{Сборка мусора}\label{cppgc}

\begin{itemize}[nosep]
\item
\url{https://rsdn.org/article/cpp/GCcpp.xml}
\item 
\url{https://habr.com/ru/post/213225/}
\end{itemize}

\term{Сборка мусора}, или \term{Garbage Collection}\note{сокращённо GC},
представляет собой часть процесса автоматического управления памятью для
повторного её использования. В задачу сборки мусора входит поиск всех объектов,
которые большее не используются системой, и их корректного удаления с вызовом
деструкторов. Объекты приложения рассматриваются как мусор, если они ни прямо,
ни косвенно не доступны работающей программе.

Использование автоматического управления памятью рассматривается как необходимый
шаг к обеспечению надёжности программы, поскольку в этом случае мы получаем
решение проблемы \term{висячих указателей} (dangling pointers) и \term{утечек
памяти} (memory leaks), ценой накладных расходов на менелжмент памяти, и
сложностей с обеспечением режима реального времени и многопоточности для
неблокирующего сборщика мусора.

Если сборка мусора не встроена в язык программирования, её очень тяжело
использовать. К сожалению, сегодня стандартный \cpp\ эту возможность не
поддерживает. Поэтому программистам, желающим использовать сборку мусора,
приходится реализовывать тот или иной алгоритм сборки мусора вручную. Самым
распространённым вариантом является так называемый подсчёт ссылок. Правда, этот
алгоритм содержит крупный недостаток\ --- проблему утилизации \emph{цикличных
графов}. Соответственно, этот метод нам принципиально не подходит, поскольку во
фреймовых графах такие ссылки встречаются постоянно.

Таким образом, сборка мусора использует \term{критерий доступности объекта}.
Доступные объекты, условно можно разделить на два вида:
\begin{description}[nosep]
\item[периметр] объекты, непосредственно доступные программе
\item[периметр] 
\end{description}
, и объекты, доступные косвенно, то есть
доступные только через указатели внутри других объектов. Объекты первого вида
часто называют ом, а второго – объектами внутри периметра. Поэтому мы
будем говорить, что всё, что находится за пределами периметра, недоступно
программе и, следовательно, подлежит удалению.

В дальнейшем мы будем называть объекты, с которыми работает сборка мусора, управляемыми (managed) объектах, в отличие от неуправляемых (unmanaged) объектов, которые не подвержены процессу сборки мусора.
