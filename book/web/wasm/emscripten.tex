\secrel{\ems: старт на \cpp}\secdown

\noindent
\url{https://tproger.ru/translations/introduction-to-webassembly/}

\bigskip
Начинать проще с высокоуровневого тулчейна для \emc/\cpp\ --- для него проще
найти tutorialы, и собрать необходимые библиотеки. Недостатком является
длительная установка, и настройка рабочей среды. 

\secrel{Локальная установка}

\noindent
SDK \ems\ предоставляет обширный набор инструментальных средств для разработки
на \cpp\ для фронтенда. \ems\ не поставляется в виде готовых пакетов для Debian
\linux, и использует собственную систему установки со встроенной поддержкой
обновлений до новых версий SDK:
\begin{lstlisting}
$ cd ~
$ git clone --depth 1 \
	https://github.com/emscripten-core/emsdk.git
$ sudo apt install git cmake nodejs python2.7
$ cd emsdk

~/emsdk$ git pull
~/emsdk$ ./emsdk update
~/emsdk$ ./emsdk install latest
~/emsdk$ ./emsdk activate latest
~/emsdk$ source ./emsdk_env.sh
\end{lstlisting}

\secrel{Первые программы}

\noindent
\url{https://tproger.ru/translations/webassembly-tutorial-first-steps/}\\
\url{https://tproger.ru/translations/introduction-to-webassembly/}

\bigskip
\lst{web/wasm/Makefile}{}

\bigskip
\lst{web/wasm/none.c}{language=C}
На выходе получаем бинарный файл, содержащий 
\begin{lstlisting}[title=none.wasm]
0000000 060400 066563 000001 000000 007400 062006 066171
0000020 160153 140201 002002 000000 000400 002426 000540
...
\end{lstlisting}

\lst{web/wasm/hello.c}{language=C}

\secup
