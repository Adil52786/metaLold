\secrel{\ems: старт на \cpp}\secdown

\noindent
\url{https://tproger.ru/translations/introduction-to-webassembly/}

\bigskip
Начинать проще с высокоуровневого тулчейна для \emc/\cpp\ --- для него проще
найти tutorialы, и собрать необходимые библиотеки. Недостатком является
длительная установка, и настройка рабочей среды. 

\secrel{Локальная установка}

\noindent
SDK \ems\ предоставляет обширный набор инструментальных средств для разработки
на \cpp\ для фронтенда. \ems\ не поставляется в виде готовых пакетов для Debian
\linux, и использует собственную систему установки со встроенной поддержкой
обновлений до новых версий SDK:
\begin{lstlisting}
$ cd ~
$ git clone --depth 1 \
	https://github.com/emscripten-core/emsdk.git
$ sudo apt install git cmake nodejs python2.7
$ cd emsdk

~/emsdk$ git pull
~/emsdk$ ./emsdk update
~/emsdk$ ./emsdk install latest
~/emsdk$ ./emsdk activate latest
~/emsdk$ source ./emsdk_env.sh
\end{lstlisting}

\secrel{Первые программы}

\noindent
\url{https://tproger.ru/translations/webassembly-tutorial-first-steps/}\\
\url{https://tproger.ru/translations/introduction-to-webassembly/}

\bigskip
\lst{web/wasm/Makefile}{}

\bigskip
\lst{web/wasm/none.c}{language=C}
На выходе получаем бинарный файл, содержащий 
\begin{lstlisting}[title=none.wasm]
0000000 060400 066563 000001 000000 007400 062006 066171
...
\end{lstlisting}

\lst{web/wasm/hello.c}{language=C}

При компиляции комплятору можно передать различные флаги:
\begin{description}

\item[-o путь\_к\_выходному\_файлу] указывает путь к файлу, который надо
сгенерировать, обычно это либо файл wasm, либо js-файл, коорый загружает
скомпилированный модуль wasm, либо html-страница, на которой загружается модул
wasm

\item[-g] генерирует отладочную информацию

\item[-s option=value] устанавливает настройки компиляции. Например, некоторые
параметры компиляции:

\item[-s WASM=1] эта опция указывает компилятору сгенерировать .wasm

\item[-s ONLY\_MY\_CODE=1] указывает компилятору не включать код из стандартной
библиотеки \emc/\cpp\ в компилируемый модуль wasm\ --- он будет включать только
неспоредственно тот код, который мы сами пишем

\item[-s EXPORTED\_FUNCTIONS='{[}...{]}'] определяет набор функций, который
должны быть экспортированы из wasm

\item[-s SIDE\_MODULE=1] эта опция указывает компилятору, что надо создать
только модуль wasm

\item[-O{[}уровень\_оптимизации{]}] указывает, какой уровень оптимизации следует
использовать при компиляции. Зачастую используется третий высший уровень, то
есть -O3

\end{description}

\secup
