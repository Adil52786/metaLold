\clearpage
\secrel{Google Cloud Platform}\label{gcp}\secdown

\href{https://cloud.google.com/appengine/docs/standard/python/getting-started/python-standard-env}{Getting
Started with Flask on App Engine Standard Environment}

\begin{itemize}[nosep]

  \item 
Перейдите в \href{https://console.cloud.google.com}{консоль Google Cloud}

  \item 
\url{https://cloud.google.com/sdk/docs/}

\end{itemize}

\medskip\noindent
структура проекта:
\medskip

\begin{description}[nosep]
\item[app.yaml]: конфигурация приложения Google App Engine
\item[metaL.py]: ваше приложение
\item[static/]: каталог для хранения статических файлов
\begin{description}[nosep]
\item[dark.css]: темная тема приложения
\end{description}
\item[template/] шаблоны HTML страниц
\begin{description}[nosep]
\item[index.html]: главная страница: лог и форма командной строки 
\item[dump.html]: дамп объекта из словаря
\item[viz.html]: визуализация через D3js
\end{description}
\end{description}

\clearpage
При развертывании приложения мы можете указать в этом файле настройки Python
Application Engine, в том числе сторонние библиотеки, установленные в каталог
\file{lib/}:
\lst{../appengine_config.py}{title=appengine\_config.py,language=Python}

Файл указывает какие библиотеки требуются приложению:
\lst{../requirements.txt}{title=requirements.txt,language=Python}

\clearpage\noindent
Для разработки под Google Cloud необходимо использование \file{virtualenv}

\begin{verbatim}
$ sudo pip install --upgrade pip virtualenv
~/metaL$ virtualenv --python python2 env
\end{verbatim}

Настройки и запуск проекта выполняются из виртуальной среды в командной строке:

\begin{verbatim}
~/metaL$ source env/bin/activate
(env) user@user:~/metaL$ deactivate 
\end{verbatim}



\secup
