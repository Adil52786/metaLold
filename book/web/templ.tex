\secrel{Шаблоны и CSS}

При разработке сайтов и веб-приложений все страницы должны выводиться в
контролируемом оформлении (цвета, шрифты, расположение элементов),
и в одном стиле. За это отвечают каскадные листы стилей\ --- CSS.

Для веб-приложений, и современных сайтов с \term{активным бэкендом}\note{когда
на сервере находится выполняемый код, осуществуляющий связку веб-представления с
базами данных, и прикладным бизнеес-кодом}\ для формирования html-вывода
применяются \term{шаблоны}: образцы файлов страниц, в тексте которых указаны
места куда будет подставляться вывод генерируемый кодом на \py.

В файловой системе Flask-приложения создаются два каталога:
\begin{description}[nosep]
\item[static/] для хранения статических файлов
\verb|web.send_static_file()|
\item[templates/] для шаблонов Jinja \verb|web.render_template()|
\end{description}

\lst{web/templ.py}{language=Python}

Веб-приложение снаружи выглядит как набор множества html-страниц. Использование
шаблонов дает возможность поместить весь повторяющийся код страниц в одно место,
и подставлять изменяющуюся часть html динамически в момент отправки страницы
браузеру клиента.

\clearpage
\lst{web/templ.css}{title=static/dark.css}
\lst{web/templ.html}{title=templates/index.html}
\fig{web/templ.png}{width=\textwidth}
