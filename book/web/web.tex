\secrel{Web-интерфейс /Flask/}\label{web}\secdown

Система по умолчанию запускается в консольном режиме, и не требует никаких
сторонних библиотек. Это удобно при запуске \hico\ в консольном окне
\eclipse, или в командной строке. Если вы хотите удобства, например иметь
несколько окон, показывающих состояние различных объектов в словаре системы,
стоит перейти к Web-интерфейсу, обслуживаемому \term{web-фрейморк}ом Flask. Он
не так известен как Django, это минималистичен и не тянет за собой массу
библиотек, базу данных, и не требует для работы настройку виртуального
окружения.

\medskip
\lst{web/web.py}{language=Python}
\begin{description}[nosep]
\item[\file{WEB()}] весь веб-интерфейс завернут в функцию\\ если у вас не
установлены библиотеки Flask, до попытки ее запуска никаких проблем не
возникнет, консольный режим будет работать
\item[\file{IP}] сервис запускается на \file{localhost}.\\ \emph{не запускайте
\hico\ на открытых IP}, так как веб-интерфейс запускается в отладочном режиме, и
становится дыркой в безопасности.
\item[\file{PORT}] порт сервиса\\ при запуске от пользователя в \linux\ доступны
порты выше 8000
\item[\file{web}] Flask-приложение
\item[\file{SECRET\_KEY}] используется в защите веб-форм от cross-site атак\\
инициализируется примитивно\ --- случайной строкой
\item[\file{.route}] роутинг задает маршрутизацию веб-запросов\\ привязывает
путь к ресурсу в URL к прикладному коду, оформленному как функция-обработчик
запроса
\item[\file{index()}] обработчик \file{http://127.0.0.1:8888 /}\\
в простейшем случае нам нужно обрабатывать запросы только к корневому URL, и
отдать кусок html, в примере отдается дамп словаря в plain text завернутый в тег
\verb|<pre>|
\item[.run(ip,port,debug=True)] запуск веб-сервиса в отладочном режиме\\ при
появлении исключений при запуске ваших команд будет выведен дамп ошибки, и
сервис не прекратит работу аварийно
\end{description}

\smallskip
\fig{web/first.png}{height=.5\textheight}

\begin{verbatim}
 * Serving Flask app "hico" (lazy loading)
 * Environment: production
   WARNING: Do not use the development server
            in a production environment.
   Use a production WSGI server instead.
 * Debug mode: on
 * Running on http://127.0.0.1:8888/ (Press CTRL+C to quit)
 * Restarting with stat
 * Debugger is active!
 * Debugger PIN: 299-303-273
127.0.0.1 - - [19/Mar/2019 14:58:45] "GET / HTTP/1.1" 200 -
127.0.0.1 - - [19/Mar/2019 15:01:24] "GET / HTTP/1.1" 200 -
127.0.0.1 - - [19/Mar/2019 15:01:25] "GET / HTTP/1.1" 200 -
\end{verbatim}

\secrel{Экспорт насроек в адресное пространство Форта}

Для настройки запуска веб-интерфейса через .ini файлы вынесем настройки в
адресное пространство \F-системы. Перез запуском команды \file{WEB} вы можете
поменять их.

\medskip
\lst{web/iport.py}{language=Python}

\secrel{Развертывание на халявном хостинге}

\url{https://www.pythonanywhere.com/}

\bigskip\noindent
Для выпендрёжа и тестирования на мобильном телефоне развернем \metal\ на одном
из бесплатных \py-хостингов.

\begin{verbatim}
Создайте веб-приложение Python 2.7 (Flask 1.0.2)
Укажите стартовый файл сервиса: /home/metaL/metaL/metaL.py
Working directory: /home/metaL/ [Go to directory]
/home/metaL/metaL [Open Bash console here]

~/metaL$ rm -rf *
$ git init
$ git remote add gh https://github.com/ponyatov/metaL.git
$ git pull -v gh master
\end{verbatim}

\url{http://metal.pythonanywhere.com/}


\secup
