\secrel{Web-интерфейс /Flask/}\label{web}\secdown

Система по умолчанию запускается в консольном режиме, и не требует никаких
сторонних библиотек. Это удобно при запуске \metal\ в консольном окне
\eclipse, или в командной строке. Если вы хотите удобства, например иметь
несколько окон, показывающих состояние различных объектов в словаре системы,
стоит перейти к Web-интерфейсу, обслуживаемому \term{web-фрейморк}ом Flask. Он
не так известен как Django, это минималистичен и не тянет за собой массу
библиотек, базу данных, и не требует для работы настройку виртуального
окружения.

\medskip
\lst{web/web.py}{language=Python}
\begin{description}[nosep]
\item[\file{WEB()}] весь веб-интерфейс завернут в функцию\\ если у вас не
установлены библиотеки Flask, до попытки ее запуска никаких проблем не
возникнет, консольный режим будет работать
\item[\file{IP}] сервис запускается на \file{localhost}.\\ \emph{не запускайте
\metal\ на открытых IP}, так как веб-интерфейс запускается в отладочном режиме, и
становится дыркой в безопасности.
\item[\file{PORT}] порт сервиса\\ при запуске от пользователя в \linux\ доступны
порты выше 8000
\item[\file{web}] Flask-приложение
\item[\file{SECRET\_KEY}] используется в защите веб-форм от cross-site атак\\
инициализируется примитивно\ --- случайной строкой
\item[\file{.route}] роутинг задает маршрутизацию веб-запросов\\ привязывает
путь к ресурсу в URL к прикладному коду, оформленному как функция-обработчик
запроса
\item[\file{index()}] обработчик \file{http://127.0.0.1:8888 /}\\
в простейшем случае нам нужно обрабатывать запросы только к корневому URL, и
отдать кусок html, в примере отдается дамп словаря в plain text завернутый в тег
\verb|<pre>|
\item[.run(ip,port,debug=True)] запуск веб-сервиса в отладочном режиме\\ при
появлении исключений при запуске ваших команд будет выведен дамп ошибки, и
сервис не прекратит работу аварийно
\end{description}

\smallskip
\fig{web/first.png}{height=.5\textheight}

\begin{verbatim}
 * Serving Flask app "metaL" (lazy loading)
 * Environment: production
   WARNING: Do not use the development server
            in a production environment.
   Use a production WSGI server instead.
 * Debug mode: on
 * Running on http://127.0.0.1:8888/ (Press CTRL+C to quit)
 * Restarting with stat
 * Debugger is active!
 * Debugger PIN: 299-303-273
127.0.0.1 - - [19/Mar/2019 14:58:45] "GET / HTTP/1.1" 200 -
127.0.0.1 - - [19/Mar/2019 15:01:24] "GET / HTTP/1.1" 200 -
127.0.0.1 - - [19/Mar/2019 15:01:25] "GET / HTTP/1.1" 200 -
\end{verbatim}

\secrel{Экспорт настроек в адресное пространство Форта}

Для настройки запуска веб-интерфейса через .ini файлы вынесем настройки в
адресное пространство \F-системы. Перез запуском команды \file{WEB} вы можете
поменять их.

\medskip
\lst{web/iport.py}{language=Python}

\secrel{Развертывание на халявном хостинге}

\url{https://www.pythonanywhere.com/}

\bigskip\noindent
Для выпендрёжа и тестирования на мобильном телефоне развернем \metal\ на одном
из бесплатных \py-хостингов.

\begin{verbatim}
Создайте веб-приложение Python 2.7 (Flask 1.0.2)
Укажите стартовый файл сервиса: /home/metaL/metaL/metaL.py
Working directory: /home/metaL/ [Go to directory]
/home/metaL/metaL [Open Bash console here]

~/metaL$ rm -rf *
$ git init
$ git remote add gh https://github.com/ponyatov/metaL.git
$ git pull -v gh master
\end{verbatim}

\url{http://metal.pythonanywhere.com/}


\clearpage
\secrel{WebAssembly}\secdown

Интересная заморочка для кросс-разработки игр, запускающихся в браузере, на
\metal. В этом разделе мы сделаем кодогенератор, который запускается по
запросу через веб-интерфейс, и отдает клиенту образ сгенерированного \wasm-кода.
Изменяя набор игровых метамоделей внутри \metal, мы можем менять логику игры, но
\wasm\ пока не поддерживает обновление кода, поэтому рестартовать игру придется
вручную.

\bigskip
\url{https://habr.com/ru/company/jugru/blog/441140/}

\secrel{WasmFiddle}

\noindent
\url{https://wasdk.github.io/WasmFiddle/}\\
\url{https://habr.com/ru/post/342180/}
\bigskip

Попробовать технологию без установки можно online.

\bigskip
\lst{web/wasm/none.c}{language=C}
% \bigskip

Слева вверху исходный код, слева внизу текстовое представление результата
компиляции по кнопке Build, справа вверху \js\ код для запуска и справа
внизу результат запуска по кнопке Run.

WebAssembly это бинарный формат, для удобства и отладки существует механизм
текстового представления один-в-один в виде текста в формате WAT.

\lst{web/wasm/none.wat}{}

\secrel{Интерфейс WASM/\js}

\noindent
\fig{web/wasm/files.png}{height=.4\textheight}
\fig{web/wasm/api.png}{height=.7\textheight}

\medskip\noindent
WebAssembly может пользоваться любыми API, но это возможно только через \js: код
WASM может вызывать код на \js, и наоборот.

\secrel{Управление памятью}

Модель памяти WebAssembly очень проста. Это плоский «кусок» памяти, в котором
находится код программы, глобальные переменные, стек и куча. Есть возможность
сделать так, чтобы память была расширяемой, то если если при очередном выделении
памяти нам не хватает места, то верхняя граница памяти автоматически
увеличивается.
Весь блок памяти доступен из JavaScript как на чтение так и на
запись как массив байт.

\fig{web/wasm/memory.png}{width=\textwidth}


\secrel{\ems: старт на \cpp}\secdown

\noindent
\url{https://tproger.ru/translations/introduction-to-webassembly/}

\bigskip
Начинать проще с высокоуровневого тулчейна для \emc/\cpp\ --- для него проще
найти tutorialы, и собрать необходимые библиотеки. Недостатком является
длительная установка, и настройка рабочей среды. 

\secrel{Локальная установка}

\noindent
SDK \ems\ предоставляет обширный набор инструментальных средств для разработки
на \cpp\ для фронтенда. \ems\ не поставляется в виде готовых пакетов для Debian
\linux, и использует собственную систему установки со встроенной поддержкой
обновлений до новых версий SDK:
\begin{lstlisting}
$ cd ~
$ git clone --depth 1 \
	https://github.com/emscripten-core/emsdk.git
$ sudo apt install git cmake nodejs python2.7
$ cd emsdk

~/emsdk$ git pull
~/emsdk$ ./emsdk update
~/emsdk$ ./emsdk install latest
~/emsdk$ ./emsdk activate latest
~/emsdk$ source ./emsdk_env.sh
\end{lstlisting}

\secrel{Первые программы}

\noindent
\url{https://tproger.ru/translations/webassembly-tutorial-first-steps/}\\
\url{https://tproger.ru/translations/introduction-to-webassembly/}

\bigskip
\lst{web/wasm/Makefile}{}

\bigskip
\lst{web/wasm/none.c}{language=C}
На выходе получаем бинарный файл, содержащий 
\begin{lstlisting}[title=none.wasm]
0000000 060400 066563 000001 000000 007400 062006 066171
...
\end{lstlisting}

\lst{web/wasm/hello.c}{language=C}

При компиляции комплятору можно передать различные флаги:
\begin{description}

\item[-o путь\_к\_выходному\_файлу] указывает путь к файлу, который надо
сгенерировать, обычно это либо файл wasm, либо js-файл, коорый загружает
скомпилированный модуль wasm, либо html-страница, на которой загружается модул
wasm

\item[-g] генерирует отладочную информацию

\item[-s option=value] устанавливает настройки компиляции. Например, некоторые
параметры компиляции:

\item[-s WASM=1] эта опция указывает компилятору сгенерировать .wasm

\item[-s ONLY\_MY\_CODE=1] указывает компилятору не включать код из стандартной
библиотеки \emc/\cpp\ в компилируемый модуль wasm\ --- он будет включать только
неспоредственно тот код, который мы сами пишем

\item[-s EXPORTED\_FUNCTIONS='{[}...{]}'] определяет набор функций, который
должны быть экспортированы из wasm

\item[-s SIDE\_MODULE=1] эта опция указывает компилятору, что надо создать
только модуль wasm

\item[-O{[}уровень\_оптимизации{]}] указывает, какой уровень оптимизации следует
использовать при компиляции. Зачастую используется третий высший уровень, то
есть -O3

\end{description}

\secup

\secrel{WebAssembly Binary Toolkit}\label{wabt}

После того как вы немного ознакомились с приненением \wasm, можно от верхнего
уровня спустится на уровень текстового и бинарного представления кода
WebAssembly. Для работы на низком уровне предназначен

\url{https://github.com/WebAssembly/wabt}


\secup

\secup
