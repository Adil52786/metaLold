\secrel{Метод EDS: исполняемые структуры данных}\label{eds}

По-настоящему мощные и гибкие языки программирования, и скриптовые
движки встроенные в прикладное ПО\note{используются как средства расширения
пользователем прикладных коммерческих программных пакетов и интегрированных
рабочих сред}, дающие множество возможностей даже в самом минимальном варианте
исполнения, в большинстве случаев строятся по методике, которую можно назвать
\term{исполняемые структуры данных}\note{EDS\ --- Executable Data Structure}.

\begin{description}[nosep]
\item[интерпретатор структур данных] рассматривает их как некоторое
представление программ, которые можно исполнять в выбранной модели: конечный
автомат, императивная программа, декларативное описание состава системы, функций
над данными и потоков данных,\ldots и
\item[память интерпретатора], в которой эти структуры хранятся.
\end{description}

Такие структуры \term{рефлективны} (программа может анализировать сама себя) и
часто \term{мутабельны} (их можно изменять в процессе работы), представляют
бизнес-логику программы на верхнем уровне абстракции, что добавляет программисту
некоторые возможности, недоступные при обычном использовании компиляторов.

Совсем необязательно, что применение метода EDS связано с созданием языка
программирования\ --- \emph{исполняемые структуры вполне могут быть заданы в
синтаксисе \emc\ и скомпилированы для микроконтроллера с минимальным объемом
памяти}. И все же свойство гомоиконичности (см.выше) EDS очень располагает к
появлению прикладного языка\note{DSL\ --- Domain Specific Language} хотя бы для
описания начального состояния таких структур\note{DDL\ --- Data Definition
Language}.
