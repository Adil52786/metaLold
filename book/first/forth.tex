\clearpage
\secrel{Язык \metal: исправленный \F}\secdown

Хотя мы стараемся уйти от использования языка программирования как основного
средства разработки \ref{nolang}, в любом случае нам нужен способ ввода данных и
систему, и управления вычислениями.

\emph{\metal\ не является языком
программирования}, это \term{командный язык} с помощью которого выполняется
\begin{itemize}[nosep]
  \item 
создание фреймов, 
  \item 
модификация \term{фреймовой базы знаний}, 
  \item 
запуск/останов скриптов и демонов. 
\end{itemize}
Однако очень простое \term{императивное
программирование} может выполняться и на \metal, так как этот язык позволяет
определять новые \F-\term{слова}, и поддерживает \term{конкатенативное
программирование} через разделяемый стек.

\clearpage
В качестве прототипа для \metal\ был выбран язык \emph{\F: это самый
элементарный язык программирования}, который вы только можете найти. Вы можете
самостоятельно написать свой \F\ за пару вечеров или пару недель на любом языке
программирования, и для любого типа компьютера.

% \smallskip\noindent
\F\ был создан в 70х годах Чаком Муром для управления оборудованием
(радиотелескопом), и \emph{\F\ до сих пор великолепен в роли командной оболочки}
(CLI) для подобных задач. В том числе \F\ очень хорошо подходит как командная
консоль для микроконтроллеров с очень небольшими объемами ОЗУ порядка 8-20
Кило(!)байт.

Но в роли основного языка программирования \F\ очень плох:
\begin{description}[nosep]
\item[низкоуровневая модель ВМ языка]: \F\ по факту является ассемблером
\term{виртуальной стековой машины}, и как с любым ассемблером вам приходится
самостоятельно выписывать все фишки, которые в mainstream языках доступны из
коробки в базовой спецификации языка. Как пример, в стандартное ядро языка не
входит поддержка вычислений с плавающей точкой, и полностью отсутствуют средства
динамического выделения памяти.
\item[прямой доступ к памяти по адресам]\ делает код на \F е крайне
нестабильным: ошибки в адресации тут же приводят к перезаписи данных и кода по
случайным адресам. В результате при программировании на \F\ нужно работать с
памятью на порядок аккуратнее, чем на \emc.
\end{description}

\secup
