\clearpage
\secrel{Язык \metal: исправленный \F}\secdown

Хотя мы стараемся уйти от использования языка программирования как основного
средства разработки \ref{nolang}, в любом случае нам нужен способ ввода данных и
систему, и управления вычислениями.

\emph{\metal\ не является языком
программирования}, это \term{командный язык} с помощью которого выполняется
\begin{itemize}[nosep]
  \item 
создание фреймов, 
  \item 
модификация \term{фреймовой базы знаний}, 
  \item 
запуск/останов скриптов и демонов. 
\end{itemize}

\medskip\noindent
Однако простое \term{императивное программирование} может выполняться и на
\metal, так как этот язык позволяет определять новые \F-\term{слова}, и
поддерживает \term{конкатенативное программирование} через разделяемый стек.

\clearpage
В качестве прототипа для \metal\ был выбран язык \emph{\F: это самый
элементарный язык программирования}, который вы только можете найти. Вы можете
самостоятельно написать свой \F\ за пару вечеров или пару недель на любом языке
программирования, и для любого типа компьютера.\note{Современному
профессиональному разработчику, особенно фрилансеру, каждые 3-5 лет приходится
изучать очередной модный язык программирования, или фреймворк. Неплохим учебным
упражнением может стать реализация но новом для вас языке минимального \F, или
лучше \metal\ --- он совместим с объектной моделью современных языков
программирования, не потребует симуляции образа памяти с побайтным доступом,
и неплох в качестве простого \term{cкриптового движка}.}

% \smallskip\noindent
\F\ был создан в 70х годах Чаком Муром для управления оборудованием
(радиотелескопом), и \emph{\F\ до сих пор великолепен в роли командной оболочки}
(CLI) для подобных задач. В том числе \F\ очень хорошо подходит как командная
консоль для микроконтроллеров с очень небольшими объемами ОЗУ порядка 8-20
Кило(!)байт.

\noindent
Но в роли основного языка программирования классический \F\ плох:
\begin{description}[nosep]
\item[низкоуровневая модель ВМ языка]: \F\ по факту является ассемблером
\term{виртуальной стековой машины}, и как с любым ассемблером вам приходится
самостоятельно выписывать все фишки, которые в mainstream языках доступны из
коробки. Как пример, в стандартное ядро \F\ не входит поддержка вычислений с
плавающей точкой, и полностью отсутствуют средства динамического выделения
памяти.
\item[прямой доступ к памяти по адресам]\ делает код на \F е крайне
нестабильным: ошибки в адресации тут же приводят к перезаписи данных и кода по
случайным адресам. В результате при программировании на \F\ нужно работать с
памятью на порядок аккуратнее, чем на \emc.
\end{description}

\noindent
Из-за этих проблем было решено полностью отказаться от использования
классического \F а, оставив от него в новом языке \metal\ кое-какие фишки типа
синтаксиса, стака данных, словаря, и постфиксной нотации.

\clearpage
Перед тем как приступать к работе с \metal, сначала следует немного поиграть с
классическим \F ом используя несколько книжек, и какую-нибудь реализацию \F\ для
вашего компьютера. Как вариант, вы можете попробовать GNU Forth для Android, или
декстопа: \url{http://gforth.org/}

\begin{framed}\noindent
читайте эти книги очень аккуратно, желательно в перчатках и защитных очках,
иначе заразитесь и станете хроническим фортёром
\end{framed}

\begin{itemize}[nosep]
  \item Броуди \cite{starting}, и \cite{thinking} если хотите копнуть идеологию
  разработки на \F\ поглубже
  \item Келли, Спайс \cite{kelly} немного путано, но доставляет
  \item Баранов, Ноздрунов \cite{baranov} книжка с качеством ниже среднего, но
  это единственная книга на русском языке \emph{по реализации} \F
  \item Threaded Languages \cite{threaded} намного глубже, но на английском
\end{itemize}

\secup
