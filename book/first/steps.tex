\secrel{Первые шаги}\secdown

Перед вами книга, посвященная созданию очень примитивных языков программирования
на \py. Здесь вы не найдете ленивых лямбд и жутких монад, живущих в лесу
Хомского, и прочей бурбулятристики про обощенный вывод типов.
И все же мне хочется пошагово показать создание собственной реализации языка
программирования, более близкого по ощущениям к Self, Smalltalk и \lisp.
В реализациях этих языков вы можете \emph{программно} строить части программ во
время исполнения, вмешиваться в процесс работы ядра языка, и добавлять в язык
различные возможности других языков, например смешивать фнукциональное,
императивное и логическое программирование.

\begin{quotation}\noindent
Метапрограммирование — вид программирования, связанный с созданием
\textit{программ, которые порождают другие программы} как результат своей работы
(в частности, на стадии компиляции их исходного кода), либо программ, которые
меняют себя во время выполнения (самомодифицирующийся код).
\end{quotation}

\begin{itemize}
  \item 
\url{https://www.youtube.com/watch?v=QKFrxEkVusg}
  \item 
\url{https://www.youtube.com/watch?v=bt6kU1kuHWA}
\end{itemize}

Такие богатейшие возможности \term{метапрограммирования} возможны благодара
тому, что эти языки \term{гомоиконичны}: их реализацации работают как
\emph{живая интерактивная система} используючая структуры данных как
представление программы и исполняемый код.

\begin{quotation}\noindent
\term{Гомоиконичность} (гомоиконносль, англ. homoiconicity, homoiconic)\\
свойство некоторых языков программирования, в которых \emph{представление
программ является одновременно структурами данных} определенных в типах самого
языка, \emph{доступных для просмотра и модификации}. Говоря иначе,
гомоиконичность\ --- это когда исходный \textit{код программы} пишется
\textit{как базовая структура данных}, и язык программирования знает, как
получить к ней доступ (в том числе в рантайме при работе программ у конечного
пользователя).
\end{quotation}

Реализация \term{виртуальной машины} гомоиконичного языка 

\clearpage
\paragraph{Применение}\ \\ \bigskip

\begin{itemize}[nosep]
\item \emph{обработка текстовых форматов данных}\\
	файлы САПР, исходные данные для расчетных программ
\item командный интерфейс для устройств на микроконтроллерах\\
	управление человеко-читаемыми командами, \emph{передача пакетов данных
	любой структуры и типов}
\item реализация специализированных скриптовых языков
\item обработка исходных текстов программ\\
	модификация, трансляция на другие языки программирования,\\ 
	\emph{универсальный язык независимых от языка шаблонов и метапрограммирования}
	для ЯП с ограниченными или отсутствующими макросами
\end{itemize}

\secrel{Установка}\label{install}

\url{https://github.com/ponyatov/metaL/releases/latest}

\begin{verbatim}
~$ git clone [-b master] https://github.com/ponyatov/metaL.git
~$ cd metaL
~/metaL$ python ./metaL.py
\end{verbatim}

Интерпретатор написан на диалекте \py 2, также в системе должны быть установлены
библиотеки

\begin{verbatim}
~$ sudo pip install --upgrade pip
~$ sudo pip install ply
\end{verbatim}

Для использования веб-интерфейса

\begin{verbatim}
~$ sudo pip install flask wtforms
\end{verbatim}



\secup
