\secrel{Первые шаги}\secdown

Перед вами книга, посвященная созданию очень примитивных языков программирования
на \py. Здесь вы не найдете ленивых лямбд и жутких монад, живущих в лесу
Хомского, и прочей бурбулятристики про обощенный вывод типов.
И все же мне хочется пошагово показать создание собственной реализации языка
программирования, более близкого по ощущениям к Self, Smalltalk и \lisp.
В реализациях этих языков вы можете \emph{автоматически строить части программ
во время исполнения}, вмешиваться в процесс работы ядра языка, и добавлять в
язык различные возможности нужные конкретно вам, например смешивать
фнукциональное, императивное и логическое программирование.

\begin{quotation}\noindent
Метапрограммирование — вид программирования, связанный с созданием
\textit{программ, которые порождают другие программы} как результат своей работы
(в частности, на стадии компиляции их исходного кода), либо программ, которые
меняют себя во время выполнения (самомодифицирующийся код).
\end{quotation}

\begin{itemize}
  \item 
\url{https://www.youtube.com/watch?v=QKFrxEkVusg}
  \item 
\url{https://www.youtube.com/watch?v=bt6kU1kuHWA}
\end{itemize}

Такие богатейшие возможности \term{метапрограммирования} возможны благодара
тому, что эти языки \term{гомоиконичны}: их реализацации работают как
\emph{живая интерактивная система} использующая структуры данных как
представление программы и исполняемый код.

\begin{quotation}\noindent
\term{Гомоикон\`{и}чность} (гомоиконность, англ. homoiconicity, homoiconic)\\
свойство некоторых языков программирования, в которых \emph{представление
программ является одновременно структурами данных} определенных в типах самого
языка, \emph{доступных для просмотра и модификации}. Говоря иначе,
гомоиконичность\ --- это когда исходный \textit{код программы} пишется
\textit{как базовая структура данных}, и язык программирования знает, как
получить к ней доступ на чтение и запись (в том числе в рантайме при работе
программ у конечного пользователя).
\end{quotation}

\clearpage
Реализация языков такого типа \ref{implement}\ строится на \emph{интерпретаторе
структур данных}. По какой-то странной причине почему-то принято
противопоставлять \term{интерпретатор} и \term{компилятор}. Могу вас
обрадовать: \textit{все современные интерпретаторные реализации языков
программирования в обязательном порядке включают компилятор}\note{как минимум в
байт-код}.

На самом деле
\emph{\term{интерпретатор} и \term{компилятор} не противоположны}:
интерпретатор может включать в себя компилятор в машинный код как
составную часть. Или наоборот Java считается компилятором, на самом деле
программы преобразуются в \textit{интерпретируемый} \term{байт-код}, который в
свою очередь еще раз компилируется в машинный код\note{JIT\ ---
\textit{необязательная} часть языка Java, см. JavaME на телефонах}.

Еще одна бредовая привычка пользоваться сочетаниями
``интерпретируемый/компилируемый язык'': эта фраза скрывает, чем \emph{язык}
отличвается его его \term{реализации} (я неоднократно и специально использовал
это слово).

\term{Язык программирования}\ --- это \emph{формальный набор
правил}, описывающих \term{синтаксис} (как программы выглядят), \term{семантику}
(что каждая часть значит в соответствии со стандартом языка), требования к
\term{рантайму/виртуальной машине} реализации языка, и \term{стандартную
библиотеку} набор функций и процедур, поставляемых в составе \term{реализации
языка}.

Например, язык Си всегда называют ``компилируемым языком'', но никто не
запрещает написать его интерпретатор. Васик очень часто называли
интерпретатором, но даже на ZX Spectrum был компилятор Lazer Basic. Строго
говоря для некоторых языков (в том числе \metal) нельзя сделать полный
компилятор: некоторые фичи языка могут потребовать перестройки программ в
процессе выполнения. Но даже в таких клинических случаях, как реалиции языка
\lisp, возможно использование техник \term{динамической компиляции} \ref{dyna}
в реальный машинный код. В \ref{llvm}\ мы также рассмотрим встраивание
\term{кросс-компилятора} для микроконтролеров в состав нашего интерпретатора
(техника \term{управляемой компиляции}).

\clearpage
\paragraph{Применение}\ методов программирования, описанных в этой книге:\\
\bigskip

\begin{itemize}[nosep]
\item \emph{обработка текстовых форматов данных}\\
	файлы САПР, исходные данные для расчетных программ
\item командный интерфейс для устройств на микроконтроллерах\\
	управление человеко-читаемыми командами, \emph{передача пакетов данных
	любой структуры и типов}
\item реализация специализированных скриптовых языков
\item обработка исходных текстов программ\\
	модификация, трансляция на другие языки программирования,\\ 
	\emph{универсальный язык независимых от языка шаблонов и метапрограммирования}
	для ЯП с ограниченными или отсутствующими макросами
\end{itemize}

\secrel{Метод EDS: исполняемые структуры данных}\label{eds}

По-настоящему мощные и гибкие языки программирования, и скриптовые
движки встроенные в прикладное ПО\note{используются как средства расширения
пользователем прикладных коммерческих программных пакетов и интегрированных
рабочих сред}, дающие множество возможностей даже в самом минимальном варианте
исполнения, в большинстве случаев строятся по методике, которую можно назвать
\term{исполняемые структуры данных}\note{EDS\ --- Executable Data Structure}.

\begin{description}[nosep]
\item[интерпретатор структур данных] рассматривает их как некоторое
представление программ, которые можно исполнять в выбранной модели: конечный
автомат, императивная программа, декларативное описание состава системы, функций
над данными и потоков данных,\ldots и
\item[память интерпретатора], в которой эти структуры хранятся.
\end{description}

Такие структуры \term{рефлективны} (программа может анализировать сама себя) и
часто \term{мутабельны} (их можно изменять в процессе работы), представляют
бизнес-логику программы на верхнем уровне абстракции, что добавляет программисту
некоторые возможности, недоступные при обычном использовании компиляторов.

Совсем необязательно, что применение метода EDS связано с созданием языка
программирования\ --- \emph{исполняемые структуры вполне могут быть заданы в
синтаксисе \emc\ и скомпилированы для микроконтроллера с минимальным объемом
памяти}. И все же свойство гомоиконичности (см.выше) EDS очень располагает к
появлению прикладного языка\note{DSL\ --- Domain Specific Language} хотя бы для
описания начального состояния таких структур\note{DDL\ --- Data Definition
Language}.

\secrel{Эффект белого ящика}\label{whitebox}

У метапрограммирования есть один неочевидный, но критический недостаток, который
скорее всего и делает эту методику разработки ПО\note{и по-настоящему
динамические гибкие языки программирования, такие как \F, \st, \self, \lisp,
которые в большей или меньшей степени позволяют \textit{адаптировать язык} под
нужны конкретного программиста
%, добавляя новые синтаксические конструкции, и иногда
%позволяя вмешиваться в работу внутренних механизмов
} настолько мало применяемой: \term{эффект белого ящика}.

Когда вы пользуетесь каким-нибудь широкого распространенным языком
программирования, \emph{вы изолированы от деталей реализации не только языка, но
и внутренностей большинства библиотек и фрейворков, которыми вы пользуетесь}. У
вас есть гугл, руководство по компилятору, документация на библиотеки и
литература, \textit{которые описывают интерфейсы}. Вам в очень редких случаях
приходится лезть в исходники распространенных библиотек, и практически никогда в
исходники компилятора.

\bigskip
Священная корова computer science\ --- \term{сложность разработки ПО}\ ---
скрыта за множеством слоев от прикладного программиста. Тяжелые прикладные
библиотеки, фрейворки, и особенно интерпретаторы/компиляторы выступают в роли
классического черного ящика, с очень хорошо описанными интерфейсами.

\bigskip
Как только вы начинаете использовать метапрограммирование, а особенно языки
программирования домашней варки, все пропало! Вы досконально знаете каждый
закоулок ваших инструментов, и голова взрывается от обилия деталей. Этот эффект
известен для обычных больших проектов, но он не так заметен, так как набор
сложности происходит постепенно, и не так болезненен: ``это то же большой
проект, вполне ожидаемо, что он будет сложным''. При интенсивном использовании
метапрограммирования набор (кажущейся?) сложности происходит очень быстро, даже
на игрушечных проектах.

\secrel{Установка}\label{install}

\url{https://github.com/ponyatov/metaL/releases/latest}

\begin{verbatim}
~$ git clone [-b master] https://github.com/ponyatov/metaL.git
~$ cd metaL
~/metaL$ python ./metaL.py
\end{verbatim}

Интерпретатор написан на диалекте \py 2, также в системе должны быть установлены
библиотеки

\begin{verbatim}
~$ sudo pip install --upgrade pip
~$ sudo pip install ply
\end{verbatim}

Для использования веб-интерфейса

\begin{verbatim}
~$ sudo pip install flask wtforms
\end{verbatim}


\clearpage
\secrel{Первый запуск}

\begin{verbatim}
~/ouroboros $ python py.py

ok> _
\end{verbatim}

\noindent
Основная реализация с веб-интерфейсом на \url{http://127.0.0.1:8888/}:

\medskip
\fig{first/webrun.png}{height=.5\textheight}

\clearpage
\fig{first/webrowser.png}{height=\textheight}

\secrel{Фреймы Мински}\label{minsky}

\begin{framed}\noindent\label{nolang}
\emph{\metal\ не рассматривает язык программирования как основной метод
разработки}: задача программиста\ --- строить \term{модели} решаемых задач в
памяти системы, формируя базу знаний с использованием \term{фреймов} Мински
\cite{minsky} для представления знаний.
\end{framed}

\term{Фрейм}\ в оригинальной формулировке понятия\ --- способ
\emph{структурного} \term{представления знаний} в искуственном интеллекте,
введенный Марвином Мински для описания струтуры знаний для восприятия
пространственных сцен \cite{minsky}. В той же книге это понятие было расширено
до универсальной структуры данных, очень близкой к понятию объекта в ООП.

\clearpage\noindent
Фреймы имеют именованные слоты (поля объекта), которые могут хранить
\begin{description}
\item[примитивные значения] (строки, числа)\note{в ИИ рассматриваются как
значения по умолчанию при описании ситуации}
\item[ссылки] на другие фреймы
\item[активные слоты] (методы объекта) содержат исполняемые процедуры,
запускаемые по внешним или внутренним событиям
\item[пустые слоты] не имеют заданного значения, и заполняются в процессе
логического вывода или работы процедур
\end{description} 

\medskip\noindent
Наличие исполняемых элементов превращает фреймовую модель\note{в общем случае\
--- любую структуру данных, см \ref{esd}} в \emph{модель вычислительную}, и
\term{парадигму программирования}, совмещающую в себе декларативное,
императивное, объектно-ориентированное и логическое программирование. Именно эта
универсальнось определила выбор фреймовой модели для \metal, как наиболее
подходящей для описания структур и процессов, специфичных для программного
обеспечения, и аппаратно-программных комплексов (embedded, \term{встраиваемые
системы}).

\vspace{7mm}
Фрейм Мински был адаптирован для представления исходного кода программ и
описания данных на любых языках (программирования). Для этого во фрейм была
добавлена возможность хранения элементов \emph{в упорядоченном виде}, что
автоматически делает \term{фреймовые сети} представлением \term{атрибутных
грамматик}. Также были добавлены дополнительные поля:

\lst{frames/frame.py}{language=Python}
\clearpage

\clearpage
\secrel{Концепт vs модель}\label{concept}

\metal\ --- язык не только мета-, но и \term{концептуального программирования}.
\term{Концепт}\ это \emph{не полностью определенная} модель. Например, указав

\begin{lstlisting}
module: hello
\end{lstlisting}
\noindent
мы сужаем множество всех возможных программ до одного программного модуля.

\begin{lstlisting}
function: main << ( добавить в module:hello )
\end{lstlisting}
\noindent
накладывает ограничение на то что целевая программа будет на \emc/\cpp.

\medskip
В mainstream языках программирования мы досконально описываем каждый элемент
процесса, пошагово, точно прописываем все свойства объектов и структуру
программы. При программировании на \metal\ мы постепенно \emph{накладываем
ограничения}, сужая все множество возможных программ с помощью условий.
При этом чем больше и больше ограничений мы применям, тем ближе концепт
программы становится ее моделью:
\begin{description}
\item[модель] точное поэлементное описание системы с указанием всех
взаимосвязей между объектами.
\item[концепт] более широкое понятие так как он преднамеренно \emph{описывает
систему только частично}, указывая что у объекта \emph{может быть} такое-то
свойство, есть такой-то класс с некоторыми методами (но при этом не
декларируются \emph{все методы}).
\end{description}

\noindent
Для реализации \term{концептуального программирования} \cite{tyugu} необходима
\term{база знаний} в которой задаются
\begin{itemize}
  \item 
\emph{наборы ограничений}, которые мы называем \term{концептами},
\item 
\emph{куски кода}\ --- \term{сниппеты}, которые добавляются в результирующий
код при срабатывании правил и ограничений, причем эти куски \emph{заданы
параметрически} и частично изменяются при использовании
 \item
предыдущие проекты, заданные как наборы ограничений\ --- вы можете
\term{наследовать} их целиком или частями
\item 
\emph{правила логического вывода} которые осуществляют синтез кода по сети
ограничений
\item 
\term{скрипты}\ --- \term{императивные} процедуры, выполняемые внутри базы
знаний, для ее анализа или трансформации
\item
\term{демоны}\ --- скрипты, выполнемые в фоновом режиме: оптимизации, сборка
мусора, JIT- и фоновая компиляция скриптов и вычислительных функций в машинный
код
\end{itemize} 

\clearpage
\secrel{Язык \metal: исправленный \F}

Хотя мы стараемся уйти от использования языка программирования как основного
средства разработки \ref{nolang}, в любом случае нам нужен способ ввода данных и
систему, и управления вычислениями. Несмотря на десятки лет  


\secup
