\clearpage
\secrel{Концепт vs модель}\label{concept}

\metal\ --- язык не только мета-, но и \term{концептуального программирования}.
\term{Концепт}\ это \emph{не полностью определенная} модель. Например, указав

\begin{lstlisting}
module: hello
\end{lstlisting}
\noindent
мы сужаем множество всех возможных программ до одного программного модуля.

\begin{lstlisting}
function: main << ( добавить в module:hello )
\end{lstlisting}
\noindent
накладывает ограничение на то что целевая программа будет на \emc/\cpp.

\medskip
В mainstream языках программирования мы досконально описываем каждый элемент
процесса, пошагово, точно прописываем все свойства объектов и структуру
программы. При программировании на \metal\ мы постепенно \emph{накладываем
ограничения}, сужая все множество возможных программ с помощью условий.
При этом чем больше и больше ограничений мы применям, тем ближе концепт
программы становится ее моделью:
\begin{description}
\item[модель] точное поэлементное описание системы с указанием всех
взаимосвязей между объектами.
\item[концепт] более широкое понятие так как он преднамеренно \emph{описывает
систему только частично}, указывая что у объекта \emph{может быть} такое-то
свойство, есть такой-то класс с некоторыми методами (но при этом не
декларируются \emph{все методы}).
\end{description}

\noindent
Для реализации \term{концептуального программирования} \cite{tyugu} необходима
\term{база знаний} в которой задаются
\begin{itemize}
  \item 
\emph{наборы ограничений}, которые мы называем \term{концептами},
\item 
\emph{куски кода}\ --- \term{сниппеты}, которые добавляются в результирующий
код при срабатывании правил и ограничений, причем эти куски \emph{заданы
параметрически} и частично изменяются при использовании
 \item
предыдущие проекты, заданные как наборы ограничений\ --- вы можете
\term{наследовать} их целиком или частями
\item 
\emph{правила логического вывода} которые осуществляют синтез кода по сети
ограничений
\item 
\term{скрипты}\ --- \term{императивные} процедуры, выполняемые внутри базы
знаний, для ее анализа или трансформации
\item
\term{демоны}\ --- скрипты, выполнемые в фоновом режиме: оптимизации, сборка
мусора, JIT- и фоновая компиляция скриптов и вычислительных функций в машинный
код
\end{itemize} 
