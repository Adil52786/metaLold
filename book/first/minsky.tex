\secrel{Фреймы Мински}\label{minsky}

\begin{framed}\noindent\label{nolang}
\emph{\metal\ не рассматривает язык программирования как основной метод
разработки}: задача программиста\ --- строить \term{модели} решаемых задач в
памяти системы, формируя базу знаний с использованием \term{фреймов} Мински
\cite{minsky} для представления знаний.
\end{framed}

\term{Фрейм}\ в оригинальной формулировке понятия\ --- способ
\emph{структурного} \term{представления знаний} в искуственном интеллекте,
введенный Марвином Мински для описания струтуры знаний для восприятия
пространственных сцен \cite{minsky}. В той же книге это понятие было расширено
до универсальной структуры данных, очень близкой к понятию объекта в ООП.

\clearpage\noindent
Фреймы имеют именованные слоты (поля объекта), которые могут хранить
\begin{description}
\item[примитивные значения] (строки, числа)\note{в ИИ рассматриваются как
значения по умолчанию при описании ситуации}
\item[ссылки] на другие фреймы
\item[активные слоты] (методы объекта) содержат исполняемые процедуры,
запускаемые по внешним или внутренним событиям
\item[пустые слоты] не имеют заданного значения, и заполняются в процессе
логического вывода или работы процедур
\end{description} 

\medskip\noindent
Наличие исполняемых элементов превращает фреймовую модель\note{в общем случае\
--- любую структуру данных, см \ref{esd}} в \emph{модель вычислительную}, и
\term{парадигму программирования}, совмещающую в себе декларативное,
императивное, объектно-ориентированное и логическое программирование. Именно эта
универсальнось определила выбор фреймовой модели для \metal, как наиболее
подходящей для описания структур и процессов, специфичных для программного
обеспечения, и аппаратно-программных комплексов (embedded, \term{встраиваемые
системы}).

\vspace{7mm}
Фрейм Мински был адаптирован для представления исходного кода программ и
описания данных на любых языках (программирования). Для этого во фрейм была
добавлена возможность хранения элементов \emph{в упорядоченном виде}, что
автоматически делает \term{фреймовые сети} представлением \term{атрибутных
грамматик}. Также были добавлены дополнительные поля:

\lst{frames/frame.py}{language=Python}
\clearpage
