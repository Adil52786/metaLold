\secrel{Управляемая компиляция}\label{managed}

Кроме динамической компиляции \ref{dyna} работающей автоматически и только на
том же компьютере в составе интерпретатора, выжно выделить особый тип
компиляции\ --- \term{управляемая компиляция} \copyright. 

Управляемая компиляция нам особенно интересна тем, что она позволяет получать
машинный код для произвольного\note{в реальности мы ограничены архитектурами,
поддерживаемыми LLVM \ref{llvm}, но при большом желании вы можете написать
собственный кросс-компилятор для малораспространенного типа процессоров,
например для вашего собственного векторного фидонета реализованного на FPGA или
ASIC/БМК\ --- вы можете предоставить те же API, интерфейсы и структуры данных,
что и LLVM, или наоборот реализовать полностью свой управляемый
компилятор, использующий информацию из фреймовых структур и динамическую
трассировку в интерпретаторе \metal\ для PGO оптимизаций} процессора,
имеющего любую \term{целевую архитектуру}, т.е. выполнять
\term{кросс-компиляцию} программ \emph{для любых микроконтроллеров} и
встраиваемых компьютеров.

\begin{quotation}\noindent
\term{Управляемая компиляция} выполняется через прямые вызовы функций
\emph{компилятора, реализованного в виде библиотеки (модуля)} для произвольно
выбранного \term{инструментального языка программирования}. Управляемая
компиляция неразрывно связана с методами метапрограммирования, а конкретно
\textit{с генерацией низкоуровневого кода, управляемой явными вызовами
компилятора из (мета)программ пользователя}.
\end{quotation}